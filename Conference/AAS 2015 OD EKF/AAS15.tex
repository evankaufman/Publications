% use paper, or submit
% use 11 pt (preferred), 12 pt, or 10 pt only

%\documentclass[letterpaper, preprint, paper,11pt]{AAS}	% for preprint proceedings
%\documentclass[letterpaper, paper,11pt]{AAS}		% for final proceedings (20-page limit)
%\documentclass[letterpaper, paper,12pt]{AAS}		% for final proceedings (20-page limit)
\documentclass[letterpaper, paper,10pt]{AAS}		% for final proceedings (20-page limit)
%\documentclass[letterpaper, submit]{AAS}			% to submit to JAS

\usepackage{bm}
\usepackage{amssymb,amsmath,amsthm,times,graphicx,subfigure,tabularx,booktabs,colortbl,multirow,threeparttable}
\usepackage{subfigure}
%\usepackage[notref,notcite]{showkeys}  % use this to temporarily show labels
\usepackage[colorlinks=true, pdfstartview=FitV, linkcolor=black, citecolor= black, urlcolor= black]{hyperref}
\usepackage{overcite}
\usepackage{footnpag}			      	% make footnote symbols restart on each page

\usepackage{color}
\newcommand{\hilight}[1]{\colorbox{green}{#1}}

\newtheorem{prop}{Proposition}

\newcommand{\abs}{\mathop{\mathrm{abs}}\nolimits}
\newcommand{\diag}{\mathop{\mathrm{diag}}\nolimits}
\newcommand{\norm}[1]{\ensuremath{\left\| #1 \right\|}}
\newcommand{\bracket}[1]{\ensuremath{\left[ #1 \right]}}
\newcommand{\braces}[1]{\ensuremath{\left\{ #1 \right\}}}
\newcommand{\parenth}[1]{\ensuremath{\left( #1 \right)}}
\newcommand{\pair}[1]{\ensuremath{\langle #1 \rangle}}
\newcommand{\met}[1]{\ensuremath{\langle\langle #1 \rangle\rangle}}
\newcommand{\refeqn}[1]{(\ref{eqn:#1})}
\newcommand{\reffig}[1]{Fig. \ref{fig:#1}}
\newcommand{\tr}[1]{\mathrm{tr}\ensuremath{\negthickspace\bracket{#1}}}
\newcommand{\trs}[1]{\mathrm{tr}\ensuremath{[#1]}}
\newcommand{\ave}[1]{\mathrm{E}\ensuremath{[#1]}}
\newcommand{\deriv}[2]{\ensuremath{\frac{\partial #1}{\partial #2}}}
\newcommand{\SO}{\ensuremath{\mathsf{SO(3)}}}
\newcommand{\T}{\ensuremath{\mathsf{T}}}
\renewcommand{\L}{\ensuremath{\mathsf{L}}}
\newcommand{\so}{\ensuremath{\mathfrak{so}(3)}}
\newcommand{\SE}{\ensuremath{\mathsf{SE(3)}}}
\newcommand{\se}{\ensuremath{\mathfrak{se}(3)}}
\renewcommand{\Re}{\ensuremath{\mathbb{R}}}
\newcommand{\aSE}[2]{\ensuremath{\begin{bmatrix}#1&#2\\0&1\end{bmatrix}}}
\newcommand{\ase}[2]{\ensuremath{\begin{bmatrix}#1&#2\\0&0\end{bmatrix}}}
\newcommand{\D}{\ensuremath{\mathbf{D}}}
\renewcommand{\d}{\ensuremath{\mathfrak{d}}}
\newcommand{\Sph}{\ensuremath{\mathsf{S}}}
\renewcommand{\S}{\Sph}
\newcommand{\J}{\ensuremath{\mathbf{J}}}
\newcommand{\Ad}{\ensuremath{\mathrm{Ad}}}
\newcommand{\intp}{\ensuremath{\mathbf{i}}}
\newcommand{\extd}{\ensuremath{\mathbf{d}}}
\newcommand{\hor}{\ensuremath{\mathrm{hor}}}
\newcommand{\ver}{\ensuremath{\mathrm{ver}}}
\newcommand{\dyn}{\ensuremath{\mathrm{dyn}}}
\newcommand{\geo}{\ensuremath{\mathrm{geo}}}
\newcommand{\Q}{\ensuremath{\mathsf{Q}}}
\newcommand{\G}{\ensuremath{\mathsf{G}}}
\newcommand{\g}{\ensuremath{\mathfrak{g}}}
\newcommand{\Hess}{\ensuremath{\mathrm{Hess}}}

\newcommand{\x}{\ensuremath{\mathbf{x}}}
\renewcommand{\r}{\mathbf{r}}
\renewcommand{\u}{\mathbf{u}}
\newcommand{\y}{\mathbf{y}}

\newcommand{\bfi}{\bfseries\itshape\selectfont}

%\renewcommand{\baselinestretch}{1.2}


\PaperNumber{23-451}



\begin{document}

\title{Nonlinear Observability Measure for Relative Orbit Determination with Angles-Only Measurements}

\author{Evan Kaufman\thanks{Doctoral Student, Mechanical and Aerospace Engineering, George Washington University, 801 22nd St NW, Washington, DC 20052, Email: evankaufman@gwu.edu.},  
T. Alan Lovell\thanks{Research Aerospace Engineer, Air Force Research Laboratory, Space Vehicles Directorate, Kirtland AFB, NM.},
\ and Taeyoung Lee\thanks{Assistant Professor, Mechanical and Aerospace Engineering, George Washington University, 801 22nd St NW, Washington, DC 20052, Tel: 202-994-8710, Email: tylee@gwu.edu.}
\thanks{This research has been supported in part by NSF under the grants CMMI-1243000 (transferred from 1029551), CMMI-1335008, and CNS-1337722.}
}


\maketitle{} 		


\begin{abstract}
%An angles-only extended Kalman filter (EKF), based off of two-body motion, is developed to estimate the position of a Deputy satellite relative to a chief satellite.
A new nonlinear observability measure is proposed for relative orbit determination when lines-of-sight between satellites are measured only. It corresponds to a generalization of the observability Gramian in linear dynamic systems to the nonlinear relative orbit dynamics represented by the two-body problems. An extended Kalman filter (EKF) is adapted to this problem and is evaluated with various gravitational harmonics and initial orbital determination (IOD) predictions. Extensive results illustrate correspondence between the proposed observability measure with filtering errors. An extensive numerical analysis in realistic scenarios includes satellite propagation of the two-body problem the $J_2$ perturbation effects.
\end{abstract}


\section{Introduction}

Space-based surveillance or relative navigation is desirable for many spacecraft missions, such as formation control and rendezvous. Spacecraft maneuvers based only on on-board measurements reduce the total operating cost significantly, and it improves safety against communication interruptions with ground stations.  Relative navigation between spacecraft in close-proximity essentially corresponds to space-based orbit determination.  In particular, vision-based navigation and estimation of relative orbits have received attention recently, since optical sensors have the desirable properties of low cost and minimal maintenance, while providing accurate line-of-sight measurements, or angles from a chief to a deputy.

Relative navigation based on angles-only measurements has been investigated in~\cite{WofGelITAES09,Tom11,PatLovPASFMM12}.  The problem is to determine the relative orbit between a chief spacecraft and a deputy spacecraft by using the line-of-sight between the two objects, assuming that the orbit of the chief is prescribed exactly.  Reference \cite{WofGelITAES09} shows that the relative orbit is unobservable from angles-only measurements when linear relative orbital dynamics are assumed, unless there are thrusting maneuvers. Reference \cite{Tom11} investigates observability by using a relative orbit model linearized in terms of spherical coordinates.  Reference~\cite{PatLovPASFMM12} introduces the concept of partial observability to determine a basis vector representing a family of relative orbits, and an initial orbit determination technique is developed for this method.

All of these results are based on linear relative orbital dynamics. It is straightforward to see that the relative orbit is not observable with angles-only measurements through its linearized dynamics. In other words, there are an infinite number of relative orbits that yield the identical line-of-sight measurements, and the orbital distance between a deputy and a chief cannot be determined by angles-only measurements. As such, it is required to study the nonlinear relative orbital dynamics to determine observability with angles-only measurements. Linear observability analysis is performed numerically for a particular case in~\cite{YimCraPASMM04}. 

Recently, observability criteria have been derived for the nonlinear relative orbital dynamics represented by the solutions of the two-body problem without linearization~\cite{LovLeePISSFD14}.  Assuming that a chief is on a circular orbit with a prescribed orbital radius, nonlinear equations of motion for the relative orbital motion of a deputy with respect to the chief are presented.  A differential geometric method, based on the Lie derivatives of the line-of-sight from the chief to the deputy, is applied to obtain sufficient conditions for observability.  It is shown that under certain geometric conditions on the relative configuration between the chief and the deputy, the nonlinear relative motion is observable from angles-only measurements.  

However, this result does not provide any information of the degree of observability. The main contribution of this paper is proposing a new  measure of observability for relative orbit determination with angles-only measurements. For a given initial condition of relative orbits, the proposed observability measure determines how much the corresponding relative orbits are easy or difficult to estimate, thereby providing a quantitative measure for relative orbit determination.

Furthermore, this paper serves as a feasibility study of an extended Kalman filter (EKF) for orbit determination using angles-only measurements in realistic scenarios. The filter is subject to measurement uncertainty and perturbation forces not considered with the two-body problem, on which the filter is based.
%The EKF is simulated with various time steps to show how the filter performs on a wide range of systems with different sample rates.
Then, the performance of the EKF is compared with the proposed observability measure to show the correspondence. 




\section{Nonlinear Relative Orbital Dynamics}\label{sec:ND}

Consider two satellites orbiting around the Earth, where each satellite is modeled as a rigid body. Suppose that a \textit{chief} satellite is on a circular orbit with a pre-determined orbital radius of $a\in\Re$.

Define a local-vertical, local-horizontal (LVLH) frame as follows. Its origin is located at the chief satellite. The $x$-axis is along the radial direction from the Earth to the chief, and the $y$-axis is along the velocity vector of the chief. The $z$-axis is normal to the orbital plane, and it is parallel to the angular momentum vector of the chief. The LVLH frame is rotating with the angular velocity of $\mathbf{\omega}=[0,0,n]^T\in\Re^3$, where $n=\sqrt{\frac{\mu}{a^3}}$ is the mean motion of the chief satellite, and $\mu$ denotes the gravitational parameters of the Earth. Note that the velocity of the chief is given by $\mathbf{v}_{chief}=[0,na,0]^T\in\Re^3$. Let the relative position of a \textit{deputy} satellite with respect to the chief satellite be given by $\mathbf{r}=[x,y,z]^T\in\Re^3$ in the LVLH frame. 

\subsection{Nonlinear Equations of Motion}

Nonlinear equations of motion for the relative motion of the deputy with respect to the chief can be derived as follows based on Lagrangian mechanics~\cite{LovLeePISSFD14}. 

\paragraph{Lagrangian} 

Considering that the LVLH frame is rotating with the angular velocity of $\mathbf{\omega}=[0,0,n]^T$, the inertial velocity of the deputy satellite is given by
\begin{align*}
\mathbf{v}_{deputy}
=
\mathbf{v}_{chief} + \dot{\r} + \mathbf{\omega}\times \mathbf{r} = 
\begin{bmatrix}
0 \\ na \\ 0
\end{bmatrix}
+
\begin{bmatrix}
\dot x \\ \dot y \\ \dot z
\end{bmatrix}
+
\begin{bmatrix}
-n y \\ nx \\ 0
\end{bmatrix}
=
\begin{bmatrix}
\dot x - ny \\ \dot y + nx + na \\ \dot z
\end{bmatrix}.
\end{align*}
Therefore, the (normalized) kinetic energy of the deputy satellite is 
\begin{align*}
T = \frac{1}{2}\|\mathbf{v}_{deputy}\|^2 = \frac{1}{2} \braces{(\dot x -ny)^2+(\dot y + nx +na)^2 + \dot z^2}.
\end{align*}
The location of the Earth from the chief is given by $[-a,0,0]^T$ in the LVLH frame. Therefore, the position vector of the deputy from the center of the Earth is given by 
$\r_a=[x+a,y,z]^T\in\Re^3$. The gravitational potential energy is 
\begin{align*}
U = -\frac{\mu}{\sqrt{(x+a)^2 + y^2 + z^2}} = -\frac{\mu}{\|\r_a\|}.
\end{align*}
From the above equations, the Lagrangian of the deputy satellite is expressed in terms of $(x,y,z)$ as 
\begin{align}
L & = T-U = \frac{1}{2} \braces{(\dot x -ny)^2+(\dot y + nx +na)^2 + \dot z^2}
+\frac{\mu}{\sqrt{(x+a)^2 + y^2 + z^2}}.
\end{align}

\paragraph{Euler-Lagrange Equations}

Using the Euler-Lagrange equations, given by
\begin{align*}
\frac{d}{dt}\deriv{L}{\dot q}-\deriv{L}{q}=0,
\end{align*}
for $q\in\{x,y,z\}$, we obtain the nonlinear equations of motion for the relative orbit as follows.
\begin{align}
\ddot x - 2n\dot y -n^2 x&=n^2 a - \frac{\mu (x+a)}{((x+a)^2 + y^2 + z^2)^{3/2}},\label{eqn:ddotx}\\
\ddot y + 2n\dot x -n^2 y &=   - \frac{\mu y}{((x+a)^2 + y^2 + z^2)^{3/2}},\label{eqn:ddoty}\\
\ddot z &= - \frac{\mu z}{((x+a)^2 + y^2 + z^2)^{3/2}}.\label{eqn:ddotz}
\end{align}

These can be written as the standard form of the state equation,
\begin{align}
\dot{\mathbf{x}} = f (\mathbf{x}), \label{eqn:xxdot}
\end{align}
where the state vector is $\mathbf{x}=[x,y,z,\dot x, \dot y,\dot z]^T\in\Re^N$ with $N=6$, and
\begin{align}
f(\mathbf{x}) 
%= \begin{bmatrix}
%\dot x\\
%\dot y\\
%\dot z\\
%2n\dot y +n^2 (x+a) - \dfrac{\mu (x+a)}{((x+a)^2 + y^2 + z^2)^{3/2}}\\
%-2n\dot x +n^2 y  - \dfrac{\mu y}{((x+a)^2 + y^2 + z^2)^{3/2}}\\
%- \dfrac{\mu z}{((x+a)^2 + y^2 + z^2)^{3/2}}
%\end{bmatrix}
=\begin{bmatrix}
\dot \r \\
-2\omega\times \dot\r - \omega\times(\omega\times \r_a) -\dfrac{\mu \r_a}{\|\r_a\|^3}
\end{bmatrix}.\label{eqn:f}
\end{align}
%where $\r=[x,y,z]^T\in\Re^3$, $\r_a = [x+a,y,z]^T\in\Re^3$, and $\dot\r =[\dot x,\dot y,\dot x]^T\in\Re^3$. 

\subsection{Line-of-sight Measurement}

We assume that the line-of-sight from the chief to the deputy is measured by onboard sensing, such as optical sensors. The measurement is represented by the unit-vector of the relative position vector, i.e.,
\begin{align}
\y = h(\mathrm{x}) =\frac{\r}{\|\r\|},\label{eqn:y}
\end{align}
where $\y\in\Re^M$ with $M=3$, and it satisfies $\|\y\|=1$.

\section{Nonlinear Observability Criteria}\label{sec:OC}

Based on the nonlinear dynamic model presented at the previous section, here we analyze the observability of the relative orbit with line-of-sight measurements.
This section serves as a summary of~\cite{LovLeePISSFD14}.


\subsection{Observability Criteria for Nonlinear Systems}

Observability of nonlinear systems has been studied in~\cite{HerKreITAC77}, and it is summarized as follows. For a given nonlinear dynamic system described with \refeqn{xxdot} and \refeqn{y}, a pair of points $\x_0$ and $\x_1$ are called \textit{indistinguishable} if the outputs of the corresponding solutions starting from each of $\x_0$ and $\x_1$ are identical for a certain time period. The systems is \textit{locally weakly observable} at $\x_0$, if there exists an open neighborhood $V$ of $\x_0$ such that for every open neighborhood $U$ of $\x_0$ contained in $V$, the only indistinguishable point to $\x_0$ is the point $\x_0$ itself. 

The Lie-derivative of the output $h(\x)$ along $f(\x)$ is defined as follows:
\begin{align*}
L_f h(\x) = \deriv{h(\x)}{\x}f(\x)\in\Re^{M\times 1},
\end{align*}
which corresponds to the directional derivative of $h(\x)$ along $f(\x)$. For a non-negative integer $i$, the $i$-th order Lie-derivative is defined by induction as $L_f^i h = L_f (L_f^{i-1} h)$ with $L_f^0 h = h$. Define an observability matrix $\mathcal{O}\in\Re^{NM\times N}$ as
\begin{align*}
\mathcal{O}(\x_0) = \deriv{}{\x} \begin{bmatrix} L_f^0 h(\x) \\ L_f^1 h(\x)\\ \vdots \\L_f^{N-1} h(\x)\end{bmatrix}\bigg|_{\x=\x_0}.
\end{align*}
It has been shown that the system is locally weakly observable at $\x_0$ if the rank of the observability matrix $\mathcal{O}(\x_0) = N$. When applied to linear dynamics, this yields the well-known observability rank condition for linear systems. Note that when there is more than a single measurement, i.e., $M>1$, the observability rank condition can be satisfied without a need for computing the higher-order Lie derivatives up to the $N-1$-th order.

\subsection{Observability Criteria for Relative Orbital Dynamics}

\paragraph{Observability Matrix}
We apply the above observability criteria for the nonlinear relative orbital dynamics. Using the fact that $\x=[\r^T,\dot \r^T]$, the observability matrix of the relative orbital dynamics can be written as
\begin{align}
\mathcal{O} =
\begin{bmatrix}
\deriv{\y}{\r} & \deriv{\y}{\dot\r}\\
\deriv{\dot\y}{\r} & \deriv{\dot\y}{\dot\r}\\
\deriv{\ddot\y}{\r} & \deriv{\ddot\y}{\dot\r}
\end{bmatrix}
\triangleq
\begin{bmatrix}
\mathcal{O}_{00} & 0_{3\times 3}\\
\mathcal{O}_{10} & \mathcal{O}_{11}\\
\mathcal{O}_{20} & \mathcal{O}_{21}
\end{bmatrix}.\label{eqn:OO}
\end{align}
Here we consider the observability matrix obtained by up to the second order Lie derivatives of the measurement due to complexity. But, this still provides sufficient conditions for observability. 

After straightforward but tedious algebraic manipulations using the following identity repeatedly,
\begin{align*}
\delta \parenth{\frac{1}{\|\mathbf{r}\|^i}} 
& = -i\frac{ \mathbf{r}^T \delta \mathbf{r}}{\|\mathbf{r}\|^{i+2}},
\end{align*}
for any positive integer $i$, we can show that each of the sub-matrices $\mathcal{O}_{ij}$ of the observability matrix $\mathcal{O}$ is given by
\begin{align}
\mathcal{O}_{00} & = \deriv{\y}{\r} = \frac{1}{\|\r\|}(I - \y\y^T),\label{eqn:O_00}\\
\mathcal{O}_{10} & = \deriv{\dot\y}{\r} = \deriv{}{\r} \parenth{\deriv{\y}{\r}\dot \r}= -\frac{1}{\|\r\|^2}\braces{
{\dot\r\y^T +\y^T\dot\r I +\y\dot\r^T}
- 3\y\y^T\dot\r \y^T},\\
\mathcal{O}_{11} & = \deriv{\dot\y}{\dot\r} = \deriv{}{\dot\r} \parenth{\deriv{\y}{\r}\dot \r}= \deriv{\y}{\r} = \mathcal{O}_{00},\label{eqn:O11}\\
\mathcal{O}_{20} & = \deriv{\ddot\y}{\r} = -\dfrac{2\dot\r \dot\r^T  +\dot\r^T\dot\r I}{\|\r\|^3}
+3\dfrac{2(\r^T\dot\r) \dot\r\r^T +(\dot\r^T\dot\r)\r\r^T
+(\r^T\dot\r)^2 I
+2(\r^T\dot\r)\r  \dot\r^T
}{\|\r\|^5}
- 15 \dfrac{(\r^T\dot\r)^2\r  \r^T}{\|\r\|^7}\nonumber \\
&\quad -\dfrac{\ddot{\r}\r^T +\r^T\ddot{\r} I +\r \ddot{\r}^T}{\|\r\|^3}
+ 3 \dfrac{(\r^T\ddot{\r})\r \r^T}{\|\r\|^5}
+\parenth{\dfrac{I}{\|\mathbf{r}\|} - \dfrac{\mathbf{r}\mathbf{r}^T}{\|\mathbf{r}\|^3}}
\parenth{-[\omega]_\times^2 -\dfrac{\mu I_{3\times 3}}{\|\r_a\|^3} + \dfrac{3\mu\r_a\r_a^T}{\|\r_a\|^5}},\\
\mathcal{O}_{21} & = \deriv{\ddot\y}{\dot\r} = \deriv{}{\dot\r}\parenth{\deriv{}{\r}\parenth{\deriv{\y}{\r}\dot\r}\dot\r + \deriv{\y}{\r}\ddot\r} = 2 \deriv{}{\r} \parenth{\deriv{\y}{\r}\dot\r} + \deriv{\y}{\r} \deriv{\ddot\r}{\dot\r}
\nonumber\\
& \quad = 2\mathcal{O}_{10} 
- 2 \mathcal{O}_{00}[\omega]_\times,\label{eqn:O_21}
\end{align}
where $[\omega]_\times\in\Re^{3\times 3}$ is defined as
\begin{align*}
[\omega]_\times = \begin{bmatrix} 0 & -n & 0 \\ n & 0 & 0 \\ 0 & 0 & 0 \end{bmatrix}.
\end{align*}
These expressions were fully verified by the Matlab symbolic computation toolbox.

After some algebraic manipulations, we can show that the sub-matrices satisfy the following identities. 
\begin{align}
\mathcal{O}_{10}\r & = 
%-\frac{1}{\|\r\|}\braces{{\dot\r\y^T +\y^T\dot\r I +\y\dot\r^T}- 3\y\y^T\dot\r \y^T}\y\\
%&=-\frac{1}{\|\r\|}\braces{\dot\r +(\y^T\dot\r)\y +\y(\dot\r^T\y)- 3\y(\y^T\dot\r) }\\
%&=-\frac{1}{\|\r\|}\braces{\dot\r -\y(\y^T\dot\r) }=
-\mathcal{O}_{00}\dot\r,\label{eqn:O10r}\\
\mathcal{O}_{10}\dot\r & = -\frac{1}{\|\r\|^2}\braces{
{2\dot\r(\y^T\dot \r)  +\y(\dot\r^T\dot \r)}
- 3\y(\y^T\dot\r)^2},\label{eqn:O10dotr}\\
\mathcal{O}_{20}\r %& = 
%-\dfrac{2\dot\r \dot\r^T\y  +\dot\r^T\dot\r \y}{\|\r\|^2}
%+3\dfrac{2(\y^T\dot\r) \dot\r\y^T\y +(\dot\r^T\dot\r)\y\y^T\y+(\y^T\dot\r)^2 y +2(\y^T\dot\r)\y \dot\r^T\y}{\|\r\|^2}
%- 15 \dfrac{(\y^T\dot\r)^2\y  \y^T\y}{\|\r\|^2}\nonumber \\
%&\quad -\dfrac{\ddot{\r}\y^Ty +\y^T\ddot{\r} \y +\y \ddot{\r}^T\y}{\|\r\|^1}
%+ 3 \dfrac{(\y^T\ddot{\r})\y \y^T\y}{\|\r\|}
%+\parenth{\dfrac{I}{\|\mathbf{r}\|} - \dfrac{\mathbf{r}\mathbf{r}^T}{\|\mathbf{r}\|^3}}
%\parenth{-[\omega]_\times^2 -\dfrac{\mu I_{3\times 3}}{\|\r_a\|^3} + \dfrac{3\mu\r_a\r_a^T}{\|\r_a\|^5}}\r,\\
%&=
%-\dfrac{2\dot\r (\dot\r^T\y)  +(\dot\r^T\dot\r) \y}{\|\r\|^2}
%+3\dfrac{2(\y^T\dot\r) \dot\r +(\dot\r^T\dot\r)\y+(\y^T\dot\r)^2 \y +2(\y^T\dot\r)^2\y }{\|\r\|^2}
%- 15 \dfrac{(\y^T\dot\r)^2\y}{\|\r\|^2}\nonumber \\
%&\quad -\dfrac{\ddot{\r} +(\y^T\ddot{\r}) \y +\y (\ddot{\r}^T\y)}{\|\r\|^1}
%+ 3 \dfrac{(\y^T\ddot{\r})\y }{\|\r\|}
%+\parenth{\dfrac{I}{\|\mathbf{r}\|} - \dfrac{\mathbf{r}\mathbf{r}^T}{\|\mathbf{r}\|^3}}
%\parenth{-[\omega]_\times^2 -\dfrac{\mu I_{3\times 3}}{\|\r_a\|^3} + \dfrac{3\mu\r_a\r_a^T}{\|\r_a\|^5}}\r,\\
%&=
%-\dfrac{2\dot\r (\dot\r^T\y)  +(\dot\r^T\dot\r) \y}{\|\r\|^2}
%+3\dfrac{2(\y^T\dot\r) \dot\r +(\dot\r^T\dot\r)\y+(\y^T\dot\r)^2 \y +2(\y^T\dot\r)^2\y }{\|\r\|^2}
%- 15 \dfrac{(\y^T\dot\r)^2\y}{\|\r\|^2}\nonumber \\
%&\quad -\dfrac{\ddot{\r} +(\y^T\ddot{\r}) \y +\y (\ddot{\r}^T\y)}{\|\r\|^1}
%+ 3 \dfrac{(\y^T\ddot{\r})\y }{\|\r\|}
%+\parenth{\dfrac{I}{\|\mathbf{r}\|} - \dfrac{\mathbf{r}\mathbf{r}^T}{\|\mathbf{r}\|^3}}
%\parenth{-[\omega]_\times^2 -\dfrac{\mu I_{3\times 3}}{\|\r_a\|^3} + \dfrac{3\mu\r_a\r_a^T}{\|\r_a\|^5}}\r,\\
%&=
%\dfrac{4\dot\r (\dot\r^T\y)  -2(\dot\r^T\dot\r)\y-6(\y^T\dot\r)^2 \y }{\|\r\|^2}
% -\mathcal{O}_{00}\ddot \r
%+\parenth{\dfrac{I}{\|\mathbf{r}\|} - \dfrac{\mathbf{r}\mathbf{r}^T}{\|\mathbf{r}\|^3}}
%\parenth{-[\omega]_\times^2 -\dfrac{\mu I_{3\times 3}}{\|\r_a\|^3} + \dfrac{3\mu\r_a\r_a^T}{\|\r_a\|^5}}\r,\\
& = -2\mathcal{O}_{10}\dot\r +\mathcal{O}_{00}\braces{-\ddot \r +\deriv{\ddot\r}{\r}\r},\label{eqn:O20r}\\
\mathcal{O}_{21}\r & = -2\mathcal{O}_{00} (\dot\r+\omega\times\r),\label{eqn:O21r}\\
\mathcal{O}_{21}\dot\r & = 2\mathcal{O}_{10} \dot\r
- 2 \mathcal{O}_{00}[\omega]_\times\dot\r,\label{eqn:O21rdot}
\end{align}
which are useful to derive the observability criteria. 

\paragraph{Observability Rank Condition}



Now we present sufficient conditions that the observability matrix $\mathcal{O}$ has full rank. 

\begin{prop}
Define three vectors $\mathbf{v}_{rel}$, $\mathbf{a}_1,\mathbf{a}_2\in\Re^3$ as 
\begin{align}
\mathbf{v}_{rel} & = \dot\r +\omega\times\r,\label{eqn:vrel}\\
\mathbf{a}_1 & = \ddot\r-\deriv{\ddot\r}{\r}\r = -2\omega\times \dot\r - [\omega]_\times^2 a \mathbf{e}_1 -\dfrac{\mu a }{\|\r_a\|^3}\mathbf{e}_1- \dfrac{3\mu\r_a^T\r}{\|\r_a\|^5}\r_a,\label{eqn:a1}\\
\mathbf{a}_2 & = \ddot\r-\deriv{\ddot\r}{\r}\r -\deriv{\ddot\r}{\dot\r}\dot\r = \mathbf{a}_1+2\omega\times\dot\r= - [\omega]_\times^2 a \mathbf{e}_1 -\dfrac{\mu a }{\|\r_a\|^3}\mathbf{e}_1- \dfrac{3\mu\r_a^T\r}{\|\r_a\|^5}\r_a.\label{eqn:a2}
\end{align}
The nonlinear relative orbital dynamics are locally weakly observable at $\x=[\r^T,\dot\r^T]$ if
%\begin{list}{}{\setlength{\leftmargin}{1cm}\setlength{\itemsep}{0mm}\setlength{\parsep}{0mm}\setlength{\topsep}{0mm}\setlength{\parskip}{0mm}\setlength{\labelwidth}{2cm}}
%\item[(i)] when $\r\times\dot\r =0$, $\r\times\mathbf{a}_1\neq 0$ and $\r^T(\mathbf{v}_{ref}\times \mathbf{a}_1)\neq 0$,
%\item[(ii)] when $\r\times\dot\r \neq 0$, $\r\times\mathbf{a}_2\neq 0$ and $\r^T(\mathbf{v}_{ref}\times \mathbf{a}_2)\neq 0$.
%\end{list}
\begin{alignat}{2}
(i)&\; &\text{when $\r\times\dot\r =0$, $\r\times\mathbf{v}_{ref}\neq0$, $\r\times\mathbf{a}_1\neq 0$,  and $\r^T(\mathbf{v}_{ref}\times \mathbf{a}_1)\neq 0$},\label{eqn:cond1}\\
(ii)& &\text{when $\r\times\dot\r \neq 0$, $\r\times\mathbf{v}_{ref}\neq0$, $\r\times\mathbf{a}_2\neq 0$ and $\r^T(\mathbf{v}_{ref}\times \mathbf{a}_2)\neq 0$.}\label{eqn:cond2}
\end{alignat}
\end{prop}

\begin{proof}
We show that if the above conditions are satisfied, then the six columns of the observability matrix are linearly independent. Suppose that for a constant vector $\mathbf{c}=[\mathbf{c}_1^T,\mathbf{c}_2^T]\in\Re^6$, where $\mathbf{c}_1,\mathbf{c}_2\in\Re^3$, we have $\mathcal{O}\mathbf{c}=0$, i.e., 
\begin{align}
\mathcal{O} \mathbf{c} = \begin{bmatrix}
\mathcal{O}_{00} & 0_{3\times 3}\\
\mathcal{O}_{10} & \mathcal{O}_{11}\\
\mathcal{O}_{20} & \mathcal{O}_{21}
\end{bmatrix}
\begin{bmatrix} \mathbf{c}_1 \\ \mathbf{c_2} \end{bmatrix}
=
\begin{bmatrix}
\mathcal{O}_{00}\mathbf{c}_1\\
\mathcal{O}_{10}\mathbf{c}_1+ \mathcal{O}_{11}\mathbf{c}_2\\
\mathcal{O}_{20}\mathbf{c}_1+ \mathcal{O}_{21}\mathbf{c}_2
\end{bmatrix}
=
\begin{bmatrix}
0_{3\times 1}\\
0_{3\times 1}\\
0_{3\times 1}
\end{bmatrix}.\label{eqn:Oc}
\end{align}

We wish to show that there is no non-zero vector $\mathbf{c}$ satisfying \refeqn{Oc}. At the first three rows of \refeqn{Oc}, we have
\begin{align*}
\mathcal{O}_{00} \mathbf{c}_1 = \frac{1}{\|\r\|}(I-\y\y^T) \mathbf{c}_1 = 0.
\end{align*}
The matrix $\mathcal{O}_{00}$ has a one-dimensional null space spanned by $\y$. Therefore, without loss of generality, we can choose $\mathbf{c}_1=0$ or $\mathbf{c}_1=\r$. Suppose $\mathbf{c}_1=\r$ for the subsequent development.

For the chosen value of $\mathbf{c}_1=\r$, we find the next three rows of \refeqn{Oc} as
\begin{align}
\mathcal{O}_{10} \mathbf{c}_1 +\mathcal{O}_{11} \mathbf{c}_2 & = \mathcal{O}_{10} \r +\mathcal{O}_{11} \mathbf{c}_2  = \mathcal{O}_{00} (-\dot\r + \mathbf{c}_2)\nonumber\\
& = \frac{1}{\|\r\|}(I-\y\y^T)(-\dot\r + \mathbf{c}_2)=0,\label{eqn:row2}
\end{align}
where we have used \refeqn{O11} and \refeqn{O10r}. Next, we consider two cases of \refeqn{row2}, namely (i) when $\r\times\dot\r=0$, and (ii) when $\r\times\dot\r\neq 0$.

\paragraph{Case (i): $\r\times\dot \r=0$} In this case, $\dot \r$ can be written as $\dot \r = \alpha\r$ for some constant $\alpha$ as $\r$ is parallel to $\dot\r$. Then, \refeqn{row2} reduces to 
\begin{align*}
\frac{1}{\|\r\|}(I-\y\y^T)\mathbf{c}_2=0,
\end{align*}
which implies that $\mathbf{c}_2= c\r$ for an arbitrary constant $c$. For the given choice of $\mathbf{c}=[\r^T,c\r^T]^T$, the last three rows of \refeqn{Oc} are given by
\begin{align*}
\mathcal{O}_{20} \mathbf{c}_1 +\mathcal{O}_{21} \mathbf{c}_2 
& = \mathcal{O}_{20} \r + c\mathcal{O}_{21} \r.
\end{align*}
Using \refeqn{O20r} and \refeqn{O21r}, this can be rewritten as
\begin{align*}
\mathcal{O}_{20} \mathbf{c}_1 +\mathcal{O}_{21} \mathbf{c}_2 
& =
-2\mathcal{O}_{10}\dot\r +\mathcal{O}_{00}\braces{-\ddot \r +\deriv{\ddot\r}{\r}\r}
-2c\mathcal{O}_{00} (\dot\r+\omega\times\r).
\end{align*}
But, from \refeqn{O10dotr}, we can show that $2\mathcal{O}_{10}\dot\r=0$ when $\r$ is parallel to $\dot \r$. Using \refeqn{vrel} and \refeqn{a1}, this further reduces to
\begin{align}
\mathcal{O}_{20} \mathbf{c}_1 +\mathcal{O}_{21} \mathbf{c}_2 
& =
-\mathcal{O}_{00}(\mathbf{a}_1 +2c \mathbf{v}_{rel})\nonumber\\
& = - \frac{1}{\|\r\|}(I-\y\y^T)(\mathbf{a}_1 +2c \mathbf{v}_{rel})=0.\label{eqn:row3i}
\end{align}
The matrix $(I-\y\y^T)$ represents the orthogonal projection of a vector into the plane normal to $\y$. From the second condition of \refeqn{cond1}, namely $\r\times\mathbf{a}_1\neq 0$, we have $(I-\y\y^T)\mathbf{a}_1\neq 0$, which implies that the constant $c$ cannot be simply chosen as $c=0$. Therefore, the only possible case to satisfy the above equation is when $\mathbf{a}_1 +2c \mathbf{v}_{rel}$ is parallel to $\r$ for some values of $c$. However, that is not feasible since the third condition of \refeqn{cond1}, namely $\r^T(\mathbf{v}_{ref}\times\mathbf{a}_1)\neq0$,  implies that the three vectors $\r$, $\mathbf{a}_1$, and $\mathbf{v}_{rel}$ do not belong to a common plane, i.e., there is no constant $c$ such that $\mathbf{a}_1 +2c \mathbf{v}_{rel}$ is parallel to $\r$.

Therefore, there is no $\mathbf{c}\in\Re^6$ satisfying \refeqn{Oc} if $\mathbf{c}_1=\r$, under the given condition \refeqn{cond1}. This implies that $\mathbf{c}_1=0$. Substituting this back to \refeqn{Oc}, we have $\mathcal{O}_{11}\mathbf{c}_2=\mathcal{O}_{00}\mathbf{c}_2=0$, which follows that $\mathbf{c}_2=c\r$ for some constant $c$. However, when $\mathbf{c}_2=c\r$, the last three rows of \refeqn{Oc} are given by
\begin{align}
\mathcal{O}_{21}\mathbf{c}_2 & = 2c(\mathcal{O}_{10}-\mathcal{O}_{00}[\omega]_\times) \r \nonumber
\\
& = -2c\mathcal{O}_{00}\mathbf{v}_{rel}=0,\label{eqn:row3i1}
\end{align}
where \refeqn{O10r} is used. But, from the first condition of \refeqn{cond1}, we have $\mathcal{O}_{00}\mathbf{v}_{rel}\neq 0$, and therefore $c=0$, i.e., $\mathbf{c}_2=c\r =0$.

In short, under the given condition \refeqn{cond1}, the equation \refeqn{Oc} implies that $\mathbf{c}=0$. Therefore, the six columns of the observability matrix $\mathcal{O}$ are linearly independent. 

\paragraph{Case (ii): $\r\times\dot \r\neq0$} Next, we consider the second case of \refeqn{row2}. It implies that $-\dot\r + \mathbf{c}_2$ is parallel to $\r$, or equivalently, $\mathbf{c}_2 = \dot\r + c\r$ for an arbitrary constant $c$. For the given choice of $\mathbf{c}=[\r^T,\dot\r+ c\r^T]^T$, the last three rows of \refeqn{Oc} are given by
\begin{align*}
\mathcal{O}_{20} \mathbf{c}_1 +\mathcal{O}_{21} \mathbf{c}_2 
& = (\mathcal{O}_{20} \r+\mathcal{O}_{21}\dot\r) + c\mathcal{O}_{21} \r.
\end{align*}
From \refeqn{O20r}, \refeqn{O21r}, \refeqn{O21rdot}, this can be rewritten as
\begin{align*}
\mathcal{O}_{20} \mathbf{c}_1 +\mathcal{O}_{21} \mathbf{c}_2 
& = -\mathcal{O}_{00}(\mathbf{a}_2+2c\mathrm{v}_{ref})\\
& = -\frac{1}{\|\r\|}(I-\y\y^T)(\mathbf{a}_2+2c\mathbf{v}_{ref})=0.
\end{align*}
By following the same argument given after \refeqn{row3i}, under \refeqn{cond2}, there is no $c$ satisfying the above equation. 

This implies that $\mathbf{c}_1=0$. Then, by the same argument given at \refeqn{row3i1}, we have $\mathbf{c}_2=0$. In short, under the given condition \refeqn{cond2}, the equation \refeqn{Oc} implies that $\mathbf{c}=0$. Therefore, the six columns of the observability matrix $\mathcal{O}$ are linearly independent. 
\end{proof}

\paragraph{Remarks} We consider the cases where the given sufficient conditions \refeqn{cond1} and \refeqn{cond2} are violated. 
For both cases, we have $\r\times\mathbf{v}_{rel}\neq 0$. At \refeqn{vrel}, $\mathbf{v}_{rel}$ corresponds to the relative velocity observed in the inertial frame. Therefore, the given sufficient conditions for observability are violated when the relative velocity vector is parallel to the relative position vector in the inertial frame. 

The second condition of each of \refeqn{cond1} and \refeqn{cond2} is satisfied in general, as the expressions for $\mathbf{a}_1$ and $\mathbf{a}_2$ are relatively arbitrary at \refeqn{a1} and \refeqn{a2}. 

The third condition of \refeqn{cond1} implies that three vectors $\r$, $\mathbf{v}_{rel}$ and $\mathbf{a}_1$ do not belong to the same plane. The third condition of \refeqn{cond2} has a similar structure as well. This condition can be easily violated if the relative motion is planar, i.e., when $z(t)\equiv 0$ for all $t$. 

However, \refeqn{cond1} and \refeqn{cond2} are sufficient conditions for observability, and the fact that any of these condition is not satisfied does not necessarily mean that the considered point is not observable. In such case, the third or higher order Lie derivatives should be checked to determine observability. The main contribution of this section of the paper is showing that under certain geometric conditions, the nonlinear relative orbital dynamics are indeed observable via angles-only measurements.








































\section{Observability Measure of Nonlinear Systems}

The measure of observability has been generalized to nonlinear dynamic systems in order to obtain balanced realizations.
An energy-like observability function is introduced~\cite{SchSCL93,Sch94,NewKriPICDC98} to measure the degree of contribution of an initial state to the output, and it is applied to a pendulum system. But, this approach is only applicable to asymptotically stable equilibrium states. An empirical observability Gramian is introduced~\cite{LalMarPIWC99,LalMarIJRNC02}, which is essentially a covariance matrix of the output computed by a number of sample trajectories.
It has been applied to a balanced realization of chemical processes~\cite{HahEdgJPC03,HahEdgCCE02}. However, this approach is based on the assumption that the output converges to a steady-state value, and it is not clear how  sample trajectories are selected. 

A similar observability Gramian is introduced~\cite{KreIdePICDC09}, which is basically the Gramian for the sensitivity of the output with respect to the initial condition:
\begin{align}
\mathcal{W} (\x_0,t_0,t_f) = \int_{t_0}^{t_f} \parenth{\deriv{\y(\tau)}{\x_0}}^T\deriv{\y(\tau)}{\x_0}\,d\tau,\label{eqn:Wo_NL}
\end{align}
where $\y(t)$ denotes the output for the initial condition given by $\x(t_0)=\x_0$, and $\deriv{\y}{\x_0}\in\Re^{p\times n}$ is defined such that its $i,j$-th element is $\deriv{\y_i}{\x_{0_j}}$. 

In general, the Gramian of a set of time-dependent functions represents the degree of linear independence of those functions over a given period of time. Therefore, the observability Gramian defined at \refeqn{Wo_NL} measures how much the sensitivities of the output with respect to the initial condition are linearly independent of each other. This represents the degree of observability, since for example, if $\deriv{\y}{\x_{0_i}}$ is linearly dependent to $\deriv{\y}{\x_{0_j}}$, then it is difficult to distinguish the effects on $\delta \x_{0_i}$ to the output $\y$ from the effects of $\delta \x_{0_j}$ on $\y$. Therefore, the condition number of $\mathcal{W} (\x_0,t_0,t_f)$ can be used as a measure of observability. 

%Furthermore, we can easily show that \refeqn{Wo_NL} reduces to the observability Gramian for linear dynamic systems
%\begin{align}
%\mathcal{W}(t_0,t_f) = \int_{t_0}^{t_f}\Phi(\tau,t_0)^T C^T(\tau)  C(\tau) \Phi(\tau,t_0) d\tau,\label{eqn:Wo_Lin}
%\end{align}
%where $\Phi(t,\tau)$ is the state transition matrix, since the output is given by $\y(t) = \Phi(t,t_0)\x_0$. 

A numerical approach is proposed as follows. The $i,j$--th element of $\mathcal{W}_o$ can be approximated by
\begin{align}
[\mathcal{W}]_{ij} = \frac{1}{4\epsilon^2}\sum_{k=0}^N (\y^{i+}_k-\y^{i-}_k)(\y^{j+}_k-\y^{j-}_k) \Delta t_k,
\end{align}
where $\y^{i\pm}_k$ denotes the value of $\y^i$ at $t=t_k$ with the initial condition of $\x(t_0)=\x_0\pm \epsilon \mathbf{e}_i $ for a positive constant $\epsilon$, and the standard basis $\mathbf{e}_i$ of $\Re^n$.  The above expression can be computed numerically.






\section{Kalman Filter}

%Preliminary results between the filter performance of an extended Kalman filter and the nonlinear observability Gramian are shown. 

We consider the relative orbital dynamics between two satellites, where the observability of the deputy relative to the chief satellite is on a prescribed circular orbit when only $\mathbf{y}$ is assumed to be measured.
Then, an extended Kalman filter is developed for numerical analyses, where the satellites are either subject to two-body forces or with gravitational $J_2$ perturbations included. 


The initial conditions for an estimate of the state vector $\hat \x_0^+$ and its initial covariance matrix $P^+_0$ are assumed to be known from an initial orbit determination (IOD) technique.

\paragraph{Flow Update}
The flow update provides an a priori state estimate $\hat \x^-_k$ with covariance matrix $P^-_k$ based on the equations of motion.
The linearized state transition matrix from \refeqn{xxdot} and \refeqn{f} from time step $k-1$ to time step $k$ is written as
\begin{align}
F_{k-1}=\exp\left(
\begin{bmatrix}
0_{3\times3} & I_{3\times3} \\
\left(-[\omega]_\times^2-\frac{\mu I_{3\times3}}{\norm{\r_a}^3}+\frac{3\mu \r_a \r_a^T}{\norm{\r_a}^5}\right) & -2[\omega]_\times
%0 & 0 & 0 & 1 & 0 & 0\\
%0 & 0 & 0 & 0 & 1 & 0\\
%0 & 0 & 0 & 0 & 0 & 1\\
%3*n^2 & 0 & 0 & 0 & 2*n & 0\\
%0 & 0 & 0 & -2*n & 0 & 0\\
%0 & 0 & -n^2 & 0 & 0 & 0
\end{bmatrix}
(t_k-t_{k-1})\right),
\end{align}
such that
\begin{align}
\hat \x_{k}^-=F_{k-1}\hat \x_{k-1}^+.
\end{align}
The process covariance matrix $Q_k$ is assumed known.
The a priori covariance matrix is defined as
\begin{align}
P^-_k=F_{k-1}P^+_{k-1}F_{k-1}^T+Q_k.
\end{align}

\paragraph{Measurement Update}
The effect of the measurement $\y_k$ with covariance matrix $R_k$ is included to determine the posterior state estimate $\hat \x^+_k$ and $P^+_k$.
The linearized measurement matrix $H_k$ is derived from \refeqn{y},
\begin{align}
H_k=
\begin{bmatrix}
\left(\frac1{\norm{\hat \r_{k}^-}}I_{3\times3}-\frac{\hat \r_{k}^-{\hat \r_{k}^{-T}}}{\norm{\hat \r_{k}^-}^3}\right) & 0_{3\times3}
\end{bmatrix},
\end{align}
where $\hat \r_{k}^-$ is the a priori estimate of $\r$ given by the first three elements of $\hat \x_{k}^-$.
Based on \refeqn{y}, the measurement prediction is given by
\begin{align}
\hat{\y}_k=h(\hat \r_{k}^-)=\frac{\hat \r_{k}^-}{\norm{\hat \r_{k}^-}}.
\end{align}
The posterior state estimate and its covariance matrix are given by
\begin{align}
\x^+_k&=\x^-_k+K_k(\y_k-\hat{\y}_k),
\\
P^+_k&=(I_{6\times6}-K_kH_k)P^-_k,
\end{align}
such that $K_k$ is chosen to minimize $\tr{P^+_k}$ with
\begin{align}
K_k=P^-_kH_k^T(H_kP^-_kH_k+R_k)^{-1}.
\end{align}

%The initial conditions of the deputy satellite are chosen as
%\begin{align*}
%\mathbf{r}_0=[r_{x_0},\, 0,\, 0]^T\,\mathrm{km},\quad \mathbf{v}_0=[0,\, -2r_{x_0}n,\, 0]^T\,\mathrm{km/s},
%\end{align*}
%for a varying $r_{x_0}$. These yield a series of elliptic orbits around the chief when applied to the linearized relative orbital dynamics. The state estimation error is represented by a time averaged normalized magnitude error, $e_{mag} = \frac{\|\mathbf{x}-\hat{\mathbf{x}}\|}{\|\mathbf{x}\|}$, where $\mathbf{x}=[\mathbf{r}^T,\dot{\mathbf{r}}^T]^T\in\Re^6$ denotes the true state and $\hat{\mathbf{x}}$ denotes the estimated state. 
%
%The normalized magnitude error $e_{mag}$ and the condition number of the proposed observability measure $\mathrm{cond}\braces{\mathcal{W}}$ are computed among various values for $r_{x_0}$, tabulated below.
%
%\hspace*{0.1cm}\centerline{
%\begin{tabular}{ccc}\hline
%$r_{x_0}$ & $e_{mag}$ & $\mathrm{cond}\braces{\mathcal{W}}$\\\hline
%\vspace*{-0.02\columnwidth}\\
%$1$ & $11.5962$ &  $10^{16.6653}$\\
%$2$ & $9.6374$ & $10^{16.8323}$\\
%$5$ & $4.0041$ & $10^{16.3535}$\\
%$10$ & $2.6318$ &  $10^{15.8194}$\\
%$20$ & $0.9250$ &  $10^{15.2832}$\\
%$50$ & $0.5315$ &  $10^{14.4052}$\\
%$100$ & $0.1675$ &  $10^{12.9884}$\\
%\hline
%\end{tabular}
%}
%
%These preliminary results illustrate that there exists a certain correspondence between the filter error and the proposed observability measure, and therefore, it can be used as a quantitative measure for strength of observability.


\section{Numerical Results}

We choose numerous cases of relative motion between chief and deputy satellite orbits.
With each case, we evaluate the observability measure and process angles-only measurements with the EKF, using both the two-body equations of motion and higher fidelity propagation with $J_2$ effects as the truth models.
%In particular, the $J_2$ gravitational harmonics will be included in the satellite propagation, so the EKF must handle simulated process uncertainty.
To begin each simulation, an initial orbit determination (IOD) estimate is determined by various angles-only observations.
These IOD estimates serve as initial estimates for the EKF.
Then, we evaluate the performance of the EKF for these scenarios, specifically focusing on the observability measure and issues arising from initial conditions.

Two IOD methods are developed in~\cite{NewLovPra14,PraLovNew14}, referred to as \emph{IOD1} and \emph{IOD2}, and employed to provide initial conditions for $13$ cases, outlined in Table \ref{tab:IODerr}.
%Two IOD methods are referred to as \emph{IOD1} and \emph{IOD2}, and employed to provide initial conditions for $13$ cases, outlined in Table \ref{tab:IODerr}.
These cases yield relative orbit trajectories illustrated in Figures \ref{fig:0CasesTraj}, \ref{fig:1CasesTraj}, \ref{fig:2CasesTraj}, \ref{fig:3CasesTraj}, and \ref{fig:4CasesTraj}.
The most important distinction between the various cases is that in Case $\mathbf{1}$ the two-body solution and the linearized dynamics of the Hill-Clohessy-Wiltshire (HCW) solution are very close and the trajectory remains largely planar with a circular shape.

\begin{figure}[h]
\centerline{
	\subfigure[Case 0A]{
		\includegraphics[width=0.3\textwidth]{OM_IOD_0A_traj.pdf}}
	\hfill
	\subfigure[Case 0B]{
		\includegraphics[width=0.3\textwidth]{OM_IOD_0B_traj.pdf}}
	\hfill
	\subfigure[Case 0C]{
		\includegraphics[width=0.3\textwidth]{OM_IOD_0C_traj.pdf}}
}
\caption{The trajectories of the Cases 0A, 0B, and 0C (solid red: HCW, dashed blue: two-body solution) show that the linear and nonlinear solutions are close.
The resulting trajectories are not cyclic as the trajectories change in the $y$-direction with time.}
\label{fig:0CasesTraj}
\end{figure}

\begin{figure}[h]
\label{fig:CaseTraj}
\centerline{
	\subfigure[Case 1]{
		\includegraphics[width=0.3\textwidth]{OM_IOD_1_traj.pdf}}
}
\caption{The trajectory of the Case 1 (solid red: HCW, dashed blue: two-body solution) shows that the linear and nonlinear solutions are very close, where both solutions are largely planar and cyclic.}
\label{fig:1CasesTraj}
\end{figure}


\begin{figure}[h]
\centerline{
	\subfigure[Case 2A]{
		\includegraphics[width=0.3\textwidth]{OM_IOD_2A_traj.pdf}}
	\hfill
	\subfigure[Case 2B]{
		\includegraphics[width=0.3\textwidth]{OM_IOD_2B_traj.pdf}}
	\hfill
	\subfigure[Case 2C]{
		\includegraphics[width=0.3\textwidth]{OM_IOD_2C_traj.pdf}}
}
\caption{The trajectories of the Cases 2A, 2B, and 2C (solid red: HCW, dashed blue: two-body solution) show differences between the linear and nonlinear solutions that vary about the $y$-axis, increasing the observabilities.}
\label{fig:2CasesTraj}
\end{figure}

\begin{figure}[h]
\centerline{
	\subfigure[Case 3A]{
		\includegraphics[width=0.3\textwidth]{OM_IOD_3A_traj.pdf}}
	\hfill
	\subfigure[Case 3B]{
		\includegraphics[width=0.3\textwidth]{OM_IOD_3B_traj.pdf}}
	\hfill
	\subfigure[Case 3C]{
		\includegraphics[width=0.3\textwidth]{OM_IOD_3C_traj.pdf}}
}
\caption{The trajectories of the Cases 3A, 3B, and 3C (solid red: HCW, dashed blue: two-body solution) exhibit large observabilities due to nonlinear solutions that move far away from the linearized solutions.}
\label{fig:3CasesTraj}
\end{figure}

\begin{figure}[h]
\centerline{
	\subfigure[Case 4A]{
		\includegraphics[width=0.3\textwidth]{OM_IOD_4A_traj.pdf}}
	\hfill
	\subfigure[Case 4B]{
		\includegraphics[width=0.3\textwidth]{OM_IOD_4B_traj.pdf}}
	\hfill
	\subfigure[Case 4C]{
		\includegraphics[width=0.3\textwidth]{OM_IOD_4C_traj.pdf}}
}
\caption{The trajectories of the Cases 4A, 4B, and 4C (solid red: HCW, dashed blue: two-body solution) experience the largest observabilities due to nonlinear solutions that move farther away from the linearized solutions than any of the trajectories from other cases considered in this paper.}
\label{fig:4CasesTraj}
\end{figure}



The initial error functions between the true orbit and and IOD predictions are defined as
\begin{align}
e_{mag,0}=\abs\left(1-{\frac{\norm{\r^+_0}}{\norm{\r_0}}}\right), \quad e_{dir,0}=\cos^{-1}\left(\frac{\r_0}{\norm{\r_0}}\cdot \frac{\r^+_0}{\norm{\r^+_0}}\right).
\end{align}

\begin{center}
\begin{threeparttable}[h]
\caption{Case Observabilities and Initial Errors}
\begin{tabularx}{0.72\textwidth}
{
>{$}c<{$}
*{1}{>{$}c<{$}} |
*{2}{>{$}c<{$}} |
*{2}{>{$}c<{$}}
}
\toprule
\multirow{2}{*}{Case} & \multirow{2}{*}{$\mathrm{cond}\braces{\mathcal{W}}$} & \multicolumn{2}{c}{\multirow{1}{*}{IOD1}} & \multicolumn{2}{c}{\multirow{1}{*}{IOD2}} \\
& &  e_{mag,0} & e_{dir,0} & e_{mag,0} & e_{dir,0} \\\midrule
0A & 10^{11.1486} & 0.0003 & 1.4901\times10^{-8} & 0.0012  & 2.1073\times10^{-8} \\
0B & 10^{10.9387} & 0.0233 & 0 & 0.0088 & 0 \\
0C & 10^{10.9851} & 0.0152 & 2.1073\times10^{-8} & 0.0139 & 1.4901\times10^{-8}  \\
\midrule
\mathbf{1} & 10^{16.6871} & 0.0001 & 0 & 0 & 0  \\
\midrule
2A & 10^{10.6492} & 0.0550 & 1.8150\times10^{-4} & 0.0238 & 1.8150\times10^{-4}  \\
2B & 10^{10.4163} & 0.9571 & 0.0734                     & 0.9571 & 0.0734  \\
2C & 10^{10.2155} & 0.0219 & 0.0044 & 0.0065 & 0.0044 \\
\midrule
3A & 10^{8.3490} & 0.0191 & 0.0031 & 0.1830 & 0.0031  \\
3B & 10^{8.5372} & 0.0612 & 0.0533 & 0.2315 & 0.0533  \\
3C & 10^{8.6586} & 6.8962 & 0.7348 & 5.4921 & 0.7348  \\
\midrule
4A & 10^{7.9062} & 8.7347 & 0.4587 & 0.1575 & 0.4587  \\
4B & 10^{7.9903} & 0.9276 & 0.1819 & 0.0621 & 0.1819  \\
4C & 10^{8.0645} & 0.1161 & 0.0554 & 0.0354 & 0.0554  \\
\bottomrule
\end{tabularx}
{\small
\begin{tablenotes}
    \item Note: $0$ indicates numbers small enough to be below Matlab numerical accuracy.
  \end{tablenotes}}
\label{tab:IODerr}
\end{threeparttable}
\end{center}


The Kalman filter is used in each case. The initial state covariance matrix is chosen as
\begin{align*}
P_0^+=\diag[50^2,\ 50^2,\ 50^2,\ (50n)^2,\ (50n)^2,\ (50n)^2],
\end{align*}
and the process and measurement covariance matrices are chosen as
\begin{align*}
Q_k=\diag[10^{-7},\ 10^{-7},\ 10^{-7},\ 10^{-9},\ 10^{-9},\ 10^{-9}], \quad R_k=\tan^2{2^\circ}I_{3\times3}\ \forall\ k.
\end{align*}
%\begin{align*}
%Q_k=10^q\diag[10^{-8},\ 10^{-8},\ 10^{-8},\ 10^{-10},\ 10^{-10},\ 10^{-10}], \quad R_k=\tan^2{2^\circ}I_{3\times3}\ \forall\ k
%\end{align*}
%where $q\in[-1,\ 0,\ 1,\ 2]$ is varied in the simulations, thereby changing the impact of measurement updates inside the Kalman filter.

Each case is simulated for every IOD estimate. The running time is $59540$ seconds (slightly more than $\frac23$ of a day) with time steps of $10$ seconds between measurements. The following RMS errors are collected from each trial:
\begin{align}
\bar e_{mag}=\left(\frac1N\sum_{k=1}^N\left\{1-{\frac{\norm{\r^+_k}}{\norm{\r_k}}}\right\}^2\right)^{\frac12}, \quad \bar e_{dir}=\left(\frac1N\sum_{k=1}^N
\left\{\cos^{-1}\left(\frac{\r_k}{\norm{\r_k}}\cdot \frac{\r^+_k}{\norm{\r^+_k}}\right)\right\}^{2}
\right)^{\frac12},
\end{align}
for $N$ time steps.
These metrics are tabulated in Tables \ref{tab:ResultsTBP} and \ref{tab:ResultsJ2}.
Over the simulated time period, figures were generated for these analyses that  serve to compare the true and estimated state magnitudes and the maximum eigenvalues of the state covariance matrices as a measure of uncertainty for the state estimates.
Because of paper constraints, a sample of these plots are shown that highlight various aspects of the observability and filter performance.



%The figure below illustrates preliminary results for angles-only relative orbit estimation with $J_2$ perturbation in the measurements.
%Note that the EKF performs well in this particular case, as evidenced by the close agreement between the true and estimated trajectory (Figure 1a) and the decrease in the maximum eigenvalue of the covariance matrix over time (Figure 1b).
%For the various cases to be chosen, we will evaluate the performance of the EKF (using metrics such as those illustrated in Figure 1) and seek a consistent correlation between the filter performance and the new observability measure proposed above.



%\begin{center}
%\begin{threeparttable}[h]
%\caption{Error variables for EKF with varying $Q_k$ Considering Only Two-Body Problem Forces}
%\begin{tabularx}{\textwidth}
%{
%>{$}c<{$}
%*{1}{>{$}c<{$}} |
%*{4}{>{$}c<{$}} |
%*{4}{>{$}c<{$}}
%}
%% & \bar e_{mag} & \bar e_{dir}
%\toprule
%\multirow{2}{*}{Case} & \multirow{2}{*}{IOD} & \multicolumn{4}{c}{$\bar e_{mag}$} & \multicolumn{4}{c}{$\bar e_{dir}$} \\
%& & q=-1 & q=0 & q=1 & q=2 & q=-1 & q=0 & q=1 & q=2 \\\midrule
%0A & IOD 1 & 0.0895 &  0.0611  &  0.3766  &  0.2861  &  0.0430  &  0.0135  &  0.0144  &  0.0183 \\
%0A & IOD 2 & 0.1793 &  0.2811  &  0.5145  &  0.7716  &  0.0203  &  0.0313  &  0.0691  &  0.1046 \\
%0B & IOD 1 & 0.3241 &  0.0531  &  0.5186  &  0.0620  &  0.1695  &  0.0143  &  0.0810  &  0.0172 \\
%0B & IOD 2 & 0.2330 &  0.2115  &  0.3092  &  0.2733  &  0.1007  &  0.0879  &  0.0804  &  0.1048 \\
%0C & IOD 1 & 0.0973 &   0.0748 &   0.2723 &   0.0607 &   0.0292 &   0.0068  &  0.1362 &   0.0443 \\
%0C & IOD 2 & 0.1473 &   0.1017 &   0.1707 &   0.3545  &  0.0702  &  0.0517  &  0.0614  &  0.1224 \\
%\midrule
%\mathbf{1} & IOD 1 & \hilight{8.922} &   \hilight{2.964}   & \hilight{4.078}  &  \hilight{3.571} &   0.0071  &  0.0089  &  0.0109  &  0.0418 \\
%\mathbf{1} & IOD 2 & \hilight{13.635} &  \hilight{14.121} &  \hilight{42.486} &  \hilight{42.573} &   0.0113 &   0.0107  &  0.0445  &  0.0115 \\
%\midrule
%2A & IOD 1 & 0.2045 &   0.0629 &   0.1592  &  0.1329  &  0.0041 &   0.0041  &  0.0043  &  0.0039 \\
%2A & IOD 2 & 0.1013 &   0.1797  &  0.1348  &  0.2247  &  0.0050 &   0.0048 &   0.0065 &   0.0061 \\
%2B & IOD 1 & \hilight{0.9886} &   \hilight{0.9895}   & \hilight{0.9843}  &  \hilight{0.9851}  &  0.0114 &   0.0116  &  0.0114  &  0.0116 \\
%2B & IOD 2 & \hilight{0.9557} &   \hilight{0.9685} &   \hilight{0.9278}  &  \hilight{0.9431} &   0.0105  &  0.0108 &   0.0113 &   0.0153 \\
%2C & IOD 1 & 0.0219 &   0.0184  &  0.0258  &  0.0238 &   0.0026 &   0.0031  &  0.0044 &   0.0038 \\
%2C & IOD 2 & 0.0303 &   0.0287  &  0.0360  &  0.0470  &  0.0044  &  0.0054  &  0.0059 &   0.0062 \\
%\midrule
%3A & IOD 1 &  0.2377 &   0.1684  &  0.0486 &   0.0695 &   0.0094 &   0.0070 &   0.0040 &  0.0039 \\
%3A & IOD 2 &  0.0854  &  0.0874 &   0.0603  &  0.0544  &  0.0035 &   0.0033  &  0.0053  &  0.0042 \\
%3B & IOD 1 &  0.0724 &   0.0260 &   0.0680  &  0.0289  &  0.0035 &   0.0029  &  0.0033  &  0.0027 \\
%3B & IOD 2 &  0.0385  &  0.0432  &  0.1459  &  0.0284  &  0.0038  &  0.0037  &  0.0070  &  0.0047 \\
%3C & IOD 1 &  \hilight{2.2433}  &  \hilight{1.5231} &   \hilight{2.0611} &   \hilight{1.5110}  &  0.1803  &  0.1172  &  0.1595  &  0.1103 \\
%3C & IOD 2 &  \hilight{4.0635} &   \hilight{1.7717}  & \hilight{10.567}  &  \hilight{4.0277} &   0.4475 &   0.1506 &   0.9011 &   0.4028 \\
%\midrule
%4A & IOD 1 &  \hilight{6.0259} &   \hilight{0.0464}  &  \hilight{6.0706}  &  \hilight{0.0325}  &  1.3353  &  0.0087   & 1.3182  &  0.0065 \\
%4A & IOD 2 &  6.3646 &   0.0173  & 11.8882 &   0.0256 &   1.3423  &  0.0064  &  1.5319 &   0.0066 \\
%4B & IOD 1 &  \hilight{0.2922}  &  \hilight{0.0258}  &  \hilight{0.2723}  &  \hilight{0.0182}   & 0.0929 &   0.0032  &  0.0789  &  0.0036 \\
%4B & IOD 2 &  0.1815  &  0.0217  &  0.3155 &   0.0771  &  0.0268   & 0.0034  &  0.1061 &   0.0044 \\
%4C & IOD 1 &  0.0247  &  \hilight{0.0421}  &  2.3699  &  0.0182  &  0.0032  &  0.0032  &  0.8486 &   0.0028 \\
%4C & IOD 2 &  0.0200 &   0.0128  &  0.0278  &  0.0299 &   0.0040  &  0.0029  &  0.0034  &  0.0033 \\
%\bottomrule
%\end{tabularx}
%{\small
%\begin{tablenotes}
%    \item \hilight{Filter diverges}
%  \end{tablenotes}}
%\label{tab:ResultsTBP}
%\end{threeparttable}
%\end{center}
%
%\begin{center}
%\begin{threeparttable}[h]
%\caption{Error variables for EKF with varying $Q_k$ with $J_2$ Perturbations Included}
%\begin{tabularx}{\textwidth}
%{
%>{$}c<{$}
%*{1}{>{$}c<{$}} |
%*{4}{>{$}c<{$}} |
%*{4}{>{$}c<{$}}
%}
%\toprule
%\multirow{2}{*}{Case} & \multirow{2}{*}{IOD} & \multicolumn{4}{c}{$\bar e_{mag}$} & \multicolumn{4}{c}{$\bar e_{dir}$} \\
%& & q=-1 & q=0 & q=1 & q=2 & q=-1 & q=0 & q=1 & q=2 \\\midrule
%0A & IOD 1 & 0.0553  &  0.1105  &  0.1918  &  0.2298  &  0.0059  &  0.0062  &  0.0069  &  0.0069 \\
%0A & IOD 2 & 0.2177  &  0.3227  &  0.3140  &  0.5013  &  0.0082  &  0.0074  &  0.0129  &  0.0085 \\
%0B & IOD 1 & 0.0473  &  0.1246  &  0.7262  &  0.0655  &  0.0077  &  0.0088  &  0.0505  &  0.0066\\
%0B & IOD 2 & 0.8872  &  0.4186  &  0.1571  &  0.1561  &  0.0390  &  0.0168  &  0.0086  &  0.0293 \\
%0C & IOD 1 & 0.0652  &  0.0809  &  0.0900  &  0.1113  &  0.0082  &  0.0079  &  0.0061  &  0.0106 \\
%0C & IOD 2 & 0.0367  &  0.0204  &  0.1002  &  0.7253  &  0.0145  &  0.0084  &  0.0154  &  0.0579 \\
%\midrule
%\mathbf{1} & IOD 1 & \hilight{9.1098}  &  \hilight{2.900}  &  \hilight{4.059}  &  \hilight{3.567}  &  0.0071  &  0.0090  &  0.0109  &   0.0418 \\
%\mathbf{1} & IOD 2 & \hilight{13.594} & \hilight{14.033} & \hilight{42.381} & \hilight{42.979}  & 0.0113   & 0.0107   & 0.0445   &   0.0115 \\
%\midrule
%2A & IOD 1 & 0.2301  &  0.0423  &  0.1978  &  0.1740  &  0.0048  &  0.0049  &  0.0047  &  0.0042 \\
%2A & IOD 2 & 0.1058  &  0.2164  &  0.1638  &  0.2644  &  0.0051  &  0.0049  &  0.0065  &  0.0061 \\
%2B & IOD 1 & \hilight{0.9895}  &  \hilight{0.9902}  &  \hilight{0.9853}  &  \hilight{0.9859}  &  0.0115  &  0.0117  &  0.0115  &  0.0117 \\
%2B & IOD 2 & \hilight{0.9585}  &  \hilight{0.9701}  &  \hilight{0.9307}  &  \hilight{0.9455}  &  0.0106  &  0.0109  &  0.0114  &  0.0153 \\
%2C & IOD 1 & 0.0481  &  0.0474  &  0.0563  &  0.0415  &  0.0041  &  0.0042  &  0.0047  &  0.0042 \\
%2C & IOD 2 & 0.0535  &  0.0397  &  0.0536  &  0.0705  &  0.0044  &  0.0056  &  0.0059  &  0.0062 \\
%\midrule
%3A & IOD 1 &  0.2462  &  0.1827  &  0.0516  &  0.0699 &   0.0097 &   0.0077  &  0.0041 &   0.0041 \\
%3A & IOD 2 &  0.1038  &  0.0986  &  0.0643  &  0.0692 &   0.0036  &  0.0034  &  0.0053  &  0.0041 \\
%3B & IOD 1 &  0.0770  &  0.0310  &  0.0739  &  0.0458 &   0.0038 &   0.0034  &  0.0035  &  0.0031 \\
%3B & IOD 2 &  0.0461  &  0.0585   & 0.1485 &   0.0391  &  0.0039  &  0.0040  &  0.0071  &  0.0047 \\
%3C & IOD 1 &  \hilight{2.239}  &  \hilight{1.522}  &  \hilight{2.063}  &  \hilight{1.514}  &  0.1817  &  0.1179  &  0.1614  &  0.1116 \\
%3C & IOD 2 &  \hilight{4.188}  &  \hilight{1.821}  & \hilight{10.136}  &  \hilight{4.510} &   0.4873 &   0.1592  &  0.9195 &   0.5532 \\
%\midrule
%4A & IOD 1 &  \hilight{6.013}  &  \hilight{0.0454}  &  \hilight{6.359}  &  \hilight{0.031}  &  1.3131  &  0.0086  &  1.4235 &   0.0066 \\
%4A & IOD 2 &  6.3272  &  0.0197  & 12.0062 &  0.0248  &  1.3116  &  0.0065  &  1.5310  &  0.0066 \\
%4B & IOD 1 &  \hilight{0.2936}  &  \hilight{0.0272}  &  \hilight{0.2773}  &  \hilight{0.0217}  &  0.0935  &  0.0032  &  0.0814  &  0.0038 \\
%4B & IOD 2 &  0.1853  &  0.0210  &  0.5244  &  0.0785  &  0.0293  &  0.0036  &  0.2305  &  0.0045 \\
%4C & IOD 1 &  0.0260  &  \hilight{0.0410}  &  2.0105  &  0.0164  &  0.0034  &  0.0031  &  0.9144  &  0.0030 \\
%4C & IOD 2 &  0.0210  &  0.0145  &  0.0303  &  0.0349  &  0.0042  &  0.0030  &  0.0036  &  0.0034 \\
%\bottomrule
%\end{tabularx}
%{\small
%\begin{tablenotes}
%    \item \hilight{Filter diverges}
%  \end{tablenotes}}
%\label{tab:ResultsJ2}
%\end{threeparttable}
%\end{center}

\begin{center}
\begin{threeparttable}[h]
\caption{Error variables for EKF Considering Only Two-Body Problem Forces}
\begin{tabularx}{.58\textwidth}
{
>{$}c<{$}
*{1}{>{$}c<{$}} |
*{2}{>{$}c<{$}} |
*{2}{>{$}c<{$}}
}
\toprule
\multirow{2}{*}{Case} & \multirow{2}{*}{$\mathrm{cond}\braces{\mathcal{W}}$} & \multicolumn{2}{c}{IOD1} & \multicolumn{2}{c}{IOD2} \\
& & \bar e_{mag} & \bar e_{dir} & \bar e_{mag} & \bar e_{dir} \\\midrule
0A & 10^{11.1486} &  0.3766  &  0.0144  &  0.5145  &  0.0691 \\
0B & 10^{10.9387} &  0.5186  &  0.0810  &  0.3092  &  0.0804\\
0C & 10^{10.9851} &  0.2723  &  0.1362 &  0.1707 &  0.0614\\
\midrule
\mathbf{1} & 10^{16.6871} & \cellcolor{green} 4.078  & \cellcolor{green}  0.0109 & \cellcolor{green} 42.486 & \cellcolor{green}  0.0445 \\
\midrule
2A & 10^{10.6492} &  0.1592  &   0.0043  &  0.1348  &   0.0065   \\
2B & 10^{10.4163} & \cellcolor{green} 0.9843   & \cellcolor{green}  0.0114  & \cellcolor{green}  0.9278 & \cellcolor{green}   0.0113 \\
2C & 10^{10.2155} & 0.0258 &  0.0044 & 0.0360 &  0.0059 \\
\midrule
3A & 10^{8.3490} &  0.0486 & 0.0040   &  0.0603  &  0.0053 \\
3B & 10^{8.5372} &  0.0680  &  0.0033 &  0.1459 &  0.0070 \\
3C & 10^{8.6586} & \cellcolor{green}  2.0611 & \cellcolor{green}  0.1595   & \cellcolor{green}  10.567  & \cellcolor{green}   0.9011 \\
\midrule
4A & 10^{7.9062} & \cellcolor{green}  6.0706  & \cellcolor{green} 1.3182  &  11.8882   &  1.5319  \\
4B & 10^{7.9903} & \cellcolor{green}  0.2723  & \cellcolor{green}  0.0789   &  0.3155  &  0.1061 \\
4C & 10^{8.0645} &  2.3699 &  0.8486   &  0.0278 &  0.0034 \\
\bottomrule
\end{tabularx}
{\small
\begin{tablenotes}
    \item \hilight{Filter diverges}
  \end{tablenotes}}
\label{tab:ResultsTBP}
\end{threeparttable}
\end{center}

\begin{center}
\begin{threeparttable}[h]
\caption{Error variables for EKF Considering $J_2$ Perturbations}
\begin{tabularx}{.58\textwidth}
{
>{$}c<{$}
*{1}{>{$}c<{$}} |
*{2}{>{$}c<{$}} |
*{2}{>{$}c<{$}}
}
\toprule
\multirow{2}{*}{Case} & \multirow{2}{*}{$\mathrm{cond}\braces{\mathcal{W}}$} & \multicolumn{2}{c}{IOD1} & \multicolumn{2}{c}{IOD2} \\
& & \bar e_{mag} & \bar e_{dir} & \bar e_{mag} & \bar e_{dir} \\\midrule
0A & 10^{11.1486} &  0.1918  &  0.0069  &  0.3140  &  0.0129 \\
0B & 10^{10.9387} &  0.7262  &  0.0505  &  0.1571  &  0.0086\\
0C & 10^{10.9851} &  0.0900  &  0.0061 &  0.1002  &  0.0154\\
\midrule
\mathbf{1} & 10^{16.6871} & \cellcolor{green} 4.059  & \cellcolor{green}  0.0109 & \cellcolor{green} 42.381 & \cellcolor{green}  0.0445 \\
\midrule
2A & 10^{10.6492} &  0.1978  &  0.0047  &  0.1638  &   0.0065   \\
2B & 10^{10.4163} & \cellcolor{green}  0.9853  & \cellcolor{green}  0.9307 & \cellcolor{green}  0.9278 & \cellcolor{green}   0.0114 \\
2C & 10^{10.2155} & 0.0563 &  0.0047 & 0.0536 &  0.0059 \\
\midrule
3A & 10^{8.3490} &  0.0516 & 0.0041   &  0.0643  &  0.0053 \\
3B & 10^{8.5372} &  0.0739  &  0.0035 &  0.1485 &  0.0071 \\
3C & 10^{8.6586} & \cellcolor{green}  2.063 & \cellcolor{green} 0.1614    & \cellcolor{green}  10.136  & \cellcolor{green}   0.9195 \\
\midrule
4A & 10^{7.9062} & \cellcolor{green}  6.359  & \cellcolor{green} 1.4235  &  12.0062   &  1.5310  \\
4B & 10^{7.9903} & \cellcolor{green}  0.2773  & \cellcolor{green}  0.0814   &  0.5244  &  0.2305 \\
4C & 10^{8.0645} &  2.0105 &  0.9144   &  0.0303 &  0.0036 \\
\bottomrule
\end{tabularx}
{\small
\begin{tablenotes}
    \item \hilight{Filter diverges}
  \end{tablenotes}}
\label{tab:ResultsJ2}
\end{threeparttable}
\end{center}

These data from the simulations show several important aspects and challenges with relative orbit determination involving angles-only measurements.
First, an important observation from the data is that the IOD guess is critical to the success of the Kalman filter.
Cases 2B, 3C, 4A, and 4B experience filter divergence when the IOD estimate is largely inaccurate, shown in Tables \ref{tab:IODerr}, \ref{tab:ResultsTBP}, and \ref{tab:ResultsJ2}, even when cases with similar observabilities do not experience filter divergence.
The Kalman filter, which is based on linear dynamic systems, is modeled as a Markov chain and hence is memoryless.
Therefore when nothing more than an inaccurate IOD prediction is poorly assumed to contain accurate information about the system, the recursive formulation of the Kalman filter is unable to converge.
In these diverging cases, a greater level of observability or an algorithm that relies on a larger history of measurements may improve convergence.


In contrast, every variation of Case $\mathbf{1}$ has a closer IOD than the IODs of other cases, but experiences the most severe divergence.
Case $\mathbf{1}$ has a greater condition number of $\mathcal W$, which corresponds with low observability according to the proposed observability measure.
A series of angles-only measurements may correspond with a large range of possible relative orbits, even with a low variation in those measurements.
Therefore, small levels of noise in the sensor or uncertainties in the orbit prevent the Kalman filter from converging to an accurate estimate, even with an accurate IOD.
The state magnitudes of both variations of Case $\mathbf{1}$ are illustrated against various converging cases in Figure \ref{fig:ExampesEachCase} to show how important this observability is to the filter performance.

\begin{figure}[h]
\centerline{
	\subfigure[Case 0C, IOD1]{
		\includegraphics[width=0.35\textwidth]{Case0Cq1IOD1TBPnormX.pdf}}
	\hfill
	\subfigure[Case 0C, IOD2]{
	\includegraphics[width=0.35\textwidth]{Case0Cq1IOD2TBPnormX.pdf}}
}
\centerline{
	\subfigure[Case $\mathbf{1}$, IOD1]{
		\includegraphics[width=0.35\textwidth]{Case1q1IOD1TBPnormX.pdf}}
	\hfill
	\subfigure[Case $\mathbf{1}$, IOD2]{
	\includegraphics[width=0.35\textwidth]{Case1q1IOD2TBPnormX.pdf}}
}
\centerline{
	\subfigure[Case 2A, IOD1]{
		\includegraphics[width=0.35\textwidth]{Case2Aq1IOD1TBPnormX.pdf}}
	\hfill
	\subfigure[Case 2A, IOD2]{
	\includegraphics[width=0.35\textwidth]{Case2Aq1IOD2TBPnormX.pdf}}
}
\centerline{
	\subfigure[Case 3A, IOD1]{
		\includegraphics[width=0.35\textwidth]{Case3Aq1IOD1TBPnormX.pdf}}
	\hfill
	\subfigure[Case 3A, IOD2]{
	\includegraphics[width=0.35\textwidth]{Case3Aq1IOD2TBPnormX.pdf}}
}
\centerline{
	\subfigure[Case 4C, IOD1]{
		\includegraphics[width=0.35\textwidth]{Case4Cq1IOD1TBPnormX.pdf}}
	\hfill
	\subfigure[Case 4C, IOD2]{
	\includegraphics[width=0.35\textwidth]{Case4Cq1IOD2TBPnormX.pdf}}
}
\caption{The magnitudes of the relative states (solid red: true, blue dashed: estimated) are shown for various two-body propagated cases with close IOD estimates where the filter only diverges when the observability is low.}\label{fig:ExampesEachCase}
\end{figure}


Another observation from the data is that the $J_2$ perturbation, which is well-known to have a noticeable impact perturbing orbits from the two-body model for satellites around the earth, yields an insignificant change when compared to the data when only two-body forces are considered in the process model.
In fact, every case when the filter diverged occurred with both the two-body problem and $J_2$ simulations.
Examples of these results are tabulated in Tables \ref{tab:ResultsTBP} and \ref{tab:ResultsJ2} and illustrated in Figure \ref{fig:TBPvsJ2}.



\begin{figure}[h]
\centerline{
	\subfigure[State magnitude with the two-body solution]{
		\includegraphics[width=0.48\textwidth]{Case3Bq1IOD1TBPnormX.pdf}}
	\hfill
	\subfigure[State magnitude with the $J_2$ solution]{
		\includegraphics[width=0.48\textwidth]{Case3Bq1IOD1HighFidelitynormX.pdf}}
}
\centerline{
	\subfigure[Uncertainty measure with the two-body solution]{
		\includegraphics[width=0.48\textwidth]{Case3Bq1IOD1TBPeigP.pdf}}
	\hfill
	\subfigure[Uncertainty measure with the $J_2$ solution]{
		\includegraphics[width=0.48\textwidth]{Case3Bq1IOD1eigP.pdf}}
}
\caption{The magnitudes and maximum eigenvalues of the state covariance matrices of the relative states (solid red: true, blue dashed: estimated) in Case 3B with IOD1 serve as an example where the $J_2$ propagated simulations give very close results to those of the two-body solution.
}\label{fig:TBPvsJ2}
\end{figure}






\section{Conclusions}

It is well known that the linearized relative orbital dynamics are not observable with angles-only
measurements.
This paper formulates a measure of nonlinear observability, and investigates a less conservative observability criteria by using higher-order Lie derivatives.
Numerous relative orbit determination cases are illustrated numerically by an extended Kalman filter, which shows that the degree of observability by the proposed measure and an accurate IOD prediction are useful for predicting filter convergence.






\bibliographystyle{AAS_publication}   % Number the references.
\bibliography{references}   % Use references.bib to resolve the labels.



\end{document}
