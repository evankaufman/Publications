%% bare_conf.tex
%% V1.4b
%% 2015/08/26
%% originally created by Michael Shell
%%
%% See:
%% http://www.michaelshell.org/
%% for current contact information.
%%
%% This is a skeleton file demonstrating the use of IEEEtran.cls
%% (requires IEEEtran.cls version 1.8b or later) with an IEEE
%% conference paper.
%%
%% Support sites:
%% http://www.michaelshell.org/tex/ieeetran/
%% http://www.ctan.org/pkg/ieeetran
%% and
%% http://www.ieee.org/
%%
%% Originally created by Michael Shell
%% Modifed by J. Kim for URAI conference

\documentclass[conference]{IEEEURAI}
\usepackage{graphicx}

\begin{document}
%
% paper title
% Titles are generally capitalized except for words such as a, an, and, as,
% at, but, by, for, in, nor, of, on, or, the, to and up, which are usually
% not capitalized unless they are the first or last word of the title.
% Linebreaks \\ can be used within to get better formatting as desired.
% Do not put math or special symbols in the title.
\title{3D Mapping with Bidding-Based Autonomous Exploration of Cooperative Quadrotors}


% author names and affiliations
% use a multiple column layout for up to three different
% affiliations
\author{\IEEEauthorblockN{Evan Kaufman$^{1}$, Taeyoung Lee$^{2}$}
\IEEEauthorblockA{$^{1,2}$Department of Mechanical and Aerospace Engineering\\The George Washington University, Washington, DC, USA\\
Email: \{evankaufman, tylee\}@gwu.edu}}

% make the title area
\maketitle

% As a general rule, do not put math, special symbols or citations
% in the abstract
\begin{abstract}
These instructions give you the basic guidelines for preparing the paper for URAI. The abstract text should be organized to include the background, method of approach, results, and conclusions. The abstract is to be in 10-point, single-spaced type, and should not exceed 300 words.\\
\end{abstract}

\begin{IEEEkeywords}
Mapping, Exploration, Cooperation, Entropy
\end{IEEEkeywords}




% For peer review papers, you can put extra information on the cover
% page as needed:
% \ifCLASSOPTIONpeerreview
% \begin{center} \bfseries EDICS Category: 3-BBND \end{center}
% \fi
%
% For peerreview papers, this IEEEtran command inserts a page break and
% creates the second title. It will be ignored for other modes.
\IEEEpeerreviewmaketitle

\subsection*{Background Research}
\begin{itemize}
	\item Simmons et. al.~\cite{SimApfBurFoxMooThrYou00}
	\begin{itemize}
		\item Introduced the concept of ``bids''
		\item Main idea: central executive assigns tasks to maximized overall utility and minimize coverage overlap
		\item \emph{bids}: ``describe their estimates of the expected information gain and costs of traveling to various locations''
		\item Keep robots separated to decrease overall time
		\item Each robot has its own local map
		\item Expected information gain: from frontier cells, count the number unknown cells before free/occupied cells are reached within a circular region
		\item Without discounting (percentage overlap), the actions would be uncoordinated
		\item Interested addition: hystereses is used to make robots finish their tasks
	\end{itemize}
	\item Zlot et. al.~\cite{ZloSteDiaTha02}
	\begin{itemize}
		\item Main idea: robots share information with each other and auction off ``tasks'' 
		\item References~\cite{SimApfBurFoxMooThrYou00} giving credit for bidding, but claiming that the centralized manner prevents robustness
		\item Proposed approach stresses robustness: introduction/loss of team members, communication interruptions \& failures
		\item Central interface is only for humans and can disappear; robots communicate directly with each other for completely distributed implementation
	\end{itemize}
	\item Cavalcante et. al.~\cite{CavNorCha13}
	\begin{itemize}
		\item Team of robots must visit a set of target points on a known map
		\item Goal: compute and execute paths to minimize total mission costs with combinatorial auctions
		\item ``Auction-based mechanisms have been used to coordinate multiple robots in exploration missions, due to their simplicity and flexibility''
		\item A different problem: given $m$ targets and $n$ robots, move robots toward targets, then return in some optimal fashion
		\item Not really about generating a map, as this is assumed known
	\end{itemize}
%	\item Elizondo-Leal et. al.~\cite{LesRamPul08}
	\item Choi et. al.~\cite{ChoBruHow09}
	\begin{itemize}
		\item Decentralized with local communications
		\item ``...presenting two decentralized algorithms: the consensus-based auction algorithm (CBAA) and its generalization to the multi-assignment problem, i.e., the consensus-based bundle algorithm (CBBA)''
		\item Both algorithms proven to guarantee convergence to a conflict-free assignment
	\end{itemize}
	\item Sariel et. al.~\cite{SarBal05,SarBal06}
	\begin{itemize}
		\item Decentralized: ``We propose a framework... that require diverse capabilities and collective work without a central planner/decision maker''
		\item Multi-agent multi-target exploration problem in a dynamic environment
		\item Each target is visited by at least 1 robot
		\item Optimal: total cost minimization
	\end{itemize}
	\item Main ideas
	\begin{itemize}
		\item Decentralized approaches are very common, using one agent as an auctioneer that considers its own situational awareness
		\item Inspired by Simmons et. al.~\cite{SimApfBurFoxMooThrYou00} that applies discounting for cooperation, but now we apply these ideas to a different mapping objective
		\item Argument against decentralized approach: Since the occupancy grid represents the entire environment, it is natural that the exploration strategy adopts a global interpretation.
	\end{itemize}
\end{itemize}

\section{Introduction}




\begin{itemize}
	\item Motivations and definitions
	\begin{itemize}
		\item Motivation for mapping and exploration
		\item Motivation for multi-agent systems
		\item Definitions of mapping and exploration
	\end{itemize}
	\item Reasons for Exact Bayesian mapping
	\begin{itemize}
		\item Uses sensor properties explicitly
		\item Stochastic properties may be different for different sensors
		\item Multiple sensors per vehicle, multiple vehicles okay
	\end{itemize}
	\item Entropy-based exploration
	\begin{itemize}
		\item Reasons for entropy, and recent advances for precise solution
		\item Difficulty of considering various robot configuration combinations
		\item How auction has cooperative benefits with computationally-efficient manner
	\end{itemize}
	\item Organization of paper
\end{itemize}

%\subsection{First Part}
%Author names and affiliations are to be centered beneath the paper title and typed in 11-point Times New Roman, non-boldface type. Multiple authors may be shown in a two- or three-column format, with their affiliations below the authors’ names. Include the e-mail addresses. The author information is followed by two single-spaced blank lines before the main text.
%
%\subsection{Format}
%In formatting your page, set the top margin to 25mm, the bottom margin to 25mm, the side margins to 20mm, and heading to 15mm. This gives a column width of 82mm and the space between two columns 6mm. Left - and right - justify the columns.  
% 
%Type the main text in 10-point Times, single-spaced. All paragraphs should be indented 4mm. Please do not place any additional blank lines between paragraphs. 
%
%
%\section{Elements of a Paper}
%\subsection{Figures and Tables}
%\subsubsection{Figures}
%All figures (graphs, line drawings, photographs, etc.) should be numbered consecutively and have a caption consisting of the figure number and a brief title or description of the figure. Figures should be referenced within the text as “Fig. 1.” When the reference to a figure begins a sentence, the abbreviation "Fig." should be spelled out as “Figure 1.” When the reference to a figure begins a sentence, the abbreviation "Fig." should be spelled out as “Figure 1.”
%
%\begin{figure}[!t]
%\centering
%\includegraphics[width=2.5in]{kros.png}
%\caption{Caption of a sample figure. It is indented 4mm and placed below the figure.}
%\label{fig_kros}
%\end{figure}
%
%\subsubsection{Tables}
%All tables should be numbered consecutively and have a caption consisting of the table number and a brief title. This number should be used when referring to the table in the text. Tables should be inserted as part of the text as close as possible to its first reference. 
%
%\begin{table}[!t]
%%% increase table row spacing, adjust to taste
%\renewcommand{\arraystretch}{1.3}
%\caption{Caption of a sample table.}
%\label{table_example}
%\centering
%\begin{tabular}{cc}
%\hline
%Number & Types\\
%\hline
%1 & Industrial robots\\
%\hline
%2 & Service robots\\
%\hline
%\end{tabular}
%\end{table}
%
%\subsection{Equations}
%Equations should be numbered consecutively beginning with (1) to the end of the paper, including any appendices. The number should be enclosed in parentheses and set flush right in the column on the same line as the equation. Equations should be referenced within the text as “Eq. (x).” When the reference to an equation begins a sentence, it should be spelled out as “Equation (x).”
%
%\begin{equation}
%x^2+y^2=r^2
%\end{equation}
%
%\subsection{References}
%Within the text, references should be cited in numerical order according to their order of appearance. The numbered reference citation within text should be enclosed in brackets (e.g., It was shown by Craig [1] that … . ). In the case of two citations, the numbers should be separated by a comma [1, 2]. In the case of more than two references, the numbers should be separated by a dash [1-4].
%
%References should be listed together at the end of the paper in numerical order according to the sequence of citations within the text. Refer to some references at the end of this template for correct format. 
%
%\section*{Acknowledgement}
%Acknowledgments may be made to individuals or institutions not mentioned elsewhere in the work who have made an important contribution.

% On the last page of your paper, adjust the lengths of the columns so that they are approximately equal.

% references section
%\begin{thebibliography}{1}
%\bibitem{IEEEhowto:kopka}
%G. D. Hong, T. Nakamura and K. G. Kim, “Article Title,” Journal Title, Vol. 10, No. 6, pp. 1-10, 2006. 
%\bibitem{IEEEhowto:kopka}
%A. C. Smith, Introduction to Robotics, Publisher, New York, 2006.
%\end{thebibliography}
\bibliography{../../BibSources}% master source for all publications always 2 directories up
\bibliographystyle{IEEEtran}

\end{document}


