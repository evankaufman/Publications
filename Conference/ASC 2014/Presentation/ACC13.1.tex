\documentclass[11pt,professionalfonts,hyperref={pdftex,pdfpagemode=none,pdfstartview=FitH}]{beamer}
%\usepackage{times}
%\usefonttheme{serif}
%\usepackage{helvet}
%\usepackage{amsmath,amssymb}
\usepackage{graphicx,multirow}
\usepackage[scriptsize]{subfigure}
%\usepackage{movie15}

%\usepackage{warmread}
%\usepackage[all,import]{xy}

%\renewcommand\mathfamilydefault{\rmdefault}

\newcommand{\norm}[1]{\ensuremath{\left\| #1 \right\|}}
\newcommand{\bracket}[1]{\ensuremath{\left[ #1 \right]}}
\newcommand{\braces}[1]{\ensuremath{\left\{ #1 \right\}}}
\newcommand{\parenth}[1]{\ensuremath{\left( #1 \right)}}
\newcommand{\pair}[1]{\ensuremath{\langle #1 \rangle}}
\newcommand{\met}[1]{\ensuremath{\langle\langle #1 \rangle\rangle}}
\newcommand{\refeqn}[1]{(\ref{eqn:#1})}
\newcommand{\reffig}[1]{Fig. \ref{fig:#1}}
\newcommand{\tr}[1]{\mathrm{tr}\ensuremath{\negthickspace\bracket{#1}}}
\newcommand{\trs}[1]{\mathrm{tr}\ensuremath{[#1]}}
\newcommand{\deriv}[2]{\ensuremath{\frac{\partial #1}{\partial #2}}}
\newcommand{\SO}{\ensuremath{\mathsf{SO(3)}}}
\newcommand{\T}{\ensuremath{\mathsf{T}}}
\renewcommand{\L}{\ensuremath{\mathsf{L}}}
\newcommand{\so}{\ensuremath{\mathfrak{so}(3)}}
\newcommand{\SE}{\ensuremath{\mathsf{SE(3)}}}
\newcommand{\se}{\ensuremath{\mathfrak{se}(3)}}
\renewcommand{\Re}{\ensuremath{\mathbb{R}}}
\newcommand{\aSE}[2]{\ensuremath{\begin{bmatrix}#1&#2\\0&1\end{bmatrix}}}
\newcommand{\ase}[2]{\ensuremath{\begin{bmatrix}#1&#2\\0&0\end{bmatrix}}}
\newcommand{\D}{\ensuremath{\mathbf{D}}}
\newcommand{\Sph}{\ensuremath{\mathsf{S}}}
\renewcommand{\S}{\Sph}
\newcommand{\J}{\ensuremath{\mathbf{J}}}
\newcommand{\Ad}{\ensuremath{\mathrm{Ad}}}
\newcommand{\intp}{\ensuremath{\mathbf{i}}}
\newcommand{\extd}{\ensuremath{\mathbf{d}}}
\newcommand{\hor}{\ensuremath{\mathrm{hor}}}
\newcommand{\ver}{\ensuremath{\mathrm{ver}}}
\newcommand{\dyn}{\ensuremath{\mathrm{dyn}}}
\newcommand{\geo}{\ensuremath{\mathrm{geo}}}
\newcommand{\Q}{\ensuremath{\mathsf{Q}}}
\newcommand{\G}{\ensuremath{\mathsf{G}}}
\newcommand{\g}{\ensuremath{\mathfrak{g}}}
\newcommand{\Hess}{\ensuremath{\mathrm{Hess}}}
\newcommand{\refprop}[1]{Proposition \ref{prop:#1}}

\definecolor{mygray}{gray}{0.9}

\mode<presentation> {
  \usetheme{Warsaw}
  \usefonttheme{serif}
  \setbeamercovered{transparent}
}

\newcommand{\mypaper}{}

\setbeamertemplate{footline}%{split theme}
{%
  \leavevmode%
  \hbox{\begin{beamercolorbox}[wd=.5\paperwidth,ht=2.5ex,dp=1.125ex,leftskip=.3cm,rightskip=.3cm plus1fill]{author in head/foot}%
    \usebeamerfont{author in head/foot}\insertshorttitle
  \end{beamercolorbox}%
  \begin{beamercolorbox}[wd=.5\paperwidth,ht=2.5ex,dp=1.125ex,leftskip=.3cm,rightskip=.3cm]{title in head/foot}
%    \usebeamerfont{title in head/foot}\mypaper\hfill \insertframenumber/\inserttotalframenumber
    \usebeamerfont{title in head/foot}\hfill \insertframenumber/\inserttotalframenumber
  \end{beamercolorbox}}%
  \vskip0pt%
} \setbeamercolor{box}{fg=black,bg=yellow}

\title[Design and Development of a Free-Floating Hexrotor UAV for 6-DOF Maneuvers]{\large Design and Development of a Free-Floating\\Hexrotor UAV for 6-DOF Maneuvers}

\author{\vspace*{-0.3cm}}

\institute{\footnotesize
{\normalsize Taeyoung Lee}\vspace*{0.2cm}\\
  Mechanical and Aerospace Engineering\\ George Washington University}

\date{}

\definecolor{tmp}{rgb}{0.804,0.941,1.0}
\setbeamercolor{numerical}{fg=black,bg=tmp}
\setbeamercolor{exact}{fg=black,bg=red}

\newtheorem{prop}{Proposition}



\renewcommand{\emph}[1]{\textit{\textbf{\color{blue}{#1}}}}


\begin{document}

\begin{frame}
  \titlepage
\end{frame}


\section*{}
\subsection*{Introduction}

\begin{frame}
\frametitle{Introduction}
\begin{itemize}
    \item Attitude Dynamics of a Rigid Body
    \begin{itemize}
    	\item Examples: robotics, flight dynamics, and space systems
    	\item Attitude: orientation of a body-fixed frame with respect to a reference frame
		\item Special Orthogonal Group
        \begin{align*}
        \SO = \{ R\in\Re^{3\times 3}\,|\, R^TR=I,\quad\mathrm{det}[R]=1\}.
        \end{align*}
        \item Compact nonlinear configuration manifold
        \item Unique dynamic properties that cannot be observed in $\Re^n$
    \end{itemize}
\end{itemize}

\only<1->{
\begin{figure}
\centerline{
    \includegraphics[width=2.1cm]{f22.jpg}\hspace*{0.1cm}
    \includegraphics[width=2.1cm]{sat.jpg}\hspace*{0.1cm}
    \includegraphics[width=2.1cm]{ast.jpg}\hspace*{0.1cm}
    \includegraphics[width=2.1cm]{rov.pdf}
}
\end{figure}}

\end{frame}


\begin{frame}
\frametitle{Introduction}
\framesubtitle{Attitude Control Systems: Classification Based on Representation}

\begin{itemize}
	\item Euler angles
	\begin{itemize}
	\item Any minimal representation of attitude has singularities
	\item Switching between several types to cover $\SO$
	\item Complex expressions for trigonometric functions
	\end{itemize}
\vspace*{0.3cm}\pause
	\item Quaternions in $\Sph^3=\{q\in\Re^4\,|\, \|q\|=1\}$
	\begin{itemize}
	\item No singularities
	\item Ambiguity: $\Sph^3$ double-covers $\SO$
	\pause
	\item Unwinding: a rigid body rotates unnecessarily through a large angle, even if the initial attitude error is small %{\footnotesize [Bhat, Bernstein 00]}
	\item Sensitive to measurement noise %{\footnotesize [Mayhew et. al. 11]}
	\pause
	\item Attitude in $\SO$ should be \emph{carefully} lifted to $\Sph^3$
	\end{itemize}
\vspace*{0.3cm}\pause
	\item Rotation matrices in $\SO$	
	\begin{itemize}
	\item No singularity, and no ambiguity
	\end{itemize}
\end{itemize}

\end{frame}



\begin{frame}
\frametitle{Introduction}
\framesubtitle{Attitude Control Systems: Classification Based on Stability Properties}

\begin{itemize}
	\item Topological Obstruction
	\begin{itemize}
	\item No \emph{smooth} control system to \emph{globally} asymptotically stabilize a desired attitude
	\item Region of attraction is homeomorphic to $\Re^n$
	\end{itemize}
\vspace*{0.3cm}\pause
	\item \emph{Almost} Global Asymptotic Stabilization
	\begin{itemize}
	\item Smooth control system
	\item Desired attitude is asymptotically stable
	\item Region of attraction excludes a zero-measure set
	\end{itemize}
\vspace*{0.3cm}\pause
	\item \emph{Hybrid} Global Asymptotic Stabilization
	\begin{itemize}
	\item Hysteresis-based switching control system
	\item Asymptotic stability from hybrid invariance principle
	\end{itemize}
\end{itemize}

\end{frame}




\begin{frame}
\frametitle{Introduction}
\framesubtitle{Contributions}

\begin{itemize}
\item \emph{Exponential} Stability
	\begin{itemize}
	\item Prior Work: asymptotic stability
		\begin{itemize}
		\item Stabilization: LaSalle's principle
		\item Tracking: an exogenous system to reformulate it into a hybrid autonomous system
		\end{itemize} 
	\item Lyapunov analysis for time-varying systems on $\SO$
	\end{itemize}
\vspace*{0.3cm}\pause
\item Simple Hybrid Control System for \emph{Global} Attractiveness
	\begin{itemize}
	\item Prior Work: error function based on stretched rotations, etc
	\item New intuitive form of attitude error function is introduced for simpler, explicit stability conditions
	\end{itemize}
\vspace*{0.3cm}\pause
\item \emph{Robustness}
	\begin{itemize}
	\item Prior Work: integral term is used heuristically without stability analysis
	\item New integral term is introduced to guarantee stability with fixed disturbances
	\end{itemize}
\end{itemize}
\end{frame}


\section*{}
\subsection*{Problem Formulation}


\begin{frame}
\frametitle{Problem Formulation}

\begin{itemize}
    \item Attitude Dynamics of a Rigid Body
    \begin{itemize}
    \item Equations of motion
    \begin{gather*}
J\dot \Omega + \Omega\times J\Omega = u +\Delta,\\
\dot R = R\hat\Omega = \hat\omega R,\quad (\omega=R\Omega).
\end{gather*}
	\item Fixed unknown disturbance: $\Delta\in\Re^3$
    \end{itemize}
\vspace*{0.4cm}\pause    
    \item Attitude Tracking Problem
    \begin{itemize}
    \item Desired attitude trajectory: $R_d(t):[t_0,t_f]\rightarrow\SO$
    \begin{gather*}
    \dot R_d = R_d \hat\Omega_d = \hat\omega_d R_d,\quad (\omega_d=R_d\Omega_d).
    \end{gather*}
    \item Goal: Design control input $u$ such that $(R,\omega)=(R_d,\omega_d)$ is globally exponentially stable
    \end{itemize}
\end{itemize}
\end{frame}


\section*{}
\subsection*{Control Systems Design}


\begin{frame}
\frametitle{Control Systems Design}

\begin{itemize}
\item Four Types of Attitude Control Systems
\end{itemize}
\vspace*{-0.8cm}
\begin{figure}
\setlength{\unitlength}{0.1\textwidth}\footnotesize
\begin{picture}(10,6)(-5.3,-2.8)
\only<1->{
%
\put(-6,0.9){\shortstack[c]{Without\\ disturbance\\ $\Delta=0$}}
\put(-3.3,2.5){\shortstack[c]{Smooth control}}
\put(-4.3,0.3){\framebox(4,2)[c]{\shortstack[c]{Smooth Attitude Tracking\\for \textit{Almost Global}\\ \textit{Exponential} Stability}}}}
%
\only<2->{
\put(-0.3,1.3){\vector(1,0){0.8}}
\put(1.6,2.5){\shortstack[c]{Hybrid control}}
\put(0.5,0.3){\dashbox{0.08}(4,2)[c]{\shortstack[c]{Hybrid Attitude Tracking\\for \textit{Global}\\ \textit{Exponential} Stability}}}}
%
\only<3->{
\put(-2.3,0.3){\vector(0,-1){0.6}}
\put(2.5,0.3){\vector(0,-1){0.6}}
\put(-6,-1.7){\shortstack[c]{With\\ disturbance\\ $\Delta\neq0$}}
{\linethickness{1.0pt}
\put(-4.3,-2.3){\framebox(4,2)[c]{\shortstack[c]{Smooth Attitude Tracking\\for \textit{\bf Robust} \textit{Almost Global}\\ \textit{Exponential} Stability}}}}
{\linethickness{1.0pt}
\put(0.5,-2.3){\dashbox{0.08}(4,2)[c]{\shortstack[c]{Hybrid Attitude Tracking\\for \textit{{\bf Robust} Global}\\ \textit{Exponential} Stability}}}}}
\end{picture}
\end{figure}

\end{frame}


\begin{frame}
\frametitle{Smooth Attitude Tracking}

\begin{itemize}
\item Attitude Error Function: {\small $\Psi(R,R_d):\SO^2\rightarrow \Re_+$}
\begin{itemize}
\item Choose characteristic direction $b_1,b_2\in\Sph^2$ fixed to the body\\
(ex: direction of an antenna of satellite)
\item \textit{Current} direction of $b_i$ in the inertial frame: $r_i = R b_i\in\Sph^2$
\item \textit{Desired} direction of $b_i$ in the inertial frame: $r_{i_d} = R_d b_i\in\Sph^2$
\item Direction error function: $\Psi_i=\frac{1}{2}\|r_i-r_{i_d}\|^2=1- r_i \cdot r_{i_d}$
\item Attitude error function:
\begin{gather*}
\Psi = k_1 \Psi_1 + k_2 \Psi_2,\quad (k_1\neq k_2).
\end{gather*}
\end{itemize}
\pause\vspace*{0.1cm}
\item Attitude Error Vector: {\small $e_r(R,R_d):\SO^2\rightarrow \Re^3$}
	\begin{itemize}
	\item Derivatives of the error function
	\begin{align*}
	\frac{d}{d\epsilon}\bigg|_{\epsilon=0} \Psi(R\exp(\epsilon\hat\eta),R_d) = e_r\cdot \eta.%(R^T r_{i_d}\times b_i)\cdot \eta \equiv e_{r_i}\cdot\eta.
	\end{align*}
%	\item Attitude error vector: $e_r = k_1 e_{r_1} + k_2 e_{r_2}$.
	\end{itemize}
	\vspace*{0.1cm}
\pause	
\item Angular Velocity Error Vector: $e_\Omega=\Omega- R^T\omega_d$.	
\end{itemize}
\end{frame}


%\begin{frame}
%\frametitle{Smooth Attitude Tracking}
%
%\begin{itemize}
%\item Angular Velocity Error Vector: $e_\Omega=\Omega- R^T\omega_d$.
%\end{itemize}
%
%\begin{prop}
%\begin{itemize}\setlength{\itemsep}{0pt}
%\item[(i)] $\Psi$ is positive definite about $R=R_d$.
%\item[(ii)] the left-trivialized derivative of $\Psi$ is $\T^*_I \L_R\, (\D_R\Psi(R,R_d))= e_r$.
%\item[(iii)] $\dot\Psi= e_r\cdot e_\Omega$, $\dot e_r\leq (k_1+k_2)\|e_\Omega\|$.
%\item[(iv)] there exist $h_1,h_2>0$ such that
%\begin{align*}
%h_1\|e_r\|^2 \leq \Psi(R,R_d) \leq h_2 \|e_r\|^2.
%\end{align*}
%\end{itemize}
%\end{prop}
%\end{frame}


\begin{frame}
\frametitle{Smooth Attitude Tracking}

\begin{itemize}
\item Smooth Attitude Tracking Control on $\SO$

\begin{prop}\label{prop:AGASSO}
For $k_1,k_2,k_\omega >0$ with $k_1\neq k_2$ a control input is chosen as
\begin{align*}
u & = -e_r -k_\Omega e_\Omega  +(R^T\omega_d)^\wedge JR^T\omega_d
+ JR^T\dot\omega_d.\label{eqn:uSO}
\end{align*}
Then, the following properties hold:
\begin{enumerate}
\item There are four equilibrium attitudes, given by $R_d$ and $180^\circ$ rotation of $R_d$ about each of $r_{1_d}$, $r_{2_d}$, and $r_{1_d}\times r_{2_d}$.
\item The desired attitude is almost globally asymptotically stable and almost semi-globally exponentially stable.
\item The three undesired equilibria are unstable.
\end{enumerate}
\end{prop}
\end{itemize}
\end{frame}


\begin{frame}
\frametitle{Numerical Example}

\begin{itemize}
\item Simulation Parameters
	\begin{itemize}
	\item Inertia Matrix: $J=\mathrm{diag}[3,\,2,\,1]\,\mathrm{kgm^2}$.
	\vspace*{0.3cm}
	\item Desired Attitude Trajectory
	\begin{gather*}
	R_d(t) =\exp(\psi(t)\hat e_3)\exp(\theta(t)\hat e_2)\exp(\phi(t)\hat e_1),\\
	\phi(t)=\sin 0.5t,\quad \theta(t)=0.1(-1+t),\quad \psi(t)=1-\cos t.
	\end{gather*}
	\item Controller Parameters
	\begin{gather*}
	k_1=10,\quad k_2=11,\quad k_\Omega=4.42
	\end{gather*}
	\end{itemize}
\end{itemize}

\end{frame}

\begin{frame}
\frametitle{Numerical Example}

\only<1>{
\begin{itemize}
\item Initial Conditions: {\small $R(0)=I$, $\Omega(0)=0_{3\times 1}$ }
\begin{figure}
\centerline{
	\subfigure[Attitude Error $\|R-R_d\|$]{\includegraphics[width=0.47\textwidth]{ACC13_1_neR.pdf}}
	\hfill
	\subfigure[Angular Velocity Error $e_\Omega$]{\includegraphics[width=0.47\textwidth]{ACC13_1_eW.pdf}}
	}
\end{figure}
\end{itemize}}

\only<2>{
\begin{itemize}
\item Initial Conditions: {\small $R(0) = \exp(0.9999\pi (r_{1_d}\times r_{2_d})^\wedge) R_d(0)$, $\Omega(0)=R(0)^T \omega_d(0)$ (close to one of the undesired equilibria)}
\begin{figure}\setcounter{subfigure}{0}
\centerline{
	\subfigure[Attitude Error $\|R-R_d\|$]{\includegraphics[width=0.47\textwidth]{ACC13_3_neR.pdf}}
	\hfill
	\subfigure[Angular Velocity Error $e_\Omega$]{\includegraphics[width=0.47\textwidth]{ACC13_3_eW.pdf}}
	}
\end{figure}
\end{itemize}}

\end{frame}



\begin{frame}
\frametitle{Hybrid Attitude Tracking}

\only<1>{
\begin{itemize}
\item Smooth Attitude Tracking
\begin{figure}\setcounter{subfigure}{0}
\centerline{
	\subfigure[Error Function $\Psi_i=1-r_i\cdot r_{i_d}$]{\includegraphics[width=0.47\textwidth]{ACC13_H_P0.pdf}}
	\hfill
	\subfigure[Error Vector $e_{r_i}$]{\includegraphics[width=0.475\textwidth]{ACC13_H_er0.pdf}}
	}
\caption{Direction error function $\Psi_i$ and error vector $e_{r_1}$ as a function of angle between $r_i$ and $r_{i_d}$}
\end{figure}
\end{itemize}}

\only<2>{
\begin{itemize}
\item Hybrid Attitude Tracking
\begin{itemize}
\item Introduce \textbf{\textit{\color{red}{expelling}}} functions: $\Psi_{E_i}=\alpha+\beta r_i\cdot (r_{1_d}\times r_{2_d})$
\end{itemize}
\begin{figure}\setcounter{subfigure}{0}
\centerline{
	\subfigure[Error Function $\Psi_i,\Psi_{E_i}$]{\includegraphics[width=0.47\textwidth]{ACC13_H_P1.pdf}}
	\hfill
	\subfigure[Error Vector $e_{r_i},e_{E_i}$]{\includegraphics[width=0.475\textwidth]{ACC13_H_er1.pdf}}
	}
\caption{Direction error function $\Psi_i$ and error vector $e_{r_1}$ as a function of angle between $r_i$ and $r_{i_d}$}
\end{figure}
\end{itemize}
}
\end{frame}


\begin{frame}
\frametitle{Hybrid Attitude Tracking}

\only<1>{
\begin{itemize}
\item Switching Algorithm
\begin{itemize}
\item Choose the error function with the minimum value
\item Sensitive to noise, chattering
\end{itemize}
\begin{figure}\setcounter{subfigure}{0}
\centerline{
	\subfigure[Error Function $\Psi_i,\Psi_{E_i}$]{\includegraphics[width=0.47\textwidth]{ACC13_H_P2.pdf}}
	\hfill
	\subfigure[Error Vector $e_{r_i},e_{E_i}$]{\includegraphics[width=0.475\textwidth]{ACC13_H_er2.pdf}}
	}
\caption{Direction error function $\Psi_i$ and error vector $e_{r_1}$ as a function of angle between $r_i$ and $r_{i_d}$}
\end{figure}
\end{itemize}}


\only<2>{
\begin{itemize}
\item Hysteresis-Based Switching Algorithm
\begin{itemize}
\item Switch if the difference to the min. value is greater than $\delta$
\item Robust to noise, eliminate chattering
\end{itemize}
\begin{figure}\setcounter{subfigure}{0}
\centerline{
	\subfigure[Error Function $\Psi_i,\Psi_{E_i}$]{\includegraphics[width=0.47\textwidth]{ACC13_H_P3.pdf}}
	\hfill
	\subfigure[Error Vector $e_{r_i},e_{E_i}$]{\includegraphics[width=0.475\textwidth]{ACC13_H_er3.pdf}}
	}
\caption{Direction error function $\Psi_i$ and error vector $e_{r_1}$ as a function of angle between $r_i$ and $r_{i_d}$}
\end{figure}
\end{itemize}}


\end{frame}

\begin{frame}
\frametitle{Hybrid Attitude Tracking}

\begin{itemize}
\item Global Exponential Attitude Tracking on $\SO$

\begin{prop}\label{prop:GES}
Consider the hybrid control system described above. For given constants $k_1,k_2,\alpha,\beta$ satisfying $k_1,k_2>0$, $k_1\neq k_2$, $1<\alpha<2$ and $|\beta|< \alpha-1$, choose the hysteresis gap $\delta$ such that
\begin{align}
0<\delta < \min\{k_1,k_2\}\min\{2-\alpha, \alpha-|\beta|-1\}.\label{eqn:deltaSO}
\end{align}
Then, the desired equilibrium $(R_d,\omega_d)$ is globally exponentially stable.
\end{prop} 

\begin{itemize}
\item Sufficient condition for stability is \emph{explicitly} given by (1)
\item Stronger global \emph{exponential} stability is guaranteed
\end{itemize}
\end{itemize}
\end{frame}



\begin{frame}
\frametitle{Numerical Example}

\only<1>{
\begin{itemize}
\item Smooth Attitude Control
\begin{itemize}
\item Initial Conditions: {\small $R(0) = \exp(0.9999\pi (r_{1_d}\times r_{2_d})^\wedge) R_d(0)$, $\Omega(0)=R(0)^T \omega_d(0)$ (close to one of the undesired equilibria)}
\end{itemize}
\begin{figure}\setcounter{subfigure}{0}
\centerline{
	\subfigure[Attitude Error $\|R-R_d\|$]{\includegraphics[width=0.47\textwidth]{ACC13_3_neR.pdf}}
	\hfill
	\subfigure[Angular Velocity Error $e_\Omega$]{\includegraphics[width=0.47\textwidth]{ACC13_3_eW.pdf}}
	}
\end{figure}
\end{itemize}}


\only<2>{
\begin{itemize}
\item Hybrid Attitude Control
\begin{itemize}
\item Initial Conditions: {\small $R(0) = \exp(0.9999\pi (r_{1_d}\times r_{2_d})^\wedge) R_d(0)$, $\Omega(0)=R(0)^T \omega_d(0)$ (close to one of the undesired equilibria)}
\end{itemize}
\begin{figure}\setcounter{subfigure}{0}
\centerline{
	\subfigure[Attitude Error $\|R-R_d\|$]{\includegraphics[width=0.47\textwidth]{ACC13_4_neR.pdf}}
	\hfill
	\subfigure[Angular Velocity Error $e_\Omega$]{\includegraphics[width=0.47\textwidth]{ACC13_4_eW.pdf}}
	}
\end{figure}
\end{itemize}}


\end{frame}



\begin{frame}
\frametitle{Control Systems Design}

\begin{itemize}
\item Four Types of Attitude Control Systems
\end{itemize}
\vspace*{-0.8cm}
\begin{figure}
\setlength{\unitlength}{0.1\textwidth}\footnotesize
\begin{picture}(10,6)(-5.3,-2.8)
\only<1->{
%
\put(-6,0.9){\shortstack[c]{Without\\ disturbance\\ $\Delta=0$}}
\put(-3.3,2.5){\shortstack[c]{Smooth control}}
\put(-4.3,0.3){\framebox(4,2)[c]{\shortstack[c]{Smooth Attitude Tracking\\for \textit{Almost Global}\\ \textit{Exponential} Stability}}}}
%
\only<1->{
\put(-0.3,1.3){\vector(1,0){0.8}}
\put(1.6,2.5){\shortstack[c]{Hybrid control}}
\put(0.5,0.3){\dashbox{0.08}(4,2)[c]{\shortstack[c]{Hybrid Attitude Tracking\\for \textit{Global}\\ \textit{Exponential} Stability}}}}
%
\only<2->{
\put(-2.3,0.3){\vector(0,-1){0.6}}
\put(2.5,0.3){\vector(0,-1){0.6}}
\put(-6,-1.7){\shortstack[c]{With\\ disturbance\\ $\Delta\neq0$}}
{\linethickness{1.0pt}
\put(-4.3,-2.3){\framebox(4,2)[c]{\shortstack[c]{Smooth Attitude Tracking\\for \textit{\bf Robust} \textit{Almost Global}\\ \textit{Exponential} Stability}}}}
{\linethickness{1.0pt}
\put(0.5,-2.3){\dashbox{0.08}(4,2)[c]{\shortstack[c]{Hybrid Attitude Tracking\\for \textit{{\bf Robust} Global}\\ \textit{Exponential} Stability}}}}}
\end{picture}
\end{figure}

\end{frame}


\begin{frame}
\frametitle{Robust Attitude Tracking}

\begin{itemize}
\item Generalized Integral Term
\begin{itemize}
\item Definition

\begin{center}\hspace*{0.01cm}
\begin{beamercolorbox}[wd=8cm,dp=0.15cm,ht=1.1cm,sep=0.05cm,center,shade=on]{numerical}
$e_I(t) = \displaystyle\int_{0}^t ce_r(\tau)+e_\Omega(\tau) \,d\tau.$
\end{beamercolorbox}
\end{center}

%
%
%\begin{align}
%e_I(t) = \int_{0}^t ce_r(\tau)+e_\Omega(\tau) \,d\tau,\label{eqn:eI}
%\end{align}
%for some positive constant $c$.
\item The first part is equivalent to the \textit{classical} integral term
\item The second term increases the proportional term effectively\\
(Note: $\dot e_r \neq e_\Omega$) 
\pause\vspace*{0.3cm}
\item Control inputs are augmented with the proposed, generalized integral term
\item Stabilities are guaranteed by rigorous Lyapunov analysis for time-varying systems
\end{itemize}
\end{itemize}
\end{frame}


\begin{frame}
\frametitle{Robust Attitude Tracking}

\begin{itemize}
\item Smooth Attitude Tracking
\begin{prop}\label{prop:RAGAS}
Control input is chosen as
{\small
\begin{align*}
u & = -e_r -k_\Omega e_\Omega -k_I e_I+(R^T\omega_d)^\wedge JR^T\omega_d
+ JR^T\dot\omega_d.\label{eqn:uI}
\end{align*}}
Then, the desired equilibrium is almost globally asymptotically stable, and locally exponentially stable with respect to $e_r,e_\Omega$. The integral term $e_I$ is globally uniformly bounded.
\end{prop}
\pause\vspace*{0.1cm}
\item Hybrid Attitude Tracking
\begin{prop}
The desired equilibrium $(R_d,\omega_d)$ is globally exponentially stable with respect to $e_r$ and $e_\Omega$, and the integral term $e_{I}$ is globally uniformly bounded.
\end{prop}
\end{itemize}

\end{frame}


\begin{frame}
\frametitle{Numerical Example}

\only<1>{
\begin{itemize}
\item Hybrid Attitude Control
\begin{itemize}
\item Initial Conditions: {\small $R(0) = \exp(0.9999\pi (r_{1_d}\times r_{2_d})^\wedge) R_d(0)$, $\Omega(0)=R(0)^T \omega_d(0)$ (close to one of the undesired equilibria)}
\item Fixed Disturbance: $\Delta=[-1,2,1]^T$.
\end{itemize}
\begin{figure}\setcounter{subfigure}{0}
\centerline{
	\subfigure[Attitude Error $\|R-R_d\|$]{\includegraphics[width=0.47\textwidth]{ACC13_7_neR.pdf}}
	\hfill
	\subfigure[Angular Velocity Error $e_\Omega$]{\includegraphics[width=0.47\textwidth]{ACC13_7_eW.pdf}}
	}
\end{figure}
\end{itemize}}


\only<2>{
\begin{itemize}
\item Robust Hybrid Attitude Control
\begin{itemize}
\item Initial Conditions: {\small $R(0) = \exp(0.9999\pi (r_{1_d}\times r_{2_d})^\wedge) R_d(0)$, $\Omega(0)=R(0)^T \omega_d(0)$ (close to one of the undesired equilibria)}
\item Fixed Disturbance: $\Delta=[-1,2,1]^T$.
\end{itemize}
\begin{figure}\setcounter{subfigure}{0}
\centerline{
	\subfigure[Attitude Error $\|R-R_d\|$]{\includegraphics[width=0.47\textwidth]{ACC13_8_neR.pdf}}
	\hfill
	\subfigure[Angular Velocity Error $e_\Omega$]{\includegraphics[width=0.47\textwidth]{ACC13_8_eW.pdf}}
	}
\end{figure}
\end{itemize}}
\end{frame}


\section*{}
\subsection*{Conclusions}

\begin{frame}
\frametitle{Conclusions}
\begin{itemize}
    \item Robust Global Exponential Attitude Tracking
    \begin{itemize}
        \item Lyapunov stability analysis for time-varying systems
        \item Hybrid attitude control with simpler controller design
        \item Robust control with new generalized integral control term
    \end{itemize}
\vspace*{0.4cm}\pause
    \item Geometric Controls on $\SO$
    \begin{itemize}
        \item \textit{Coordinate-free} nonlinear controls on a nonlinear manifold.
        \item Completely avoid \textit{singularities, ambiguities, and complexities}
        \item No need to introduce exogenous system to lift $\SO$ to $\Sph^3$
    \end{itemize}
\end{itemize}
\end{frame}



\end{document}

