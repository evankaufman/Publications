\documentclass[10pt]{article}
\usepackage[letterpaper,text={6.5in,8.7in},centering]{geometry}
\usepackage{epic,eepic}
\usepackage{amssymb,amsmath,times,subfigure,graphicx,theorem}
%\usepackage{warmread}
%\usepackage[all,import]{xy}
%\usepackage{eepic}
\usepackage{subfigure}
\usepackage[]{algorithm2e}
\usepackage{amsmath}
\newcommand{\argmax}{\operatornamewithlimits{argmax}}

\newcommand{\norm}[1]{\ensuremath{\left\| #1 \right\|}}
\newcommand{\bracket}[1]{\ensuremath{\left[ #1 \right]}}
\newcommand{\braces}[1]{\ensuremath{\left\{ #1 \right\}}}
\newcommand{\parenth}[1]{\ensuremath{\left( #1 \right)}}
\newcommand{\pair}[1]{\ensuremath{\langle #1 \rangle}}
\newcommand{\met}[1]{\ensuremath{\langle\langle #1 \rangle\rangle}}
\newcommand{\refeqn}[1]{(\ref{eqn:#1})}
\newcommand{\reffig}[1]{Figure \ref{fig:#1}}
\newcommand{\tr}[1]{\mathrm{tr}\ensuremath{\negthickspace\bracket{#1}}}
\newcommand{\trs}[1]{\mathrm{tr}\ensuremath{[#1]}}
\newcommand{\deriv}[2]{\ensuremath{\frac{\partial #1}{\partial #2}}}
\newcommand{\SO}{\ensuremath{\mathsf{SO(3)}}}
\newcommand{\T}{\ensuremath{\mathsf{T}}}
\renewcommand{\L}{\ensuremath{\mathsf{L}}}
\newcommand{\so}{\ensuremath{\mathfrak{so}(3)}}
\newcommand{\SE}{\ensuremath{\mathsf{SE(3)}}}
\newcommand{\se}{\ensuremath{\mathfrak{se}(3)}}
\renewcommand{\Re}{\ensuremath{\mathbb{R}}}
\newcommand{\aSE}[2]{\ensuremath{\begin{bmatrix}#1&#2\\0&1\end{bmatrix}}}
\newcommand{\ase}[2]{\ensuremath{\begin{bmatrix}#1&#2\\0&0\end{bmatrix}}}
\newcommand{\D}{\ensuremath{\mathbf{D}}}
\newcommand{\Sph}{\ensuremath{\mathsf{S}}}
\renewcommand{\S}{\Sph}
\newcommand{\J}{\ensuremath{\mathbf{J}}}
\newcommand{\Ad}{\ensuremath{\mathrm{Ad}}}
\newcommand{\intp}{\ensuremath{\mathbf{i}}}
\newcommand{\extd}{\ensuremath{\mathbf{d}}}
\newcommand{\hor}{\ensuremath{\mathrm{hor}}}
\newcommand{\ver}{\ensuremath{\mathrm{ver}}}
\newcommand{\dyn}{\ensuremath{\mathrm{dyn}}}
\newcommand{\geo}{\ensuremath{\mathrm{geo}}}
\newcommand{\Q}{\ensuremath{\mathsf{Q}}}
\newcommand{\G}{\ensuremath{\mathsf{G}}}
\newcommand{\g}{\ensuremath{\mathfrak{g}}}
\newcommand{\Hess}{\ensuremath{\mathrm{Hess}}}

\renewcommand{\baselinestretch}{1.2}
\date{}

\renewcommand{\thesubsection}{\arabic{subsection}. }
\renewcommand{\thesubsubsection}{\arabic{subsection}.\arabic{subsubsection} }

\theoremstyle{plain}\theorembodyfont{\normalfont}
\newtheorem{prob}{Question}[section]
%\renewcommand{\theprob}{\arabic{section}.\arabic{prob}}
\renewcommand{\theprob}{\arabic{prob}}

\newenvironment{subprob}%
{\renewcommand{\theenumi}{\alph{enumi}}\renewcommand{\labelenumi}{(\theenumi)}\begin{enumerate}}%
{\end{enumerate}}%


\begin{document}

\section*{Background Reasearch for CDC 2016}

\subsection*{Summary Points}
\begin{itemize}
	\item Autonomous exploration governed by entropy:
	\begin{itemize}
		\item Frontier-based: move toward uncertain grid cells
		\item Expected entropy based on expected measurements: unclear how those are chosen
	\end{itemize}
	\item Other common problem: choosing movements based on map uncertainty AND localization uncertainty
\end{itemize}

\subsection*{Reference Information}
\begin{itemize}
	\item Autonomous Robotic Exploration Using Occupancy Grid Maps and Graph SLAM Based on Shannon and Renyi Entropy (CarDamKumCas15)
	\begin{itemize}
		\item Graph-based SLAM, occupancy grid
		\item tradeoff: localization and mapping
		\item The 
		\item Contribution: information-theoretic utility function: fuse uncertainties without manual tuning
		\item When identifying pose candidates, frontier-based exploration is typically used (Yam98, [2] in this paper)
		\item How do we get ``hallucinated'' measurements $\hat z$? Commonly through approximate ray-casting (no explanation)
		\item $\hat z$ obtained through ray casting ``This reduces the computational complexity by not considering all possible combinations of measurements for an action a.'' See (BurMooStaSch05, [1] in this paper) and (StaGriBur05, [3] in this paper) below.
	\end{itemize}
	\item A Frontier-Based Approach for Autonomous Exploration (Yam97)
	\begin{itemize}
		\item Define exploration: act of moving through an unknown environment while building a map that can be used for subsequent navigation
		\item Defined open, occupied, and unknown cells vaguely (no focus on the probability values)
		\item Process analogous to edge detection and region extraction in computer vision to find boundaries
		\item Any frontier above a minimum size is considered a frontier
		\item Robot goes to nearest accessible available frontier
		\item Then the robot does a $360^\text{o}$ sweep
	\end{itemize}
	\item Frontier-based Exploration Using Multiple Robots (Yam98)
	\begin{itemize}
		\item Central idea is a false assumption: ``To gain the most new information about the world, move to the boundary between open space and uncharted territory.''
		\item Paper is just about finding frontiers
	\end{itemize}
	\item Coordinated Multi-Robot Exploration (BurMooStaSch05)
	\begin{itemize}
		\item About optimizing multi-robot exploration with simultaneous coverage
		\item Moving toward frontiers: ``When exploring an unknown environment we are especially interested in “frontier cells” [63]. As a frontier cell we denote each already explored cell that is an immediate neighbor of an unknown, unexplored cell. If we direct a robot to such a cell, we can expect that it gains information about the unexplored area when it arrives at its target location." and "However, if there already is a robot that moves to a particular frontier cell, the utility of that cell can be expected to be lower for other robots."
		\item Not much on how we get expected measurements; only that ray-casting may be used to find where obstacles exist between frontier cells
	\end{itemize}	
	\item Information Gain-based Exploration Using Rao-Blackwellized Particle Filters (StaGriBur05)
	\begin{itemize}
		\item Rao-Blackwellized particle filter to simultaneously consider map and pose uncertainy to choose actions
		\item Consider uncertainty about the map (in paper: [2], [14], [23], [26], check: [27], [29])
		\item Not too useful because we don't use partical filters and $\hat z$ is based on that.
	\end{itemize}	
	\item Autonomous Exploration for 3D Map Learning (JohStaPfaBur07)
	\begin{itemize}
		\item About the tradeoff between poser uncertainty and expected information gain
		\item Again they look at the difference in entropy with expected measurements, but fail to explain how such expectations are calculated
	\end{itemize}	
	\item Dense Entropy Decrease Estimation for Mobile Robot Exploration (ValAnd14)
	\begin{itemize}
		\item Not very helpful
	\end{itemize}	
	\item Mapping and Exploration with Mobile Robots using Coverage Maps (StaBur03)
	\begin{itemize}
		\item ``we apply a ray-tracing technique similar to Moravec and Elfes [13] using the current maximum likelihood coverage map.''
	\end{itemize}
	\item  Probabilisitic Robotics (ThrBurFox05)
	\begin{itemize}
		\item Says that expected entropy is needed, but does not detail how this is done
		\item Totally based on particle filters
	\end{itemize}
	\item Principles of Robotic Motion (ChoLynHutKanBurKavThr05)
	\begin{itemize}
		\item Motion planning makes reference to the speed of execution (or some objective function along the way), whereas path planning does not
		\item Not one mention of entropy or information gain of the map
		\item Well-known, but not a great resource
	\end{itemize}	
	\item  ()
	\begin{itemize}
		\item 
	\end{itemize}
\end{itemize}


\end{document}



