\documentclass[letterpaper, 10pt, conference]{ieeeconf}
\IEEEoverridecommandlockouts \overrideIEEEmargins
\usepackage{amsmath,amssymb,url}
\usepackage{graphicx,subfigure,hyperref}
\usepackage{color}
%\usepackage{refcheck}

\newcommand{\norm}[1]{\ensuremath{\left\| #1 \right\|}}
\newcommand{\abs}[1]{\ensuremath{\left| #1 \right|}}
\newcommand{\bracket}[1]{\ensuremath{\left[ #1 \right]}}
\newcommand{\braces}[1]{\ensuremath{\left\{ #1 \right\}}}
\newcommand{\parenth}[1]{\ensuremath{\left( #1 \right)}}
\newcommand{\ip}[1]{\ensuremath{\langle #1 \rangle}}
\newcommand{\refeqn}[1]{(\ref{eqn:#1})}
\newcommand{\reffig}[1]{Fig. \ref{fig:#1}}
\newcommand{\tr}[1]{\mbox{tr}\ensuremath{\negthickspace\bracket{#1}}}
\newcommand{\trs}[1]{\mbox{tr}\ensuremath{\!\bracket{#1}}}
\newcommand{\deriv}[2]{\ensuremath{\frac{\partial #1}{\partial #2}}}
\newcommand{\G}{\ensuremath{\mathsf{G}}}
\newcommand{\SO}{\ensuremath{\mathsf{SO(3)}}}
\newcommand{\T}{\ensuremath{\mathsf{T}}}
\renewcommand{\L}{\ensuremath{\mathsf{L}}}
\newcommand{\so}{\ensuremath{\mathfrak{so}(3)}}
\newcommand{\SE}{\ensuremath{\mathsf{SE(3)}}}
\newcommand{\se}{\ensuremath{\mathfrak{se}(3)}}
\renewcommand{\Re}{\ensuremath{\mathbb{R}}}
\newcommand{\Sph}{\ensuremath{\mathsf{S}}}
\newcommand{\aSE}[2]{\ensuremath{\begin{bmatrix}#1&#2\\0&1\end{bmatrix}}}
\newcommand{\ase}[2]{\ensuremath{\begin{bmatrix}#1&#2\\0&0\end{bmatrix}}}
\newcommand{\D}{\ensuremath{\mathbf{D}}}
\renewcommand{\d}{\ensuremath{\mathbf{d}}}
\newcommand{\pair}[1]{\ensuremath{\left\langle #1 \right\rangle}}
\newcommand{\met}[1]{\ensuremath{\langle\!\langle #1 \rangle\!\rangle}}
\newcommand{\Ad}{\ensuremath{\mathrm{Ad}}}
\newcommand{\ad}{\ensuremath{\mathrm{ad}}}
\newcommand{\g}{\ensuremath{\mathfrak{g}}}

\title{\LARGE \bf
Geometric Adaptive Control for Aerial Transportation of a Rigid Body}
%Geometric control and differential flatness of a quadrotor {UAV} with a cable-suspended load on {$\SE\times\Sph^2$}}

\author{Evan Kaufman, Taeyoung Lee%\authorrefmark{1}%, Koushil Sreenath, and Vijay Kumar%\authorrefmark{2}%
\thanks{Evan Kaufman, Taeyoung Lee, Mechanical and Aerospace Engineering, George Washington University, Washington, DC 20052 {\tt tylee@gwu.edu}}%
%\thanks{Koushil Sreenath, Vijay Kumar, GRASP Lab, University of Pennsylvania, Philadelphia, PA 190104 {\tt \{koushils,kumar\}@seas.upenn.edu}}%
%\thanks{Vijay Kumar, GRASP Lab, University of Pennsylvania, Philadelphia, PA 190104 {\tt kumar@seas.upenn.edu}}%
\thanks{This research has been supported in part by NSF under the grants CMMI-1243000 (transferred from 1029551), CMMI-1335008, and CNS-1337722.}
%\thanks{\textsuperscript{\footnotesize\ensuremath{\dagger}}This research has been supported in part by .}
}

\newcommand{\EditTL}[1]{{\color{red}\protect #1}}
\renewcommand{\EditTL}[1]{{\protect #1}}


\newtheorem{definition}{Definition}
\newtheorem{lem}{Lemma}
\newtheorem{prop}{Proposition}
\newtheorem{remark}{Remark}


\begin{document}
\allowdisplaybreaks


\maketitle \thispagestyle{empty} \pagestyle{empty}

\begin{abstract}
This paper is focused on tracking control for a rigid body payload, that is connected to an arbitrary number of quadrotor unmanned aerial vehicles via rigid links. A geometric adaptive controller is constructed such that the payload asymptotically follows a given desired trajectory for its position and attitude in the presence of uncertainties. The coupled dynamics between the rigid body payload, links, and quadrotors are explicitly incorporated into control system design and stability analysis. These are developed directly on the nonlinear configuration manifold in a coordinate-free fashion to avoid singularities and complexities that are associated with local parameterizations. %The desirable features of the proposed control system are illustrated by a numerical example.
\end{abstract}

\section{Introduction}

%Quadrotor unmanned aerial vehicles have desirable features of simple mechanical structures, hovering capabilities, and high thrusts. It has been widely studied for various applications, such as aerial photography and commercials~\cite{GuiAck13}, and complex maneuvers are also demonstrated~\cite{LeeLeoPICDC10,GilHuaPICRA10,MelMicIJRR12}. In particular, the high thrust-to-weight ratio of quadrotors or multi-rotor aerial vehicles is useful for aerial load transportation.

%Aerial transport of payloads by towed cables is common in emergency response, industrial operations, and military missions. Examples of aerial transportation range from search and rescue missions where individuals are lifted from dangerous situations, to the delivery of heavy equipment to the top of a tall building that is difficult to reach by other means. 

%Transportation of a cable-suspended load has been studied traditionally for human-operated helicopters~\cite{CicKanJAHS95,BerPICRA09}. Recently, 

By utilizing the high thrust-to-weight ratio, quadrotor unmanned aerial vehicles have been envisaged for aerial load transportation. Most of the existing results for the control of quadrotors to transport a cable-suspended payload are based on the assumption that the dynamics of the payload is decoupled from the dynamics of quadrotors. For example, the effects of the payload are considered as arbitrary external forces and torques exerted to quadrotors. As such, these results may not be suitable for agile load transportation where the motion of cable and payload should be actively suppressed.

Recently, the full dynamic model for an arbitrary number of quadrotors transporting a payload has been developed, and based on that, geometric tracking controllers are constructed in an intrinsic fashion. In particular, autonomous transportation of a point mass connected to quadrotors via rigid links is developed in~\cite{LeeSrePICDC13}. It has been generalized into a more realistic dynamic model that considers the deformation of cables in~\cite{GooLeePACC14}, and also the attitude dynamics of a payload, that is considered as a rigid body instead of a point mass, is incorporated in~\cite{LeePICDC14}. However, these results are based on the assumption that the exact properties of the quadrotors and the payload are available, and that there are no external disturbances, thereby making it challenging to implement those results in  actual hardware systems. 

The objective of this paper is to construct a control system for an arbitrary number of quadrotors connected to a rigid body payload via rigid links with explicit consideration on uncertainties. A coordinate-free form of the equations of motion that have been developed in~\cite{LeePICDC14} is extended to include the effects of unknown forces and moments acting on each of the quadrotors, the cables, and the payload. A geometric nonlinear adaptive control system is designed such that both the position and the attitude of the payload asymptotically follow their desired trajectories, while maintaining a certain formation of quadrotors relative to the payload. 

The unique property is that the coupled dynamics of the payload, the cables, and quadrotors are explicitly incorporated in control system design for agile load transportation, where the motions of the payload relative to the quadrotors are excited nontrivially. Another distinct feature is that the equations of motion and the control systems are developed directly on the nonlinear configuration manifold intrinsically. Therefore, singularities of local parameterization are completely avoided.

As such, the proposed control system is particularly useful for rapid and safe payload transportation in complex terrain, where the position and attitude of the payload should be controlled concurrently. Most of the existing control systems of aerial load transportation suffer from limited agility as they are based on reactive assumptions that ignore the inherent complexities in the dynamics of aerial load transportation. The proposed control system explicitly integrates the comprehensive dynamic characteristics to achieve extreme maneuverability in aerial load transportation. Nonlinear adaptive tracking controls of a cable-suspended rigid body with uncertainties have not been studied as mathematically rigorously as presented in this paper. 

\begin{figure}
\centerline{
		\includegraphics[width=0.8\columnwidth]{Quad_N_Rgbd_Adaptive_3d0}
}
%\centerline{
%	\subfigure[Top view]{
%	\setlength{\unitlength}{0.1\columnwidth}\scriptsize
%	\begin{picture}(9.5,3.2)(0,0)
%	\put(0,0){\includegraphics[width=0.95\columnwidth]{Quad_N_Rgbd_Adaptive_3d2}}
%	\put(8.4,3.2){$t=0$}
%	\put(6.0,3.2){$t=3.3$}
%	\put(2.2,-0.2){$t=6.6$}
%	\put(0.2,2.3){$t=10$}
%	\put(2.2,3.2){$t=13.3$}
%	\put(5.8,-0.2){$t=16.6$}
%	\put(7.9,0.4){$t=20$}
%	\end{picture}}	
%}
%\centerline{
%	\subfigure[Side view]{
%		\includegraphics[width=0.95\columnwidth]{Quad_N_Rgbd_Adaptive_3d1}}
%}
\caption{Snapshots of controlled maneuver (red:desired trajectory, blue:actual trajectory). A short animation illustrating this maneuver is available at {\href{http://youtu.be/nOWErfdzZLU}{http://youtu.be/nOWErfdzZLU}}.}\label{fig:SS}
\end{figure}


\vfill
\bibliography{/Users/tylee/Documents/BibMaster}
\bibliographystyle{IEEEtran}

\end{document}

%This paper is organized as follows. A dynamic model is presented and the problem is formulated at Section \ref{sec:DM}. Control systems are constructed at Sections \ref{sec:SDM} and \ref{sec:FDM}, which are followed by a numerical example at Section V. 


%
%
%These properties of quadrotors are also desirable for load carrying and transportation. %Areal transportation of a cable-suspended load has been studied traditionally for helicopters~\cite{CicKanJAHS95,BerPICRA09}. 
%Small-size single or multiple autonomous vehicles are considered for load transportation and deployment~\cite{PalCruIRAM12,MicFinAR11,MazKonJIRS10}. Nonlinear tracking control systems are developed for a single quadrotor UAV with a cable-suspended load in~\cite{SreMicPICRA13} and a companion paper~\cite{SreLeePICDC13}.
%
%Load transportation with multiple quadrotors is useful when the load is heavy compared with the maximum thrust of a single quadrotor, or when additional redundancy is required for safety. But, this is challenging since \textit{dynamically coupled} quadrotors should cooperate safely to transport load. This is in contrast to the existing results on formation control of decoupled multi-agent systems. 
%
%In this paper, we consider an arbitrary number of quadrotors that are connected to a point mass via rigid links. The equations of motion are derived from the variational principle, and control systems are developed such that the point mass asymptotically follows a given smooth desired trajectory. Two formation flight modes are introduced to control the formation of quadrotors with respect to the point mass.  In the existing control systems for a load-carrying quadrotor, such as~\cite{PalCruIRAM12}, a quadrotor is designed to follow pre-computed minimum swing trajectories while rejecting the force and moment exerted by the load that are considered as disturbances. The control systems proposed in this paper explicitly consider the coupling effects between the load dynamics and the dynamics of multiple quadrotors for safe transportation along complex trajectories. 
%
%Another distinct feature is that the equations of motion and the control systems are developed directly on the nonlinear configuration manifold in a coordinate-free fashion. This yields remarkably compact expressions for the dynamic model and controllers, compared with local coordinates that often require symbolic computational tools due to complexity of multibody systems. Furthermore, singularities of local parameterization are completely avoided to generate agile maneuvers in a uniform way.
%
%If the links are assumed to be rigid, the proposed control system is also applied to the cases where selected quadrotors are below the load, to obtain so-called \textit{flying inverted pendulum}. Linear control systems have been developed and implemented to stabilize few selected nominal trajectories in~\cite{HehDAnPICRA11}. The proposed control system is based on the full nonlinear dynamics, and it guarantees exponential stability of flying inverted pendulum for arbitrary desired trajectories.
%
%Compared with the companion paper~\cite{SreLeePICDC13} that is focused on tracking, differential flatness, and experiments of a single quadrotor with a suspended load, this paper proposes a cooperative framework of an arbitrary number of quadrotors.
%%The paper is organized as follows. We develop a globally defined dynamic in Section \ref{sec:DM}. Tracking control results are presented in Section \ref{sec:CSD}, followed by numerical examples. %.  Several numerical results are presented in Section IV.% Proofs are relegated to the Appendix.



\section{Problem Formulation}\label{sec:DM}


Consider $n$ quadrotor UAVs that are connected to a payload, that is modeled as a rigid body, via massless links (see Figure \ref{fig:DM}). Throughout this paper, the variables related to the payload is denoted by the subscript $0$, and the variables for the $i$-th quadrotor are denoted by the subscript $i$, which is assumed to be an element of $\mathcal{I}=\{1,\cdots\, n\}$ if not specified. We choose an inertial reference frame $\{\vec e_1,\vec e_2,\vec e_3\}$ and body-fixed frames $\{\vec b_{j_1},\vec b_{j_2},\vec b_{j_3}\}$ for $0\leq j\leq n$ as follows. For the inertial frame, the third axis $\vec e_3$ points downward along the gravity and the other axes are chosen to form an orthonormal frame. %The origin of the $j$-th body-fixed frame is located at the center of mass of the payload for $j=0$ and at the mass center the quadrotor for $1\leq j\leq n$. The third body-fixed axis $\vec b_{i_3}$ is normal to the plane defined by the centers of rotors, and it points downward.

\begin{figure}
\centerline{\setlength{\unitlength}{0.1\columnwidth}\footnotesize
\begin{picture}(10,6.5)(0.0,0)
\put(1.25,0){\includegraphics[width=0.75\columnwidth]{Quad_N_Rgbd_Adaptive_3d.pdf}}
\put(2.35,1.45){$\vec e_1$}
\put(1.9,0.3){$\vec e_2$}
\put(1.4,-0.1){$\vec e_3$}
\put(3.1,1.0){$x_0\in\Re^3$}
\put(6.7,0.5){\shortstack[c]{$m_0,J_0$\\$R_0\in\SO$}}
\put(5.3,5.3){\shortstack[c]{$m_i,J_i$\\$R_i\in\SO$}}
\put(7.8,4.2){$q_i\in\Sph^2$}
\put(6.8,1.8){$\rho_i$}
\put(7.65,3.3){$l_i$}
\end{picture}}
\caption{Dynamics model: $n$ quadrotors are connect to a rigid body $m_0$ via massless links $l_i$. The configuration manifold is $\Re^3\times\SO\times(\Sph^2\times\SO)^n$.}\label{fig:DM}
\vspace*{-0.1cm}
\end{figure}


The location of the mass center of the payload is denoted by $x_0\in\Re^3$, and its attitude is given by $R_0\in\SO$, where the special orthogonal group is defined by $\SO=\{R\in\Re^{3\times 3}\,|\, R^TR=I,\,\mathrm{det}[R]=1\}$. Let $\rho_i\in\Re^3$ be the point on the payload where the $i$-th link is attached, and it is represented with respect to the zeroth body-fixed frame. The other end of the link is attached to the mass center of the $i$-th quadrotor. The direction of the link from the mass center of the $i$-th quadrotor toward the payload is defined by the unit-vector $q_i\in\Sph^2$, where $\Sph^2=\{q\in\Re^3\,|\,\|q\|=1\}$, and the length of the $i$-th link is denoted by $l_i\in\Re$.

Let $x_i\in\Re^3$ be the location of the mass center of the $i$-th quadrotor with respect to the inertial frame. As the link is assumed to be rigid, we have $x_i=x_0+R_0\rho_i-l_i q_i$. The attitude of the $i$-th quadrotor is defined by $R_i\in\SO$, which represents the linear transformation of the representation of a vector from the $i$-th body-fixed frame to the inertial frame. 

In summary, the configuration of the presented system is described by the position $x_0$ and the attitude $R_0$ of the payload, the direction $q_i$ of the links, and the attitudes $R_i$ of the quadrotors. The corresponding configuration manifold of this system is $\Re^3\times\SO\times(\Sph^2\times \SO)^n$.

The mass and the inertia matrix of the payload are denoted by $m_0\in\Re$ and $J_0\in\Re^{3\times 3}$, respectively. The dynamic model of each quadrotor is identical to~\cite{LeeLeoPICDC10}. The mass and the inertia matrix of the $i$-th quadrotor are denoted by $m_i\in\Re$ and $J_i\in\Re^{3\times 3}$, respectively. The $i$-th quadrotor can generates a thrust $-f_iR_ie_3\in\Re^3$ with respect to the inertial frame, where $f_i\in\Re$ is the total thrust magnitude and $e_3=[0,0,1]^T\in\Re^3$. It also generates a moment $M_i\in\Re^3$ with respect to its body-fixed frame. The control input of this system corresponds to $\{f_i,M_i\}_{1\leq i\leq n}$.

In this paper, we include external disturbances as follows. Let $\Delta_{x_0},\Delta_{R_0}\in\Re^3$ be arbitrary, but fixed disturbance force and moment acting on the mass center of the payload, respectively. The disturbance force and moment exerted on the $i$-th quadrotor are denoted by $\Delta_{x_i},\Delta_{R_i}\in\Re^3$, respectively. The disturbance forces are represented with respect to the inertial frame, and the disturbance moments are represented with respect to the corresponding body-fixed frame. 

Throughout this paper, the 2-norm of a matrix $A$ is denoted by $\|A\|$, and its maximum eigenvalue and minimum eigenvalues are denoted by $\lambda_{M}[A]$ and $\lambda_{m}[A]$, respectively. The standard dot product is denoted by $x \cdot y = x^Ty$ for any $x,y\in\Re^3$.




\subsection{Equations of Motion}

The kinematic equations for the payload, quadrotors, and links are given by
\begin{gather}
\dot q_{i} = \omega_i\times q_i=\hat\omega_i q_i,\label{eqn:dotqi}\\
\dot R_0  = R_0\hat\Omega_0,\quad \dot R_i  = R_i\hat\Omega_i,\label{eqn:dotRi}
\end{gather}
where $\omega_i\in\Re^3$ is the angular velocity of the $i$-th link, satisfying $q_i\cdot\omega_i=0$, and $\Omega_0$ and $\Omega_i\in\Re^3$ are the angular velocities of the payload and the $i$-th quadrotor expressed with respect to its body-fixed frame, respectively. The \textit{hat map} $\hat\cdot:\Re^3\rightarrow\so$ is defined by the condition that $\hat x y=x\times y$ for all $x,y\in\Re^3$, and the inverse of the hat map is denoted by the \textit{vee map} $\vee:\so\rightarrow\Re^3$, where $\so$ denotes the set of $3\times 3$ skew-symmetric matrices, i.e., $\so=\{S\in\Re^{3\times 3}\,|\, S^T=-S\}$, and it corresponds to the Lie algebra of $\SO$. %Several properties of the hat map are summarized as follows~\cite{ShuJAS93}:
%\begin{gather}
%    \hat x y = x\times y = - y\times x = - \hat y x,\\
%%    \hat x^T \hat x = (x^T x) I - x x^T,\\
%%    \hat x \hat y \hat x=-(y^Tx)\hat x,\\
%%    -\frac{1}{2}\tr{\hat x \hat y} = x^T y,\\
%%    \widehat{x\times y} = \hat x \hat y -\hat y \hat x = yx^T-xy^T,\\
%    %\tr{\hat x A}=
%    \tr{A\hat x }=\frac{1}{2}\tr{\hat x (A-A^T)}=-x^T (A-A^T)^\vee,\label{eqn:hat1}\\
%%    \widehat{Ax} = \hat x \parenth{\frac{1}{2}\tr{A}I-A} + \parenth{\frac{1}{2}\tr{A}I-A}^T \hat x,\\
%    \hat x  A+A^T\hat x=(\braces{\tr{A}I_{3\times 3}-A}x)^{\wedge},\label{eqn:xAAx}\\
%R\hat x R^T = (Rx)^\wedge,\label{eqn:RxR}
%\end{gather}
%for any $x,y\in\Re^3$, $A\in\Re^{3\times 3}$, and $R\in\SO$.



We derive equations of motion according to Lagrangian mechanics. The velocity of the $i$-th quadrotor is given by $\dot x_i = \dot x_0+\dot R_0\rho_i - l_i\dot q_i$. The kinetic energy of the system is composed of the translational kinetic energy and the rotational kinetic energy of the payload and quadrotors:
\begin{align}
\mathcal{T} & = \frac{1}{2}m_0 \|\dot x_0\|^2 + \frac{1}{2}\Omega_0\cdot J_0\Omega_0\nonumber\\
&\quad +\sum_{i=1}^n \frac{1}{2} m_i \|\dot x_0+\dot R_0\rho_i - l_i\dot q_i\|^2+ \frac{1}{2}\Omega_i\cdot J_i\Omega_i.\label{eqn:TT}
\end{align}
The gravitational potential energy is given by
\begin{align}
\mathcal{U} = - m_0g e_3 \cdot x_0 - \sum_{i=1}^n m_ige_3\cdot (x_0+R_0\rho_i-l_iq_i),\label{eqn:UU}
\end{align}
where the unit-vector $e_3$ points downward along the gravitational acceleration as shown at \reffig{DM}. The resulting Lagrangian of the system is $\mathcal{L}=\mathcal{T}-\mathcal{U}$. 

The corresponding Euler-Lagrange equations have been developed according to Hamilton's principle in~\cite{LeePICDC14},  Here, it is generalized to include the effects of disturbances via the Lagrange-d'Alembert principle. Let the action integral be $\mathfrak{G}=\int_{t_0}^{t_f} \mathcal{L}\, dt$. Next, let the total control thrust at the $i$-th quadrotor with respect to the inertial frame be denoted by $u_i = -f_iR_ie_3\in\Re^3$ and the total control moment at the $i$-th quadrotor is defined as $M_i\in\Re^3$. There exist the disturbances $\Delta_{x_0},\Delta_{R_0}$ for the payload, and the disturbances $\Delta_{x_i},\Delta_{R_i}$ for the $i$-th quadrotor. The corresponding virtual work can be written as
\begin{align*}
\delta\mathcal{W} & = \int_{t_0}^{t_f} \sum_{i=1}^n (u_i+\Delta_{x_i})\cdot\braces{\delta x_0 + R_0\hat\eta_0\rho_i -l_i\xi_i\times q_i}\\
&\quad + \sum_{i=1}^n(M_i+\Delta_{R_i}) \cdot\eta_i+\Delta_{x_0}\cdot\delta x_0 + \Delta_{R_0}\cdot \eta_0\,dt.
\end{align*}
The Lagrange-d'Alembert principle states that $\delta\mathfrak{G}=-\delta\mathcal{W}$ for any variation of trajectories with fixed end points. This yields the following equations of motion (see~\cite{LeeACC151ext} for detailed derivations),
\begin{gather}
M_q (\ddot x_0-ge_3) - \sum_{i=1}^n m_iq_iq_i^T R_0\hat\rho_i\dot\Omega_0 = \Delta_{x_0}\nonumber\\
 +\sum_{i=1}^n u_i^\parallel+\Delta_{x_i}^\parallel-m_il_i\|\omega_i\|^2q_i- m_iq_iq_i^T R_0\hat\Omega_0^2\rho_i,
\label{eqn:ddotx0}\\
%
  (J_0-\sum_{i=1}^n m_i\hat\rho_i R_0^T q_iq_i^T R_0\hat\rho_i)\dot\Omega_0 \nonumber\\
 + \sum_{i=1}^n m_i\hat\rho_i R_0^Tq_iq_i^T(\ddot x_0-ge_3) + \hat\Omega_0   J_0\Omega_0
=\Delta_{R_0}\nonumber \\
+\sum_{i=1}^n \hat\rho_i R_0^T (u_i^\parallel+\Delta_{x_i}^\parallel-m_il_i\|\omega_i\|^2 q_i-m_iq_iq_i^TR_0\hat\Omega_0^2\rho_i),\label{eqn:dotW0}\\
%
\dot\omega_i =\frac{1}{l_i}\hat q_i(\ddot x_0-ge_3 - R_0\hat\rho_i\dot\Omega_0+
R_0\hat\Omega_0^2\rho_i)\nonumber\\
 -\frac{1}{m_il_i}\hat q_i(u_i^\perp+\Delta_{x_i}^\perp),\label{eqn:dotwi}\\
% \dot\omega_i = \frac{1}{l_i}\hat q_i (\ddot y-ge_3) -\frac{1}{m_il_i} \hat q_i u_i^\perp,\label{eqn:dotwi}\\
J_i\dot\Omega_i + \Omega_i\times J_i\Omega_i = M_i+\Delta_{R_i},\label{eqn:dotWi}
\end{gather}
where $M_q=m_y I + \sum_{i=1}^n m_i q_i q_i^T\in\Re^{3\times 3}$, which is symmetric, positive-definite for any $q_i$. 

Recall the vector $u_i\in\Re^3$ represents the control force at the $i$-th quadrotor, i.e., $u_i=-f_i R_i e_3$.  The vectors $u_i^\parallel$ and $u_i^\perp\in\Re^3$ denote the orthogonal projection of $u_i$ along $q_i$, and the orthogonal projection of $u_i$ to the plane  normal to $q_i$, respectively, i.e.,
\begin{align}
u_i^\parallel & = q_iq_i^T u_i,\\
u_i^\perp &= -\hat q_i^2 u_i =(I-q_iq_i^T) u_i.
\end{align}
Therefore, $u_i = u_i^\parallel + u_i^\perp$. Throughout this paper, the subscripts $\parallel$ and $\perp$ of a vector denote the component of the vector that is parallel to $q_i$ and the other component of the vector that is perpendicular to $q_i$. Similarly, the disturbance force at the $i$-th quadrotor is decomposed as
\begin{align}
\Delta_{x_i}^\parallel & = q_iq_i^T \Delta_{x_i},\label{eqn:delxiparallel}\\
\Delta_{x_i}^\perp & = -\hat q_i^2 \Delta_{x_i} = (I-q_iq_i^T) \Delta_{x_i}.\label{eqn:delxiperp}
\end{align}

\subsection{Tracking Problem}

Define a fixed matrix $\mathcal{P} \in \Re^{6\times 3n}$ as 
\begin{align}
\mathcal{P} = \begin{bmatrix} I_{3\times 3} & \cdots & I_{3\times 3} \\ \hat\rho_1 & \cdots & \hat\rho_n \end{bmatrix}. \label{eqn:P}
\end{align}
Recall that $\rho_i$ describe the point on the payload where the $i$-th link is attached. 
Assume the links are attached to the payload such that
\begin{align}
\mathrm{rank}[\mathcal{P}] \geq 6.\label{eqn:A1}
\end{align}
This is to guarantee that there exist enough degrees of freedom in control inputs for both the translational motion and the rotational maneuver of the payload. The assumption \refeqn{A1} requires that the number of quadrotor is at least three, i.e., $n\geq 3$.

It is also assumed that the bounds of the disturbance forces and moments are available, i.e., for a known positive constant $B_{\delta}\in\Re$, we have
\begin{gather}
\max\{\|\Delta_{x_{0}}\|,\,
\|\Delta_{R_{0}}\|,\,
\|[\Delta_{x_{1}}^T,\cdots\Delta_{x_n}^T]^T\|,\,\|\Delta_{R_i}\|\} < B_{\delta}.\label{eqn:Bdelta}
\end{gather}

Suppose that the desired trajectories for the position and the attitude of the payload are given as smooth functions of time, namely $x_{0_d}(t)\in\Re^3$ and $R_{0_d}(t)\in\SO$. From the attitude kinematics equation, we have
\begin{align*}
\dot R_{0_d}(t) = R_{0_d}(t) \hat\Omega_{0_d}(t),
\end{align*}
where $\Omega_{0_d}(t)\in\Re^3$ corresponds to the desired angular velocity of the payload. It is assumed that the velocity and the acceleration of the desired trajectories are bounded by known constants.

We wish to design a control input of each quadrotor $\{f_i,M_i\}_{1\leq i\leq n}$ such that  the tracking errors asymptotically converge to zero along the solution of the controlled dynamics.

\section{Control System Design For Simplified Dynamic Model}\label{sec:SDM}

In this section, we consider a simplified dynamic model where the attitude dynamics of each quadrotor is ignored, and we design a control input by assuming that the thrust at each quadrotor, namely $u_i$ can be arbitrarily chosen. It corresponds to the case where every quadrotor is replaced by a fully actuated aerial vehicle that can generates a thrust along any direction arbitrarily. The effects of the attitude dynamics of quadrotors will be incorporated in the next section. 

In the simplified dynamic model given by \refeqn{ddotx0}-\refeqn{dotwi}, the dynamics of the payload are affected by the parallel components $u_i^\parallel$ of the thrusts, and the dynamics of the links are directly affected by the normal components $u_i^\perp$ of the thrusts. This structure motivates the following control system design procedure: first, the parallel components $u_i^\parallel$ are chosen such that the payload follows the desired position and attitude trajectory while yielding the desired direction of each link, namely $q_{i_d}\in\Sph^2$; next, the normal components $u_i^\perp$ are designed such that the actual direction of the links $q_i$ follows the desired direction $q_{i_d}$.

\subsection{Design of Parallel Components}

Let $a_i\in\Re^3$ be the acceleration of the point on the payload where the $i$-th link is attached, that is measured relative to the gravitational acceleration:
\begin{align}
a_i = \ddot x_0 - ge_3 + R_0\hat\Omega_0^2\rho_i -R_0\hat\rho_i\dot\Omega_0.\label{eqn:ai0}
\end{align}
The parallel component of the control input is chosen as
\begin{align}
u_i^\parallel & = \mu_i + m_il_i\|\omega_i\|^2q_i + m_i q_iq_i^T a_i,\label{eqn:uip}
\end{align}
where $\mu_i\in\Re^3$ is a virtual control input that is designed later, with a constraint that $\mu_i$ is parallel to $q_i$. Note that the expression of $u_i^\parallel$ is guaranteed to be parallel to $q_i$ due to the projection operator $q_iq_i^T$ at the last term of the right-hand side of the above expression.

The motivation for the proposed parallel components becomes clear if \refeqn{uip} is substituted into \refeqn{ddotx0}-\refeqn{dotW0} and rearranged to obtain
\begin{gather}
m_0 (\ddot x_0 -g e_3) = \Delta_{x_0}+\sum_{i=1}^n (\mu_i+\Delta_{x_i}^\parallel),\label{eqn:ddotx0s}\\
J_0\dot\Omega_0 +\hat\Omega_0 J_0\Omega_0 = \Delta_{R_0}+\sum_{i=1}^n \hat\rho_i R_0^T (\mu_i+\Delta_{x_i}^\parallel).\label{eqn:dotW0s}
\end{gather}
Therefore, considering a free-body diagram of the payload, the virtual control input $\mu_i$ corresponds to the force exerted to the payload by the $i$-link, or the tension of the $i$-th link in the absence of disturbances. 

%When there is no control force from each quadrotor, i.e., $u_i^\parallel=0$, the tension of the $i$-th link is composed of the projected relative inertial force at the point where the $i$-th link is attached to the payload and the centrifugal force due to the rotation of the link. Substituting \refeqn{ddotx0s} and \refeqn{dotW0s} back into \refeqn{ai0}, we obtain
%\begin{align}
%a_i & = \frac{1}{m_0}(\Delta_{x_0}+\sum_{j=1}^n(\mu_j+\Delta_{x_j}^\parallel))+R_0\hat\Omega_0^2\rho_i\nonumber\\
%&\quad + R_0\hat\rho_iJ_0^{-1}(\hat\Omega_0 J_0\Omega_0-\Delta_{R_0} - \sum_{j=1}^n \hat\rho_j R_0^T(\mu_j+\Delta_{x_j}^\parallel)).\label{eqn:ai}
%%\\
%%u_i^\parallel & = \mu_i + m_il_i\|\omega_i\|^2q_i +\frac{m_i}{m_0}q_iq_i^T\sum_{j=1}^n\mu_j \nonumber\\
%%&\quad+m_iq_iq_i^T R_0 \{\hat\Omega_0^2 \rho_i+\hat\rho_iJ_0^{-1}(\hat\Omega_0 J_0\Omega_0 - \sum_{j=1}^n \hat\rho_j R_0^T\mu_j)\},\label{eqn:uip}
%\end{align}
%In short, the parallel component of the control input $u_i^\parallel$ is determined by the virtual control input $\mu_i$.

Next, we determine the virtual control input $\mu_i$. As in~\cite{GooLeePECC13}, define position, attitude, and angular velocity tracking error vectors $e_{x_0},e_{R_0},e_{\Omega_0}\in\Re^3$ for the payload as
\begin{align*}
e_{x_0} & = x_0 -x_{0_d},\\
e_{R_0} & = \frac{1}{2} (R_{0_d}^T R_0-R_0^T R_{0_d})^\vee,\\
e_{\Omega_0} & = \Omega_0 - R_0^T R_{0_d}\Omega_{0_d}.
\end{align*}
The desired resultant control force $F_d\in\Re^3$ and moment $M_d\in\Re^3$ acting on the payload are given as
\begin{align}
F_d & = m_0 (-k_{x_0}e_{x_0} - k_{\dot x_0} \dot e_{x_0} + \ddot x_{0_d} - g e_3)\nonumber\\
&\quad -\bar\Delta_{x_0} - \sum_{i=1}^n\bar\Delta_{x_i}^\parallel,\label{eqn:Fd}\\
M_d & = -k_{R_0} e_{R_0} - k_{\Omega_0} e_{\Omega_0}   +(R_0^TR_{0_d}\Omega_{0_d})^\wedge J_0 R_0^TR_{0_d} \Omega_{0_d}\nonumber\\
& \quad + J_0 R_0^T R_{0_d} \dot\Omega_{0_d}-\bar\Delta_{R_0}-\sum_{i=1}^n\hat\rho_iR_0 \bar\Delta_{x_i}^\parallel,\label{eqn:Md}
\end{align}
for positive constants $k_{x_0},k_{\dot x_0},k_{R_0},k_{\Omega_0}\in\Re$. Here, the estimates of $\Delta_{x_0},\Delta_{x_i},\Delta_{R_0}$ are denoted by $\bar\Delta_{x_0},\bar\Delta_{x_i},\bar\Delta_{R_0}\in\Re^3$, and the orthogonal projection of $\bar\Delta_{x_i}$ along $q_i$ is denoted by $\bar\Delta_{x_i}^\parallel = q_iq_i^T \bar\Delta_{x_i}\in\Re^3$. Adaptive control laws to update the estimates of disturbances are introduced later at Section \ref{sec:AL}.

These are the ideal resultant force and moment to achieve the control objectives. One may try to choose the virtual control input $\mu_i$ by making the expressions in the right-hand sides of \refeqn{ddotx0s} and \refeqn{dotW0s}, namely $\sum_{i}\mu_i$ and $\sum_i\hat\rho_iR_0^T\mu_i$, become identical to $F_d$ and $M_d$, respectively. But, this is not valid in general, as each $\mu_i$ is constrained to be parallel to $q_i$. Instead, we choose the desired value of $\mu_i$, without any constraint, such that 
\begin{align}
\sum_{i=1}^n \mu_{i_d} = F_d,\quad \sum_{i=1}^n \hat\rho_i R_0^T \mu_{i_d} = M_d,\label{eqn:muid0}
\end{align}
or equivalently, using the matrix $\mathcal{P}$ defined at \refeqn{P},
\begin{align*}
\mathcal{P}
\begin{bmatrix}
R_0^T\mu_{1_d} \\ \vdots \\ R_0^T\mu_{n_d}
\end{bmatrix}
=
\begin{bmatrix}
R_0^T F_d \\ M_d
\end{bmatrix}.
\end{align*}
From the assumption stated at \refeqn{A1}, there exists at least one solution to the above matrix equation for any $F_d, M_d$. Here, we find the minimum-norm solution given by
\begin{align}
\begin{bmatrix} \mu_{1_d}\\\vdots\\\mu_{n_d} \end{bmatrix} 
= \mathrm{diag}[R_0,\cdots R_0]\;\mathcal{P}^T (\mathcal{P}\mathcal{P}^T)^{-1}\begin{bmatrix} R_0^T F_d\\M_d\end{bmatrix}.\label{eqn:muid}
\end{align}
The virtual control input $\mu_i$ is selected as the projection of its desired value $\mu_{i_d}$ along $q_i$,
\begin{align}
\mu_i = (\mu_{i_d}\cdot q_i) q_i=q_iq_i^T\mu_{i_d},\label{eqn:mui}
\end{align}
and the desired direction of each link, namely $q_{i_d}\in\Sph^2$ is defined as
\begin{align}
q_{i_d} = -\frac{\mu_{i_d}}{\|\mu_{i_d}\|}.\label{eqn:qid}
\end{align}
It is straightforward to verify that when $q_{i}=q_{i_d}$, the resultant force and moment acting on the payload become identical to their desired values.

%Here, the extra degrees of freedom in control inputs are used to minimize the magnitude of the desired tension at \refeqn{muid}, but they can be applied to other tasks, such as controlling the relative configuration of links~\cite{LeeSrePICDC13}. This is referred to future investigation.

\subsection{Design of Normal Components}

Substituting \refeqn{ai0} into \refeqn{dotwi} and using \refeqn{delxiperp}, the equation of motion for the $i$-link is given by
\begin{align}
\dot\omega_i & = \frac{1}{l_i}\hat q_i a_i-\frac{1}{m_il_i}\hat q_i (u_i^\perp+\Delta_{x_i}^\perp).\label{eqn:widotf0}
\end{align}
Here, the normal component of the control input $u_i^\perp$ is chosen such that $q_i\rightarrow q_{i_d}$ as $t\rightarrow\infty$. Control systems for the unit-vectors on the two-sphere have been studied in~\cite{BulLew05,Wu12}. In this paper, we adopt the control system developed in terms of the angular velocity in~\cite{Wu12}, and we augment it with an adaptive control term to handle the disturbance $\Delta_{x_i}^\perp$.

For the given desired direction of each link, its desired angular velocity is obtained from the kinematics equation as
\begin{align*}
\omega_{i_d} = q_{i_d}\times \dot q_{i_d}.%\label{eqn:wid}
\end{align*}
Define the direction and the angular velocity tracking error vectors for the $i$-th link, namely $e_{q_i},e_{\omega_i}\in\Re^3$ as 
\begin{align*}
e_{q_i} & = q_{i_d}\times q_i,\\
e_{\omega_i} & = \omega_i + \hat q_i^2\omega_{i_d}.
\end{align*}
For positive constants $k_{q},k_{\omega}\in\Re$, the normal component of the control input is chosen as
\begin{align}
u_i^\perp & = m_il_i\hat q_i \{-k_q e_{q_i} -k_{\omega}e_{\omega_i} -(q_i\cdot\omega_{i_d})\dot q_i -\hat q_i^2\dot\omega_d\}\nonumber\\
&\quad - m_i\hat q_i^2 a_i- \bar\Delta_{x_i}^\perp,\label{eqn:uiperp}
\end{align}
where $\bar\Delta_{x_i}^\perp\in\Re^3$ corresponds to the component of the estimate $\bar\Delta_{x_i}$ that is projected onto the plane normal to $q_i$, i.e., $\bar\Delta_{x_i}^\perp=-\hat q_i^2 \bar\Delta_{x_i}$.

Note that the expression of $u_i^\perp$ is perpendicular to $q_i$ by definition. Substituting \refeqn{uiperp} into \refeqn{widotf0}, and rearranging by the facts that the matrix $-\hat q_i^2$ corresponds to the orthogonal projection to the plane normal to $q_i$ and $\hat q_i^3=-\hat q_i$, we obtain
\begin{align}
\dot\omega_i & = -k_q e_{q_i} -k_{\omega}e_{\omega_i} -(q_i\cdot\omega_{i_d})\dot q_i -\hat q_i^2\dot\omega_d\nonumber\\
&\quad - \frac{1}{m_il_i}\hat q_i \tilde\Delta_{x_i}^\perp,\label{eqn:dotwif}
\end{align}
where the estimation error is defined as $\tilde\Delta_{x_i}^\perp=\Delta_{x_i}^\perp-\bar\Delta_{x_i}^\perp\in\Re^3$.

In short, the control force for the simplified dynamic model is given by
\begin{align}
u_i = u_i^\parallel + u_i^\perp.\label{eqn:ui}
\end{align}


\subsection{Design of Adaptive Law}\label{sec:AL}
The above control inputs require estimates of the uncertainties. They are updated according to the following adaptive laws,
\begin{align}
\dot{\bar \Delta}_{x_0} & = \frac{h_{x_0}}{m_0}(\dot e_{x_0}+c_x e_{x_0}),\label{eqn:hatDx0_dot}\\
\dot{\bar \Delta}_{R_0} & = h_{R_0}(e_{\Omega_0}+ c_R e_{R_0}),\label{eqn:hatDR0_dot}\\
\dot{\bar \Delta}_{x_i} & = h_{x_i}q_iq_i^T\{\frac{1}{m_0}(\dot e_{x_0}+c_x e_{x_0})
-R_0\hat\rho_i(e_{\Omega_0}+c_R e_{R_0})\}\nonumber\\
&\quad +  \frac{h_{x_i}}{m_il_i}\hat q_i (e_{\omega_i} + c_q e_{q_i}),\label{eqn:hatDxi_dot}
\end{align}
for positive constants $c_x,c_R,c_q\in\Re$ and adaptive gains $h_{x_0},h_{R_0},h_{x_i}\in\Re$. Note that the adaptive law for $\Delta_{x_i}$ is composed of two parts: the first part that is parallel to $q_i$ determined by the tracking errors of the payload, and the second part that is normal to $q_i$ defined by the dynamics of the link. 

The resulting stability properties are summarized as follows.

\begin{prop}\label{prop:SDM}
Consider the simplified dynamic model defined by \refeqn{ddotx0}-\refeqn{dotwi}. For given tracking commands $x_{0_d},R_{0_d}$, a control input is designed as \refeqn{ui}-\refeqn{hatDxi_dot}. Then, there exist the values of controller gains and controller parameters such that the following properties are satisfied.

\begin{itemize}
\item[(i)] The zero equilibrium of tracking errors $(e_{x_0},\dot e_{x_0},e_{R_0},e_{\Omega_0},e_{q_i},e_{\omega_i})$ and the estimation errors $(\bar\Delta_{x_0},\bar\Delta_{R_0},\bar\Delta_{x_i})$ is stable in the sense of Lyapunov.
\item[(ii)] The tracking errors asymptotically coverage to zero.
\item[(iii)] The estimation errors are uniformly bounded.
\end{itemize}
\end{prop}
\begin{proof}
Due to the page limit, the proof is relegated to~\cite{LeeACC151ext}.
\end{proof}


%At \refeqn{qid}, the negative sign appeared to make the tension at each cable positive when $q_i=q_{i_d}$.  Assuming that the tracking errors $e_{x_0},\dot e_{x_0},e_{R_0},e_{\Omega_0}$ and the variables $\ddot x_{0_d},\Omega_{0_d},\dot\Omega_{0_d}$ obtained from the desired trajectories are sufficiently small, this guarantees that quadrotors remain above the payload. If desired, the negative sign at \refeqn{qid} can be eliminated to place quadrotors below the payload, resulting in a tracking control of an inverted rigid body multi-link pendulum, that can be considered as a generalization of a flying spherical  inverted spherical pendulum illustrated in~\cite{LeeSrePICDC13}. 

%The proposed control system guarantees stability in the sense of Lyapunov and asymptotic convergence of tracking errors variables, but as the convergence of estimation error is not guaranteed, asymptotic stability is not achieved. However, in the absence of the disturbances, we can achieve stronger exponential stability by eliminating the adaptive parts of the proposed control system via setting $\bar\Delta_{x_0}=\bar\Delta_{R_0}=\bar\Delta_{x_i}=0$ and $h_{x_0},h_{R_0},h_{x_i}=0$~\cite{LeePICDC14}. 

\section{Control System Design for Full Dynamic Model}\label{sec:FDM}

The control system designed at the previous section is based on a simplifying assumption that each quadrotor can generate a thrust along any arbitrary direction instantaneously. However, the dynamics of quadrotor is underactuated since the direction of the total thrust is always parallel to its third body-fixed axis, while the magnitude of the total thrust can be arbitrarily changed. This can be directly observed from the expression of the total thrust, $u_i = -f_i R_i e_3$, where $f_i$ is the total thrust magnitude, and $R_ie_3$ corresponds to the direction of the third body-fixed axis. Whereas, the rotational attitude dynamics is fully actuated by the control moment $M_i$.

Based on these observations, the attitude of each quadrotor is controlled such that the third body-fixed axis becomes parallel to the direction of the ideal control force $u_i$ designed in the previous section within a finite time. More explicitly, the desired attitude of each quadrotor is constructed as follows. The desired direction of the third body-fixed axis of the $i$-th quadrotor, namely $b_{3_i}\in\Sph^2$ is given by
\begin{align}
b_{3_i} = -\frac{u_i}{\|u_i\|}.\label{eqn:b3i}
\end{align}
This provides two-dimensional constraint on the three-dimensional desired attitude of each quadrotor, and there remains one degree of freedom. To resolve it, the desired direction of the first body-fixed axis $b_{1_i}(t)\in\Sph^2$ is introduced as a smooth function of time~\cite{LeeLeoPICDC10}. This corresponds to controlling the additional one dimensional yawing angle of each quadrotor. From these, the desired attitude of the $i$-th quadrotor is given by
\begin{align*}
R_{i_c} = \begin{bmatrix}-\frac{(\hat b_{3_i})^2 b_{1_i}}{\|(\hat b_{3_i})^2 b_{1_i}\|}, &
\frac{\hat b_{3_i}b_{1_i}}{\|\hat b_{3_i}b_{1_i}\|},& b_{3_i}\end{bmatrix},
\end{align*}
which is guaranteed to be an element of $\SO$. The desired angular velocity is obtained from the attitude kinematics equation, $\Omega_{i_c} = (R_{i_c}^T\dot R_{i_c})^\vee\in\Re^3$.  

In the prior work described in~\cite{LeePICDC14}, the attitude of each quadrotor is controlled such that the equilibrium $R_i=R_{i_c}$ becomes exponentially stable, and the stability of the combined full dynamic model is achieved via singular perturbation theory~\cite{Kha96}. However, we can not follow such approach in this paper, as the presented adaptive control system guarantees only the asymptotical convergence of the tracking error variables due to the disturbances, thereby making is challenging to apply the singular perturbation theory. Here, we design the attitude controller of each quadrotor such that $R_i$ becomes equal to $R_{i_c}$ within a finite time via finite-time stability theory~\cite{BhaBerSJCO00,YuYuA05,WuRadAA11}. 

Define the tracking error vectors $e_{R_i},e_{\Omega_i}\in\Re^3$ for the attitude and the angular velocity of the $i$-th quadrotor as
\begin{align*}
e_{R_i} = \frac{1}{2}(R_{i_c}^T R_i -R_i^T R_{i_c})^\vee,\quad
e_{\Omega_i} = \Omega_i - R_i^T R_{i_c}\Omega_{i_c}.
\end{align*}
The time-derivative of $e_{R_i}$ can be written as~\cite{LeeLeoPICDC10}
\begin{align}
\dot e_{R_i} & = \frac{1}{2}(\trs{R_i^T R_{i_c}}I-R_i^T R_{i_c})e_{\Omega_i}\triangleq E(R_i,R_{i_c})e_{\Omega_i}.\label{eqn:eRi_dot}
\end{align}
For $0< r <1$, define $S:\Re\times\Re^3\rightarrow\Re^3$ as
\begin{align*}
S(r,y) = \begin{bmatrix}|y_1|^r\mathrm{sgn}(y_1),&
|y_2|^r\mathrm{sgn}(y_2),&
|y_3|^r\mathrm{sgn}(y_3)\end{bmatrix}^T,
\end{align*}
where $y=[y_1,y_2,y_3]^T\in\Re^3$, and $\mathrm{sgn}(\cdot)$ denotes the sign function. For positive constants $k_{R},l_{R}$, the terminal sliding surface $s_i\in\Re^3$ is designed as
\begin{align}
s_i = e_{\Omega_i} + k_{R} e_{R_i} + l_{R}S(r,e_{R_i}).\label{eqn:si}
\end{align}
We can show that when confined to the surface of $s_i\equiv 0$, the tracking errors become zero in a finite time. To reach the sliding surface, for positive constants $k_s,l_s$, the control moment is designed as
\begin{align}
M_i & = -k_s s_i - l_s S(r,s_i)+\Omega_i\times J_i\Omega_i\nonumber\\
&\quad - (k_R J_i+ l_s r J_i\mathrm{diag}_j[|e_{R_{i_j}}|^{r-1}]) E(R_i,R_{c_i}) e_{\Omega_i}\nonumber\\
&\quad  -J_i(\hat\Omega_i R_i^T R_{i_c} \Omega_{i_c} - R_i^T R_{i_c}\dot\Omega_{i_c}).\label{eqn:Mi}
\end{align}

The thrust magnitude is chosen as the length of $u_i$, projected on to $-R_ie_3$,
\begin{align}
f_i & = - u_i\cdot R_i e_3,\label{eqn:fi}
\end{align}
which yields that the thrust of each quadrotor becomes equal to its desired value $u_i$ when $R_i=R_{i_c}$.

Stability of the corresponding controlled systems for the full dynamic model can be shown by using the fact that the full dynamic model becomes exactly same as the simplified dynamic model within a finite time.


\begin{prop}\label{prop:FDM}
Consider the full dynamic model defined by \refeqn{ddotx0}-\refeqn{dotWi}. For given tracking commands $x_{0_d},R_{0_d}$ and the desired direction of the first body-fixed axis $b_{1_i}$, control inputs for quadrotors are designed as \refeqn{Mi} and \refeqn{fi}. Then, there exists controller parameters such that the tracking error variables 
$(e_{x_0},\dot e_{x_0},e_{R_0},e_{\Omega_0},e_{q_i},e_{\omega_i})$ asymptotically converge to zero, and the estimation errors are uniformly bounded.
\end{prop}
\begin{proof}
See Appendix \ref{sec:prfFDM}.
\end{proof}

This implies that the payload asymptotically follows any arbitrary desired trajectory both in translations and rotations in the presence of uncertainties. In contrast to the existing results in aerial transportation of a cable suspended load, it does not rely on any simplifying assumption that ignores the coupling between payload, cable, and quadrotors. Also, the presented global formulation on the nonlinear configuration manifold avoids singularities and complexities that are inherently associated with local coordinates. As such, the presented control system is particularly useful for agile load transportation involving combined translational and rotational maneuvers of the payload in the presence of uncertainties.



\section{Numerical Example}\label{sec:NE}
We consider a numerical example where three quadrotors ($n=3$) transport a rectangular box along a figure-eight curve. More explicitly, the mass of the payload is $m_0=1.5\,\mathrm{kg}$, and its length, width, and height are $1.0\,\mathrm{m}$, $0.8\,\mathrm{m}$, and $0.2\,\mathrm{m}$, respectively. Mass properties of three quadrotors are identical, and they are given by
\begin{gather*}
m_i=0.755\,\mathrm{kg},\quad J_i=\mathrm{diag}[0.0820,\, 0.0845,\, 0.1377]\,\mathrm{kgm^2}.
\end{gather*}
The length of cable is $l_i=1\,\mathrm{m}$, and they are attached to the following points of the payload.
\begin{gather*}
\rho_1 = [0.5,\,0,\,-0.1]^T,\\
\rho_1 = [-0.5,\,0.4,\,-0.1]^T,\quad
\rho_3 = [-0.5,\,-0.4,\,-0.1]^T\,\mathrm{m}.
\end{gather*}
In other words, the first link is attached to the center of the top, front edge, and the remaining two links are attached to the vertices of the top, rear edge (see Figure \ref{fig:DM}). 

\begin{figure}
\centerline{
	\subfigure[3D perspective]{
		\includegraphics[width=0.95\columnwidth]{Quad_N_Rgbd_Adaptive_3d0}}
}
\centerline{
	\subfigure[Top view]{
	\setlength{\unitlength}{0.1\columnwidth}\scriptsize
	\begin{picture}(9.5,3.2)(0,0)
	\put(0,0){\includegraphics[width=0.95\columnwidth]{Quad_N_Rgbd_Adaptive_3d2}}
	\put(8.4,3.2){$t=0$}
	\put(6.0,3.2){$t=3.3$}
	\put(2.2,-0.2){$t=6.6$}
	\put(0.2,2.3){$t=10$}
	\put(2.2,3.2){$t=13.3$}
	\put(5.8,-0.2){$t=16.6$}
	\put(7.9,0.4){$t=20$}
	\end{picture}}	
}
\centerline{
	\subfigure[Side view]{
		\includegraphics[width=0.95\columnwidth]{Quad_N_Rgbd_Adaptive_3d1}}
}
\caption{Snapshots of controlled maneuver (red:desired trajectory, blue:actual trajectory). A short animation illustrating this maneuver is available at {\href{http://youtu.be/nOWErfdzZLU}{http://youtu.be/nOWErfdzZLU}}.}\label{fig:SS}
\end{figure}

The desired trajectory of the payload is chosen as
\begin{align*}
x_{0_d}(t) = [1.2\sin (0.2\pi t),\, 4.2\cos (0.1\pi t),\, -0.5]^T\,\mathrm{m}.
\end{align*}
The desired attitude of the payload is chosen such that its first axis is tangent to the desired path, and the third axis is parallel to the direction of gravity, it is given by
\begin{align*}
R_{0_d}(t) = \begin{bmatrix}\frac{\dot x_{0_d}}{\|\dot x_{0_d}\|}&\frac{\hat e_3\dot x_{0_d}}{\|\hat e_3\dot x_{0_d}\|}& e_3\end{bmatrix}.
\end{align*}
Initial conditions are chosen as
\begin{gather*}
x_0(0)=[1,\,4.8,\,0]^T\,\mathrm{m},\quad v_0(0)=0_{3\times 1}\,\mathrm{m/s},\\
q_i(0)= e_3,\; \omega_i(0)=0_{3\times 1},\;
R_i(0)=I_{3\times 3},\; \Omega_i(0)=0_{3\times 1}.
\end{gather*}
The uncertainties are specified as
\begin{align*}
\Delta_{x_0}=[1,\, 3,\, -2.5]^T,\quad \Delta_{R_0}=[-0.5,\, 0.1,\, -1.5]^T,\\
\Delta_{x_i}=[0.5,\, -0.2,\, 0.3]^T,\quad \Delta_{R_i}=[0.2,\, 0.3,\, -0.7]^T.
\end{align*}


\begin{figure}
\centerline{
	\subfigure[Position of payload ($x_0$:blue, $x_{0_d}$:red)]{
		\includegraphics[width=0.48\columnwidth]{Quad_N_Rgbd_Adaptive_x0}}
	\hfill
	\subfigure[Attitude tracking error of payload $\Psi_0=\frac{1}{2}\|R_0-R_{0_d}\|^2$]{
		\includegraphics[width=0.48\columnwidth]{Quad_N_Rgbd_Adaptive_Psi0}}
}
\centerline{
	\subfigure[Link direction error $\Psi_{q_i}=\frac{1}{2}\|q_i-q_{i_d}\|^2$]{
		\includegraphics[width=0.48\columnwidth]{Quad_N_Rgbd_Adaptive_Psiq}}
	\hfill
	\subfigure[Attitude tracking error of quadrotors $\Psi_i=\frac{1}{2}\|R_i-R_{i_d}\|^2$]{
		\includegraphics[width=0.48\columnwidth]{Quad_N_Rgbd_Adaptive_Psi}}
}
\centerline{
	\subfigure[Tension at links]{
		\includegraphics[width=0.45\columnwidth]{Quad_N_Rgbd_Adaptive_Tension}}
	\hfill
	\subfigure[Control input for quadrotors $f_i,M_i$]{
		\includegraphics[width=0.48\columnwidth]{Quad_N_Rgbd_Adaptive_fM}}
}
\caption{Simulation results for tracking errors and control inputs. (for figures (c)-(f): $i=1$:blue, $i=2$:green, $i=3$:red)}\label{fig:SR}
\end{figure}

The corresponding simulation results are presented at Figures \ref{fig:SS} and \ref{fig:SR}. Figure \ref{fig:SS} illustrates the desired trajectory that is shaped like a figure-eight curve around two obstacles represented by cones, and the actual maneuver of the payload and quadrotors. Figure \ref{fig:SR} shows tracking errors for the position and the attitude of the payload, tracking errors for the link directions and the attitude of quadrotors, as well as tension and control inputs. These illustrate excellent tracking performances of the proposed control system. 


\appendix
%
%
%
%
%\paragraph{Error Dynamics}
%
%From \refeqn{ddotx0} and \refeqn{mui}, the dynamics of the position tracking error is given by
%\begin{align*}
%m_0\ddot e_{x_0} & = m_0(ge_3-\ddot x_{0_d}) + \Delta_{x_0} + \sum_{i=1}^n q_iq_i^T \mu_{i_d}+ \Delta_{x_i}^\parallel.
%\end{align*}
%From \refeqn{muid0} and \refeqn{Fd}, this can be rearranged as
%\begin{align}
%\ddot e_{x_0} & = ge_3-\ddot x_{0_d} + \frac{1}{m_0}F_d  +Y_x+\frac{1}{m_0}(\tilde\Delta_{x_0}+\sum_{i=1}^n \tilde\Delta_{x_i}^\parallel),\nonumber\\
%& = -k_{x_0}e_{x_0} - k_{\dot x_0} \dot e_{x_0} +\frac{1}{m_0}(\tilde\Delta_{x_0}+\sum_{i=1}^n \tilde\Delta_{x_i}^\parallel) +Y_x,
%\label{eqn:ddotex0}
%\end{align}
%where $\tilde\Delta_{x_i}^\parallel= \Delta_{x_i}^\parallel-\bar\Delta_{x_i}^\parallel\in\Re^{3}$ corresponds to the parallel component of the estimation error. The last term $Y_x\in\Re^3$ represents the error caused by the difference between $q_i$ and $q_{i_d}$, and it is given by
%\begin{align*}
%Y_x=\frac{1}{m_0}\sum_{i=1}^n (q_iq_i^T -I) \mu_{i_d}.
%\end{align*}
%We have $\mu_{i_d}=q_{i_d}q_{i_d}^T\mu_{i_d}$ from \refeqn{qid}. Using this, the error term can be written in terms of $e_{q_i}$ as
%\begin{align*}
%Y_x  & = \frac{1}{m_0}\sum_{i=1}^n %q_iq_i^Tq_{i_d}q_{i_d}^T\mu_{i_d} -q_{i_d}q_{i_d}^T\mu_{i_d}\\
% (q_{i_d}^T\mu_{i_d})\{(q_i^Tq_{i_d})q_i-q_{i_d}\}\nonumber\\
%% = (q_{i_d}^T\mu_{i_d}) \{q_i\times (q_i\times q_{i_d})\}
%& = -\frac{1}{m_0}\sum_{i=1}^n (q_{i_d}^T\mu_{i_d}) \hat q_i e_{q_i}.%\label{eqn:Yx}
%\end{align*}
%Using \refeqn{muid}, an upper bound of $Y_x$ can be obtained as
%\begin{align*}
%\|Y_x\| & \leq \frac{1}{m_0} \sum_{i=1}^n \|\mu_{i_d}\|\|e_{q_i}\|\leq  \sum_{i=1}^n\gamma (\|F_d\|+\|M_d\|)\|e_{q_i}\|,
%\end{align*}
%where $\gamma= \frac{1}{m_0\sqrt{\lambda_{m}[\mathcal{P}\mathcal{P}^T]}}$. From \refeqn{Fd} and \refeqn{Md}, this can be further bounded by
%\begin{align}
%\|Y_x\|  \leq & \sum_{i=1}^n \{\beta(k_{x_0}\|e_{x_0}\|+k_{\dot x_0}\|\dot e_{x_0}\|)\nonumber\\
%&\quad  + \gamma(k_{R_0} \|e_{R_0}\| + k_{\Omega_0}\|e_{\Omega_0}\|) + B\}\|e_{q_i}\|,\label{eqn:YxB}
%\end{align}
%where $\beta=m_0\gamma$, and the constant $B$ is determined by the given desired trajectories of the payload and \refeqn{D}, which defines the domain $D$ of the error variables that the presented stability proof is considered. Throughout the remaining parts of the proof, any bound that can be obtained from $x_{0_d},R_{0_d}$ or \refeqn{D} is denoted by $B$ for simplicity. In short, the position tracking error dynamics of the payload can be written as \refeqn{ddotex0}, where the error term is bounded by \refeqn{YxB}.
%
%Similarly, we find the attitude tracking error dynamics for the payload as follows. Using \refeqn{dotW0}, \refeqn{Md}, and \refeqn{mui}, the time-derivative of $J_0 e_{\Omega_0}$ can be written as
%\begin{align}
%J_0\dot e_{\Omega_0} & = (J_0e_{\Omega_0}+d)^\wedge e_{\Omega_0} - k_{R_0}e_{R_0}-k_{\Omega_0}e_{\Omega_0}\nonumber\\
%&\quad + \tilde\Delta_{R_0} + \sum_{i=1}^n \hat\rho_i R_0^T\tilde\Delta_{x_i}^\parallel+Y_R,\label{eqn:doteW0}
%\end{align}
%where $d=(2J_0-\trs{J_0}I)R_0^T R_{0_d}\Omega_{0_d}\in\Re^3$~\cite{GooLeePECC13} that is bounded, and $\tilde\Delta_{R_0}\in\Re^{3}$ denotes the estimation error given by $\tilde\Delta_{R_0} = \Delta_{R_0}-\hat\Delta_{R_0}$. The error term in the attitude dynamics of the payload, namely $Y_R\in\Re^3$ is given by
%\begin{align*}
%Y_R & = \sum_{i=1}^n\hat\rho_i R_0^T (q_iq_i^T - I)\mu_{i_d}
%= -\sum_{i=1}^n\hat\rho_i R_0^T (q_{i_d}^T \mu_{i_d}) \hat q_i e_{q_i}.%\label{eqn:YR}
%\end{align*}
%Similar with \refeqn{YxB}, an upper bound of $Y_R$ can be obtained as
%\begin{align}
%\|Y_R\|  \leq & \sum_{i=1}^n\{\delta_i (k_{x_0}\|e_{x_0}\|+k_{\dot x_0}\|\dot e_{x_0}\|)\nonumber\\
%&\quad  + \sigma_i(k_{R_0} \|e_{R_0}\| + k_{\Omega_0}\|e_{\Omega_0}\|) + B\}\|e_{q_i}\|,\label{eqn:YRB}
%\end{align}
%where $\delta_i = m_0\frac{\|\hat\rho_i\|}{\sqrt{\lambda_{m}[\mathcal{P}\mathcal{P}^T]}},\sigma_i=\frac{\delta_i}{m_0}\in\Re$. 
%
%
%Next, from \refeqn{dotwif}, the time-derivative of the angular velocity error, projected on to the plane normal to $q_i$ is given as
%\begin{align}
%-\hat q_i^2 \dot e_{\omega_i} %& = -\hat q_i^2(\dot\omega_i + (\dot q_i\cdot\omega_{i_d}) q_i
%%+(q_i\cdot\dot\omega_{i_d}) q_i+( q_i\cdot\omega_{i_d}) \dot q_i - \dot \omega_{i_d})
%& = \dot\omega +( q\cdot\omega_d) \dot q +\hat q^2 \dot \omega_d\nonumber\\
%& = -k_q e_{q_i} - k_\omega e_{\omega_i}-\frac{1}{m_il_i}\hat q_i \tilde\Delta_{x_i}^\perp.\label{eqn:dotewi}
%\end{align}
%In summary, the error dynamics of the simplified dynamic model are given by \refeqn{ddotex0}, \refeqn{doteW0} and \refeqn{dotewi}.
%
%\paragraph{Stability Proof}
%
%Define an attitude configuration error function $\Psi_{R_0}$ for the payload as
%\begin{align*}
%\Psi_{R_0} = \frac{1}{2}\trs{I-R_{0_d}^T R_0},
%\end{align*}
%which is positive-definite about $R_0=R_{0_d}$, and $\dot\Psi_{R_0} = e_{R_0}\cdot e_{\Omega_0}$~\cite{LeeLeoPICDC10,GooLeePECC13}. We also introduce a configuration error function $\Psi_{q_i}$ for each link that is positive-definite about $q_i=q_{i_d}$ as
%\begin{align*}
%\Psi_{q_i} = 1-q_{i}\cdot q_{i_d}.
%\end{align*}
%For positive constants $e_{x_{\max}}, \psi_{R_0}, \psi_{q_i}, B_\delta\in\Re$, consider the following open domain containing the zero equilibrium of tracking error variables:
%\begin{align}
%D=\{&(e_{x_0},\dot e_{x_0}, e_{R_0}, e_{\Omega_0}, e_{q_i}, e_{\omega_i},\tilde\Delta_{x_0},\tilde\Delta_{R_0},\tilde\Delta_{x_i})\nonumber\\
%& \in(\Re^3)^4\times (\Re^3\times\Re^3)^n\times (\Re^3)^2\times \Re^{3n}\,|\,\nonumber\\
%& \|e_{x_0}\|< e_{x_{\max}},\,\Psi_{R_0}< \psi_{R_0}<1,\,\Psi_{q_i}< \psi_{q_i}<1,\nonumber\\
%& \|\tilde\Delta_{x_0}\| < B_\delta,\, \|\tilde\Delta_{R_0}\|< B_\delta,\, \|\tilde\Delta_{x_i}\| < B_\delta\}.\label{eqn:D}
%\end{align}
%In this domain, we have $\|e_{R_0}\|=\sqrt{\Psi_{R_0}(2-\Psi_{R_0})} \leq \sqrt{\psi_{R_0}(2-\psi_{R_0})} \triangleq \alpha_0 < 1$, and $\|e_{q_i}\|=\sqrt{\Psi_{q_i}(2-\Psi_{q_i})} \leq \sqrt{\psi_{q_i}(2-\psi_{q_i})} \triangleq \alpha_i < 1$. It is assumed that $\psi_{q_i}$ is sufficiently small such that $n\alpha_i\beta < 1$. 
%
%We can show that the configuration error functions are quadratic with respect to the error vectors in the sense that
%\begin{gather*}
%\frac{1}{2}\|e_{R_0}\|^2 \leq \Psi_{R_0} \leq \frac{1}{2-\psi_{R_0}} \|e_{R_0}\|^2,\\
%\frac{1}{2}\|e_{q_i}\|^2 \leq \Psi_{q_i} \leq \frac{1}{2-\psi_{q_i}} \|e_{q_i}\|^2,
%\end{gather*}
%where the upper bounds are satisfied only in the domain $D$. 
%
%Define
%\begin{align*}
%\mathcal{V}_0 & = \frac{1}{2}\|\dot e_{x_0}\|^2 + \frac{1}{2}k_{x_0}\|e_{x_0}\|^2   +c_x e_{x_0}\cdot \dot e_{x_0}\\
%&\quad + \frac{1}{2}e_{\Omega_0}\cdot J_0\Omega_0 + k_{R_0}\Psi_{R_0} + c_R e_{R_0}\cdot J_0e_{\Omega_0}\\
%&\quad+ \sum_{i=1}^n \frac{1}{2}\|e_{\omega_i}\|^2 + k_q\Psi_{q_i} + c_q e_{q_i}\cdot e_{\omega_i},
%\end{align*}
%where $c_x,c_R,c_q$ are positive constants. This is composed of tracking error variables only, and we define another function for the estimation errors of the adaptive laws as
%\begin{align*}
%\mathcal{V}_a & = \frac{1}{2h_{x_0}}\|\tilde \Delta_{x_0}\|^2
%+\frac{1}{2h_{R_0}}\|\tilde \Delta_{R_0}\|^2
%+ \sum_{i=1}^n \frac{1}{2h_{x_i}} \|\tilde\Delta_{x_i}\|^2.
%\end{align*}
%The Lyapunov function for the complete simplified dynamic model is chosen as $\mathcal{V}=\mathcal{V}_0+\mathcal{V}_a$.
%
%Let $z_{x_0} = [\|e_{x_0}\|,\|\dot e_{x_0}\|]^T$, $z_{R_0}=[\|e_{R_0}\|,\|e_{\Omega_0}\|]^T$, $z_{q_i}= [\|e_{q_i}\|,\|e_{\omega_i}\|]^T\in\Re^2$. The first part of the Lyapunov function $\mathcal{V}_0$ satisfies
%\begin{align*}
%z_{x_0}^T &\underline{P}_{x_0}z_{x_0} + 
%z_{R_0}^T \underline{P}_{R_0}z_{R_0} + 
%\sum_{i=1}^n z_{q_i}^T \underline{P}_{q_i}z_{q_i}\leq\mathcal{V}_0\\
%&\leq z_{x_0}^T \overline{P}_{x_0}z_{x_0} + 
%z_{R_0}^T \overline{P}_{R_0}z_{R_0} + 
%\sum_{i=1}^n z_{q_i}^T \overline{P}_{q_i}z_{q_i},
%\end{align*}
%where the matrices $\underline{P}_{x_0},\underline{P}_{R_0},\underline{P}_{q_i},\underline{P}_{x_0},\underline{P}_{R_0},\underline{P}_{q_i}\in\Re^{2\times 2}$ are given by
%\begin{alignat*}{2}
%\underline{P}_{x_0}&=\frac{1}{2}\begin{bmatrix} k_{x_0} & -c_x\\-c_x & 1\end{bmatrix},&\;
%\overline{P}_{x_0}&=\frac{1}{2}\begin{bmatrix} k_{x_0} & c_x\\c_x & 1\end{bmatrix},\\
%\underline{P}_{R_0}&=\frac{1}{2}\begin{bmatrix} 2k_{R_0} & -c_R\overline\lambda\\-c_R\overline\lambda & \underline\lambda\end{bmatrix},&
%\overline{P}_{R_0}&=\frac{1}{2}\begin{bmatrix} \frac{2k_{R_0}}{2-\psi_{R_0}} & c_R\overline\lambda\\c_R\overline\lambda & \overline\lambda\end{bmatrix},\\
%\underline{P}_{q_i}&=\frac{1}{2}\begin{bmatrix} 2k_{q} & -c_q\\-c_q & 1\end{bmatrix},&
%\overline{P}_{q_i}&=\frac{1}{2}\begin{bmatrix} \frac{2k_{q}}{2-\psi_{q_i}} & c_q\\c_q & 1\end{bmatrix},
%\end{alignat*}
%where $\underline\lambda=\lambda_{m}[J_0]$ and $\overline\lambda=\lambda_{M}[J_0]$. If the constants $c_x,c_{R_0},c_q$ are sufficiently small, all of the above matrices are positive-definite. As the second part of the Lyapunov function $\mathcal{V}_a$ is already given as a quadratic form, it is straightforward to see that the complete Lyapunov function $\mathcal{V}$ is positive-definite and decrescent. 
%
%The time-derivative of the Lyapunov function along the error dynamics \refeqn{ddotex0}, \refeqn{doteW0}, and \refeqn{dotewi} is given by
%\begin{align}
%\dot{\mathcal{V}} & = -(k_{\dot x_0}-c_x) \|\dot e_{x_0}\|^2 
%-c_xk_{x_0} \|e_{x_0}\|^2 -c_xk_{\dot x_0} e_{x_0}\cdot \dot e_{x_0}\nonumber\\
%&\quad+ (c_x e_{x_0}+\dot e_{x_0})\cdot Y_x -k_{\Omega_0}\|e_{\Omega_0}\|^2 + c_R \dot e_R\cdot J_0e_{\Omega_0}\nonumber\\
%&\quad -c_R k_{R_0}\|e_{R_0}\|^2 + c_R e_{R_0}\cdot ((J_0e_{\Omega_0}+d)^\wedge e_{\Omega_0}-k_{\Omega_0} e_{\Omega_0})\nonumber\\
%&\quad + (e_{\Omega_0}+c_R e_{R_0}) \cdot Y_R\nonumber\\
%&\quad +\sum_{i=1}^n -(k_\omega-c_q) \|e_{\omega_i}\|^2 -c_qk_q \|e_{q_i}\|^2 -c_q k_\omega e_{q_i}\cdot e_{\omega_i}\nonumber\\
%&\quad + \frac{1}{m_0}(\dot e_{x_0}+c_xe_{x_0})\cdot (\tilde\Delta_{x_0} +  \sum_{i=1}^n \tilde\Delta_{x_i}^\parallel) \nonumber\\
%&\quad +(e_{\Omega_0}+ c_Re_{R_0})\cdot (\tilde\Delta_{R_0}+\sum_{i=1}^n \hat\rho_i R_0^T \tilde\Delta_{x_i}^\parallel)\nonumber\\
%&\quad -\parenth{\sum_{i=1}^n ( e_{\omega_i} + c_qe_{q_i})\cdot \frac{\hat q_i}{m_il_i} \tilde\Delta_{x_i}^\perp}-\frac{1}{h_{x_0}}\tilde\Delta_{x_0}\cdot \dot{\bar\Delta}_{x_0}\nonumber\\
%&\quad 
%-\frac{1}{h_{R_0}}\tilde\Delta_{R_0}\cdot\dot{\bar\Delta}_{R_0}
%-\sum_{i=1}^n\frac{1}{h_{x_i}}(\tilde\Delta_{x_i}^\parallel\cdot \dot{\bar\Delta}_{x_i}^\parallel+\tilde\Delta_{x_i}^\perp\cdot \dot{\bar\Delta}_{x_i}^\perp),
%\label{eqn:dotV0}
%\end{align}
%where the last term has been obtained using the facts that $\tilde\Delta_{x_i}^\perp\cdot \dot{\bar \Delta}_{x_i}^\parallel
%=\tilde\Delta_{x_i}^\parallel\cdot \dot{\bar \Delta}_{x_i}^\perp=0$. Substituting the adaptive laws given by \refeqn{hatDx0_dot}, \refeqn{hatDR0_dot}, and \refeqn{hatDxi_dot} into \refeqn{dotV0}, the expressions at the last four lines of \refeqn{dotV0} that are dependent of estimation errors vanish.
%
%An upper bound of the remaining expressions of $\dot{\mathcal{V}}$ can be obtained as follows.
%Since $\|e_{R_0}\|\leq 1$, $\| \dot e_{R_0}\|\leq \|e_{\Omega_0}\|$ and $\|d\|\leq B$, 
%\begin{align}
%\dot{\mathcal{V}} & \leq -(k_{\dot x_0}-c_x) \|\dot e_{x_0}\|^2 
%-c_xk_{x_0} \|e_{x_0}\|^2 -c_xk_{\dot x_0} e_{x_0}\cdot \dot e_{x_0}\nonumber\\
%&\quad+ (c_x e_{x_0}+\dot e_{x_0})\cdot Y_x -(k_{\Omega_0}-2c_R\overline\lambda)\|e_{\Omega_0}\|^2 \nonumber\\
%&\quad -c_R k_{R_0}\|e_{R_0}\|^2 + c_R(k_{\Omega_0}+B) \|e_{R_0}\|\|e_{\Omega_0}\|\nonumber\\
%&\quad + (e_{\Omega_0}+c_R e_{R_0}) \cdot Y_R\nonumber\\
%&\quad +\sum_{i=1}^n -(k_\omega-c_q) \|e_{\omega_i}\|^2 -c_qk_q \|e_{q_i}\|^2 -c_q k_\omega e_{q_i}\cdot e_{\omega_i}.
%\label{eqn:dotV2}
%\end{align}
%From \refeqn{YxB}, an upper bound of the fourth term of the right-hand side is given by
%\begin{align}
%\|&(c_x e_{x_0}+\dot e_{x_0})\cdot Y_x\|  \leq  \nonumber\\
%&\sum_{i=1}^n\alpha_i\beta( c_xk_{x_0}\|e_{x_0}\|^2 + c_x k_{\dot x_0}\|e_{x_0}\|\|\dot e_{x_0}\| +k_{\dot x_0} \|\dot e_{x_0}\|^2)
%\nonumber\\
%&+\{c_xB \|e_x\|+(\beta k_{x_0}e_{x_{\max}}+B) \|\dot e_{x_0}\| \} \|e_{q_i}\|\nonumber\\
%&+\alpha_i\gamma(c_x\|e_{x_0}\|+\|\dot e_{x_0}\|)(k_{R_0}\|e_{R_0}\|+k_{\Omega_0}\|e_{\Omega_0}\|).
%\label{eqn:YxB2}
%\end{align}
%Similarly, using \refeqn{YRB},
%\begin{align}
%\|&(c_R e_{R_0}+e_{\Omega_0})\cdot Y_R\|\leq \nonumber\\
%&\sum_{i=1}^n \alpha_i\sigma_i( c_R k_{R_0}\|e_{R_0}\|^2 + c_R k_{\Omega_0}\|e_{R_0}\|\|e_{\Omega_0}\| + k_{\Omega_0}\|e_{\Omega_0}\|^2)\nonumber\\
%&+\{c_R B\|e_{R_0}\| + (\alpha_0\sigma_ik_{R_0}+B)\|e_{\Omega_0}\|\}\|e_{q_i}\|\nonumber\\
%&+\alpha_i\delta_i(c_R \|e_{R_0}\|+\|e_{\Omega_0}\|)(k_{x_0}\|e_{x_0}\|+k_{\dot x_0}\|\dot e_{x_0}\|).
%\end{align}
%Substituting these into \refeqn{dotV2} and rearranging, 
%\begin{align}
%\dot{\mathcal{V}} \leq \sum_{i=1}^n - z_i^T W_i z_i,\label{eqn:U0}
%\end{align}
%where $z_i=[\|z_{x_0}\|,\, \|z_{R_0}\|,\,\ \|z_{q_i}\|]^T\in\Re^3$, and the matrix $W_i\in\Re^{3\times 3}$ is defined as
%\begin{align}
%W_i = 
%\begin{bmatrix} \lambda_m[W_{x_i}] & -\frac{1}{2}\|W_{xR_i}\| & -\frac{1}{2}\|W_{xq_i}\|\\
%-\frac{1}{2}\|W_{xR_i}\| & \lambda_m[W_{R_i}] & -\frac{1}{2}\|W_{Rq_i}\| \\
%-\frac{1}{2}\|W_{xq_i}\| & -\frac{1}{2}\|W_{Rq_i}\| & \lambda_m[W_{q_i}]
%\end{bmatrix},\label{eqn:Wi}
%\end{align}
%where the sub-matrices are given by
%\begin{gather*}
%W_{x_i} = \frac{1}{n}\begin{bmatrix}
%c_x k_{x_0} (1-n\alpha_i\beta) & -\frac{c_x k_{\dot x_0}}{2}(1+n\alpha_i\beta)\\
%-\frac{c_x k_{\dot x_0}}{2}(1+n\alpha_i\beta) & k_{\dot x_0}(1-n\alpha_i\beta)-c_x 
%\end{bmatrix},\\
%W_{R_i} = \frac{1}{n}\begin{bmatrix}
%c_R k_{R_0} (1-n\alpha_i\sigma_i) & -\frac{c_R }{2}(k_{\Omega_0}+B+n\alpha_i\sigma_i)\\
%-\frac{c_R }{2}(k_{\Omega_0}+B+n\alpha_i\sigma_i) & k_{\Omega_0}(1-n\alpha_i\sigma_i)-2c_R\overline\lambda
%\end{bmatrix},\\
%W_{q_i}=\begin{bmatrix}
%c_q k_q & -\frac{c_qk_\omega}{2}\\
%-\frac{c_qk_\omega}{2} & k_\omega-c_q\\
%\end{bmatrix},\\
%W_{xR_i}= \alpha_i\begin{bmatrix}
%\gamma c_x k_{R_0}+\delta_i c_Rk_{x_0} 
%& \gamma c_x k_{\Omega_0}+\delta_i k_{x_0}\\
%\gamma k_{R_0}+\delta_ic_Rk_{\dot x_0}  &  
%\gamma k_{\Omega_0} + \delta_i k_{\dot x_0}
%\end{bmatrix},\\
%W_{xq_i}= \begin{bmatrix}
%c_x B & 0\\
%\beta k_{x_0}e_{x_{\max}}+B & 0 
%\end{bmatrix},\quad
%W_{xR_i}= \begin{bmatrix}
%c_R B & 0\\
%\alpha_0\sigma_i k_{R_0} +B & 0 
%\end{bmatrix}.
%\end{gather*}
%
%%On the other hand, substituting the adaptive laws given by \refeqn{hatDx0_dot}, \refeqn{hatDxi_dot}, and \refeqn{hatDR0_dot} into \refeqn{dotV0}, the second part of $\dot{\mathcal{V}}$ that is dependent of the estimation errors is given by
%%\begin{align}
%%\mathcal{U}_a & = \tilde\Delta_{x_0} \cdot (I_{3\times 3}-\mathrm{Proj}[\bar \Delta_{x_0},\phi_{x_0}]) \phi_{x_0}\nonumber\\
%%&\quad + \sum_{i=1}^n\tilde\Delta_{x_i} \cdot (I_{3\times 3}-\mathrm{Proj}[\bar \Delta_{x_i},\phi_{x_i}])\phi_{x_i}\nonumber\\
%%&\quad + \tilde\Delta_{R_0} \cdot (I_{3\times 3}-\mathrm{Proj}[\bar \Delta_{R_0},\phi_{R_0}])\phi_{R_0}.
%%\label{eqn:Ua}
%%\end{align}
%%Note that from the definition of the projection operator given at \refeqn{Proj}, for some $\tilde\Delta,\bar\Delta,\phi\in\Re^3$, we have
%%\begin{align*}
%%&\tilde\Delta \cdot (I_{3\times 3}-\mathrm{Proj}[\bar \Delta,\phi]) \phi\\
%%&=\begin{cases}
%%0\;	&\text{if $\|\bar\Delta\| \leq B_{\delta}$}\\
%%    &\text{or $\|\bar\Delta\| = B_{\delta}$ and $\bar\Delta^T\phi \leq 0$,}\\
%%\frac{\tilde\Delta^T\bar\Delta \bar\Delta^T\phi}{\bar\Delta^T\bar\Delta}
%%&\text{if $\|\bar\Delta\| = B_{\delta}$ and $\bar\Delta^T\phi > 0$},
%%\end{cases}
%%\end{align*}
%%For the second case, we have $\tilde\Delta^T\bar\Delta=(\Delta -\bar\Delta)^T\bar\Delta \leq 0$ due to the assumption that $\|\Delta\|\leq B_{\delta}$ and $\|\bar\Delta\|=B_\delta$. This implies that $\tilde\Delta \cdot (I_{3\times 3}-\mathrm{Proj}[\bar \Delta,\phi]) \phi\leq 0$ always for any $\tilde\Delta,\bar\Delta,\phi\in\Re^3$. Therefore, by applying this to \refeqn{Ua} repeatedly, we obtain
%%\begin{align}
%%\mathcal{U}_a \leq 0.\label{eqn:Uaa}
%%\end{align}
%%Therefore, the time-derivative of the Lyapunov function satisfies
%%\begin{align}
%%\dot{\mathcal{V}} = \mathcal{U}_0+\mathcal{U}_a \leq \sum_{i=1}^n - z_i^T W_i z_i.\label{eqn:dotV}
%%\end{align}
%
%If the constants $c_x,c_R,c_q$ that are independent of the control input are sufficiently small, the matrices $W_{x_i},W_{R_i},W_{q_i}$ are positive-definite. Also, if the error in the direction of the link is sufficiently small relative to the desired trajectory, we can choose the controller gains such that the matrix $W_i$ is positive-definite, which follows that the zero equilibrium of tracking errors is stable in the sense of Lyapunov, and all of the tracking error variables $z_i$ and the estimation error variables are uniformly bounded, i.e., $e_{x_0},\dot e_{x_0},e_{R_0},e_{\Omega_0},e_{q_i},e_{\omega_i},\tilde\Delta_{x_0},\tilde\Delta_{R_0},\tilde\Delta_{x_i}\in\mathcal{L}_\infty$. These also imply that $e_{x_0},\dot e_{x_0},e_{R_0},e_{\Omega_0},e_{q_i},e_{\omega_i}\in\mathcal{L}_2$ from \refeqn{U0}, and that $\dot e_{x_0},\ddot e_{x_0},\dot e_{R_0},\dot e_{\Omega_0},\dot e_{q_i},\dot e_{\omega_i}\in\mathcal{L}_\infty$. According to Barbalat's lemma~\cite{IoaSun95}, all of the tracking error variables $e_{x_0},\dot e_{x_0},e_{R_0},e_{\Omega_0},e_{q_i},e_{\omega_i}$ and their time-derivatives asymptotically converge to zero. 
%
\subsection{Proof of Proposition \ref{prop:FDM}}\label{sec:prfFDM}



We first show that the attitude of the $i$-th quadrotor becomes exactly equal to its desired value within a finite time, i.e., $R_i(t)=R_{i_c}(t)$ for any $t\geq T$ for some $T>0$. This is achieved by finite-time stability theory~\cite{BhaBerSJCO00}. This proof is composed of two parts: (i) $s_i(t)=0$ for any $t>T_s$ for some $T_s<\infty$; (ii) when the state is confined to the surface defined by $s_i=0$, we have $e_{R_i}(t)=e_{\Omega_i}(t)=0$ for any $t>T_R$ for some $T_R<\infty$. From now on, we drop the subscript $i$ for simplicity, as the subsequent development is identical for all quadrotors.

From~\cite{LeeLeoPICDC10}, the error dynamics for $e_\Omega$ is given by
\begin{align*}
%\dot{e}_{R} & = \frac{1}{2}(\tr{R^T R_{c}}I-R^T R_{c}) e_{\Omega}\triangleq E(R,R_c)e_\Omega,\label{eqn:eRi_dot}\\
J\dot e_{\Omega} & = -\Omega\times \Omega +M +\Delta_R+ J(\hat\Omega R^T R_c \Omega_c - R^T R_c\dot\Omega_c).%\label{eqn:JeWi_dot}
\end{align*}
Substituting \refeqn{Mi},
\begin{align}
J\dot e_{\Omega} & = -k_s s - l_s S(r,s)+\Delta_R-B_\delta\frac{s}{\|s\|}\nonumber\\
&\quad - (k_R J+ l_s r J\mathrm{diag}_j[|e_{R_{j}}|^{r-1}]) E(R,R_{c})e_\Omega  .\label{eqn:JeWi_dot}
\end{align}

Let a Lyapunov function be
\begin{align*}
\mathcal{W} & = \frac{1}{2} s\cdot Js.
\end{align*}
From \refeqn{si} and \refeqn{eRi_dot}, its time-derivative is given by
\begin{align*}
\dot{\mathcal{W}} & = s\cdot\{J\dot e_\Omega + (k_R J + l_s r J\mathrm{diag}_j[|e_{R_j}|^{r-1}]) E(R,R_c) e_\Omega\}.
\end{align*}
Substituting \refeqn{JeWi_dot} and \refeqn{Mi}, and using \refeqn{Bdelta}, it reduces to
\begin{align*}
\dot{\mathcal{W}} & = s\cdot\{-k_s s -l_s S(r,s)+\Delta_R - \frac{s}{\|s\|}B_\delta\}\\
& \leq -k_s \|s\|^2 - l_s \sum_{j=1}^n |s_j|^{r+1}+B_\delta \|s\| -B_\delta \|s\| \\
& \leq -k_s\|s\|^2 - l_s \|s\|^{r+1},
\end{align*}
where the last inequality is obtained from the fact that $\|x\|^\alpha\leq \sum_{i=1}^n|x_i|^\alpha$ for any $x=[x_1,\ldots,x_n]^T$ and $0<\alpha<2$~\cite[Lemma 2]{WuRadAA11}. Therefore, 
\begin{align*}
\dot{\mathcal{W}} & \leq -\epsilon_1 \mathcal{W}  -\epsilon_2\mathcal{W}^{(r+1)/2},
\end{align*}
where $\epsilon_1=\frac{2k_s}{\lambda_{M}[J]}$ and $\epsilon_2=l_s(\frac{2}{\lambda_{M}[J]})^{(r+1)/2}$. This implies that $s(t)=0$ for any $t \geq T_s$, where the settling time $T_s$ satisfies
\begin{align*}
T_s\leq \frac{2}{\epsilon_1(1-r)}\ln \frac{\epsilon_1 \mathcal{W}(0)^{(1-r)/2}+\epsilon_2}{\epsilon_2},
\end{align*}
according to~\cite[Remark 2]{YuYuA05}.

Next, consider the second part of the proof when $s=0$. Let a configuration error function for the attitude of a quadrotor be
\begin{align*}
\Psi_R =\frac{1}{2}\trs{I-R_c^T R}.
\end{align*}
Consider a domain give by $D_R=\{(R,\Omega)\in\SO\times\Re^3\,|\, \Psi_R < \psi_R < 2\}$. It has been shown that the following inequality is satisfied in the domain,
\begin{align}
\frac{1}{2}\|e_R\|^2 \leq \Psi_R \leq \frac{1}{2-\psi_R} \|e_R\|^2.\label{eqn:PsiB}
\end{align}
Therefore, it is positive-definite about $e_R=0$. The time-derivative of $\Psi_R$ is given by $\dot\Psi_R= e_R\cdot e_\Omega$. Therefore, when $s=0$, we have
\begin{align*}
\dot\Psi_R & = 
-k_{R} \|e_{R}\|^2 - l_{R} \sum_{j=1}^n |e_{R_j}|^{r+1}\\
& \leq 
-k_{R} \|e_{R}\|^2 - l_{R} \|e_R\|^{r+1},
\end{align*}
Substituting \refeqn{PsiB}, we obtain 
\begin{align*}
\dot\Psi_R & \leq -\epsilon_3\Psi_R - \epsilon_4\Psi_R^{(r+1)/2},
\end{align*}
where $\epsilon_3=\frac{k_R}{2-\psi_R}$ and $\epsilon_4=\frac{l_R }{(2-\psi_R)^{(r+1)/2}}$.
This implies that $e_R(t)=e_\Omega(t)=0$ for any $t \geq T_R$, where the settling time $T_R$ satisfies
\begin{align*}
T_R\leq \frac{2}{\epsilon_3(1-r)}\ln \frac{\epsilon_3 \Psi_R(0)^{(1-r)/2}+\epsilon_4}{\epsilon_4}.
\end{align*}

In summary, whenever $t\geq T^*\triangleq \max  \{T_s,T_R\}$, it is guaranteed that $R_i(t)=R_{i_c}(t)$ for the $i$-th quadrotor. Next, we consider the \textit{reduced system}, which corresponds to the dynamics of the payload and the rotational dynamics of the links when $R_i(t)\equiv R_{i_c}(t)$. From \refeqn{fi} and \refeqn{b3i}, the control force of quadrotors when $R_i=R_{i_c}$ is given by
\begin{align*}
-f_i \cdot R_i e_3 = (u_i\cdot R_{c_i} e_3) R_{c_i} e_3 = 
(u_i\cdot -\frac{u_i}{\|u_i\|}) -\frac{u_i}{\|u_i\|} = u_i.
\end{align*}
Therefore, the reduced system is given by the controlled dynamics of the simplified model. 

If the controller gains $k_R,l_R,k_s,l_s$ are selected large such that $T^*$ is sufficiently small, the solution stays inside of the domain $D$, where the stability results of Proposition 1 hold, during $0\leq t<T^*$. After $t\geq T^*$, the controlled system corresponds to the controlled system of the simplified dynamic model, and from Proposition \ref{prop:SDM}, the tracking errors asymptotically coverage to zero, and the estimation error are uniformly bounded.


\bibliography{/Users/tylee/Documents/BibMaster}
\bibliographystyle{IEEEtran}


\end{document}



\end{document}

