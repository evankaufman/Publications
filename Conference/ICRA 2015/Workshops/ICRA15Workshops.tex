\documentclass[10pt]{article}
\usepackage[letterpaper,text={6.5in,8.7in},centering]{geometry}
\usepackage{epic,eepic}
\usepackage{amssymb,amsmath,times,subfigure,graphicx,theorem}
%\usepackage{warmread}
%\usepackage[all,import]{xy}
%\usepackage{eepic}
\usepackage{subfigure}
\usepackage[]{algorithm2e}
\usepackage{amsmath}
\newcommand{\argmax}{\operatornamewithlimits{argmax}}

\newcommand{\norm}[1]{\ensuremath{\left\| #1 \right\|}}
\newcommand{\bracket}[1]{\ensuremath{\left[ #1 \right]}}
\newcommand{\braces}[1]{\ensuremath{\left\{ #1 \right\}}}
\newcommand{\parenth}[1]{\ensuremath{\left( #1 \right)}}
\newcommand{\pair}[1]{\ensuremath{\langle #1 \rangle}}
\newcommand{\met}[1]{\ensuremath{\langle\langle #1 \rangle\rangle}}
\newcommand{\refeqn}[1]{(\ref{eqn:#1})}
\newcommand{\reffig}[1]{Figure \ref{fig:#1}}
\newcommand{\tr}[1]{\mathrm{tr}\ensuremath{\negthickspace\bracket{#1}}}
\newcommand{\trs}[1]{\mathrm{tr}\ensuremath{[#1]}}
\newcommand{\deriv}[2]{\ensuremath{\frac{\partial #1}{\partial #2}}}
\newcommand{\SO}{\ensuremath{\mathsf{SO(3)}}}
\newcommand{\T}{\ensuremath{\mathsf{T}}}
\renewcommand{\L}{\ensuremath{\mathsf{L}}}
\newcommand{\so}{\ensuremath{\mathfrak{so}(3)}}
\newcommand{\SE}{\ensuremath{\mathsf{SE(3)}}}
\newcommand{\se}{\ensuremath{\mathfrak{se}(3)}}
\renewcommand{\Re}{\ensuremath{\mathbb{R}}}
\newcommand{\aSE}[2]{\ensuremath{\begin{bmatrix}#1&#2\\0&1\end{bmatrix}}}
\newcommand{\ase}[2]{\ensuremath{\begin{bmatrix}#1&#2\\0&0\end{bmatrix}}}
\newcommand{\D}{\ensuremath{\mathbf{D}}}
\newcommand{\Sph}{\ensuremath{\mathsf{S}}}
\renewcommand{\S}{\Sph}
\newcommand{\J}{\ensuremath{\mathbf{J}}}
\newcommand{\Ad}{\ensuremath{\mathrm{Ad}}}
\newcommand{\intp}{\ensuremath{\mathbf{i}}}
\newcommand{\extd}{\ensuremath{\mathbf{d}}}
\newcommand{\hor}{\ensuremath{\mathrm{hor}}}
\newcommand{\ver}{\ensuremath{\mathrm{ver}}}
\newcommand{\dyn}{\ensuremath{\mathrm{dyn}}}
\newcommand{\geo}{\ensuremath{\mathrm{geo}}}
\newcommand{\Q}{\ensuremath{\mathsf{Q}}}
\newcommand{\G}{\ensuremath{\mathsf{G}}}
\newcommand{\g}{\ensuremath{\mathfrak{g}}}
\newcommand{\Hess}{\ensuremath{\mathrm{Hess}}}

\renewcommand{\baselinestretch}{1.2}
\date{}

\renewcommand{\thesubsection}{\arabic{subsection}. }
\renewcommand{\thesubsubsection}{\arabic{subsection}.\arabic{subsubsection} }

\theoremstyle{plain}\theorembodyfont{\normalfont}
\newtheorem{prob}{Question}[section]
%\renewcommand{\theprob}{\arabic{section}.\arabic{prob}}
\renewcommand{\theprob}{\arabic{prob}}

\newenvironment{subprob}%
{\renewcommand{\theenumi}{\alph{enumi}}\renewcommand{\labelenumi}{(\theenumi)}\begin{enumerate}}%
{\end{enumerate}}%


\begin{document}

\section*{ICRA15 Workshops}


\begin{itemize}
\item Robotic Vision: Challenges and Opportunities
\begin{itemize}
\item Vision goals: how many objects can I recognize and how well?
\item Robotics goals: what can we do with those objects?
\item Deep reinforcement of Learning
\begin{itemize}
\item Deep learning (2012) - 8 layer neural network
\item $0.216$ trillion sets of training data--could only be done with large computing power, linearized functions
\item Reinforcement Learning: find a control policy to optimize the ability to accomplish a goal
\item Training: guided policy search: either removing the nonlinearities or a complicated policy simplifies the process
\end{itemize}
\item The 100/100 Tracking Challenge
\begin{itemize}
\item Assume the environment and everything inside it is known
\item Goal: track 100\% of the objects with 100\% accuracy
\item Modeling non-rigid objects
\item Fairly decent tracking down to the joints of an object (hand, body)
\item Detecting articulated objects: data association challenge
\item Challenges: robust data association and occlusions for very fast motion, activity and task recognition
\end{itemize}
\item Geometry vs. Appearance
\begin{itemize}
\item Geometry: properties and relations in wold space--map representations (occupancy grids), sparse point clouds, landmark-based SLAM, dense SLAM
\item Appearance: the was something looks in sensor space (typical definitions)--feature detectors, bag of words, pixel trackers, artificial landmarks (QR codes)
\item Lots of ways to relate the two, but rarely are they considered together
\item Why? physics of the sensors that relates geometry and appearance is difficult
\item Removing illumination does great things
\end{itemize}
\item Natural Vision for Robotic Vision: Pillage or Pass Over?
\begin{itemize}
\item Mouses have much blurrier vision than humans, although they can quickly learn how to navigate and do tasks
\item There is a large disconnect between measuring vision as a whole and the actual tasks of animals
\item Cameras are getting so cheap that mostly nature-based approaches are not very useful now
\end{itemize}
\item Visual Navigation
\begin{itemize}
\item Self-driving vehicles have a perception problem
\item Occlusion is incredibly difficult
\item Signals become very difficult (waving you through an intersection)
\item Signs may be blocked, but must be understood
\item Challenges: maintaining maps, dealing with adverse weather, dealing with people, robust computer vision
\item We need object-based understanding
\end{itemize}
\end{itemize}
\newpage
\item Tutorial: Industrial Applications with Large-Scale 3D Poin Cloud Processing
\begin{itemize}
\item Basic principles, solving problems of modern sensor data processing
\item 3DTK: 3D Toolkit, OpenSource/GPL
\item How to save a point cloud?
\begin{itemize}
\item vector of (x,y,z) as a text file
\item range/intensity image: 2D array of kinect-like sensors, but there is distortion of a spherical sensor space onto a 2D image plane: projecting each 3D point to an image plane, but you cannot map a complete sphere into a 2D image
\item oc-tree: occupancy trees break down the space into blocks, where free space is removed and those being measured are split to increase resolution in this region
\item k-d tree: binary, heavily used in nearest-neighbor-like scenarios
\end{itemize}
\item
\end{itemize}
\end{itemize}

\end{document}



