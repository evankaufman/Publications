
\section*{Problem Formulation}
\begin{frame} %-----------------------------%
\frametitle{Problem Description}
\begin{itemize} 
\item Vector Measurements
	\begin{itemize} 
	\item Let $v_i^I\in\Sph^2$ be the unit vectors in the reference frame, for $i={1,2,\ldots,n}$ ($n\geq2$). 
	\vspace*{0.05cm}
	\item The rigid body is equipped with sensors to measure such reference directions: 
	{\small \begin{gather*} v_i^B(t)=R(t)^\T v_i^I\in\Sph^2. \end{gather*}} 
	\end{itemize}
\pause
\item Angular velocity Measurements\\
The angular velocity measurements is assumed to be corrupted with \Emph{fixed bias $\ga$}
	{\small \begin{gather*} \W_y=\W+\ga\in\Re^3. \end{gather*}} 	
\item Equation of Motion: $\dot{R}=R\hat{\W}$. 
\pause
	\begin{tcolorbox}
	Design observer to estimate the attitude $R$ of the rigid body based on these measurements with gyro bias.
	\end{tcolorbox}
\end{itemize} 
\end{frame}   %-----------------------------%

\subsection*{Problem Formulation}
\begin{frame} %-----------------------------%
\frametitle{Esitmate Frame \& Estimate Error}
\begin{itemize} 
\item Estimate Frame
	\begin{itemize} 
	\item Estimate attitude and bias are given by $\bar{R}\in\SO$, $\bar{\ga}\in\Re^3$.
	\vspace*{0.05cm}
	\item Express the reference direction $v_i^I$ with respect to the estimate frame:	
	{\small \begin{gather*} v_i^E=\bar{R}^\T v_i^I\in\Sph^2. \end{gather*}}
	\end{itemize}
%\vspace*{0.2cm}
\pause	
\item Estimate Error
	\begin{itemize} 	 	 	
	\item The discrepancy between the true attitude $R$ and the estimate attitude $\bar{R}$
	{\small \begin{gather*} \td{R}=\bar{R}^\T R\in\SO. \end{gather*}} 	
	\item Estimate Error Function 
	{\small \begin{gather*} \Psi_i=1-{v_i^E}^\T v_i^B\in\Re,\quad \Psi=\sum_{i=1}^nk_i\Psi_i\in\Re. \end{gather*}}. 
	\item Bias estimate error: $\td{\ga}=\ga-\bar{\ga}\in\Re^3$.	
	\end{itemize}  
\end{itemize} 
\end{frame}   %-----------------------------%


\begin{frame} %-----------------------------%
\frametitle{Complementary Filter [Mahony, 2008] }
%\begin{itemize} 
%\item Review [Mahony, 2008] 
	\begin{prop}[] %~~~~~Poposition~~~~~%
	{\small The complementary filter is defined as
		\begin{gather*} 
		\dot{\bar R}=\bar{R}[(\W_y-\bar{\ga})+k_R\tc{purple}{e_R}]^\vee,\quad \dot{\bar \ga}=-k_I\tc{purple}{e_R},\\
		\tc{purple}{e_R}=\sum_{i=1}^nk_i v_i^B\t v_i^E,
		\end{gather*}
	for positive constants $k_R,k_I\in\Re$.		
	}
		\begin{itemize} 
		\item There are 4 equilibria, given by
%		\vspace*{-0.1cm}
			{\footnotesize\begin{gather*} 
			(\bar R,\bar{\ga})\in\{(R,\ga),(UD_iU^\T R,\ga)\}.
			\end{gather*}}				
%		\vspace*{-0.5cm} 
		\item The desired equilibrium $(\bar R,\bar{\ga})=(R,\ga)$ is almost globally asymptotically stable.
%		\vspace*{0.05cm} 
		\item Convergence rate is slow when the initial state is close to undesired equilibria.		
		\end{itemize} 	
	\end{prop} 		
%\end{itemize} 
\end{frame}   %-----------------------------%


\begin{frame} %-----------------------------%
\frametitle{Alternative Form of Error Function}
\begin{itemize} 
\item Estimate Error Function
	{\footnotesize 
	\begin{gather*}  \Psi=\sum_{i=1}^nk_i(1-{v_i^E}^\T v_i^B)=\sum_{i=1}^nk_i-\tr[\bar{R}^\T\Emph{K}R],\\
	K=\sum_{i=1}^nk_iv_i^I{v_i^I}^\T=K^\T =\Emph{\sum_{i=1}^3\lam_is_is_i^\T}.
	\end{gather*}} 	
As $K$ is symmetric and positive-(semi)definite, we have
	{\footnotesize 
	\begin{gather*}  
	K=UGU^\T=\left[\begin{array}{ccc}|&|&|\\s_1& s_2& s_3\\|&|&| \end{array}\right] \left[\begin{array}{ccc}\lam_1&0&0\\0&\lam_2&0\\0&0&\lam_3\end{array}\right] \left[\begin{array}{ccc}-&s_1&-\\-& s_2& -\\-& s_3&- \end{array}\right],
	\end{gather*}}
%	\vspace*{-0.3cm} 
	where $U\in\SO$ and $G$ is defined by eigenvalues.
\pause
\item Using $\tc{purple}{b_i}=R^\T s_i$ and $\tc{purple}{\bar{b}_i}=\bar{R}^\T s_i$,
%	{\small\begin{gather*} 
%	b_i=R^\T s_i,\quad \bar{b}_i=\bar{R}^\T s_i.	
%	\end{gather*}} 	
the estimate function can be rewritten as 
	{\small\begin{gather*} 
	\Psi=\sum_{i=1}^3 \lam_i(1-\bar{b}_i^\T b_i).
	\end{gather*}}
\end{itemize} 
\end{frame}   %-----------------------------%

\begin{frame} %-----------------------------%
\frametitle{Motivation of Hybrid Observer}
\begin{itemize}
\item Necessity of Hybrid System
\begin{itemize} 
\item Equilibria configuration
	\begin{itemize} 
	\item 1 desired equilibrium: $(\bar{b}_1,\bar{b}_2,\bar{b}_3)=(b_1,b_2,b_3)$
	\item 3 undesired equilibria:
	{\small\begin{gather*} 
	(\bar{b}_1,\bar{b}_2,\bar{b}_3)=(b_1,-b_2,-b_3),\qquad\qquad\\
	(\bar{b}_1,\bar{b}_2,\bar{b}_3)=(-b_1,b_2,-b_3),\qquad\qquad\\
	(\bar{b}_1,\bar{b}_2,\bar{b}_3)=(-b_1,-b_2,b_3).\qquad\qquad
	\end{gather*}}
	\end{itemize} 
\vspace*{0.1cm} 
\pause
\item Error functions for hybrid observer
	\begin{itemize} 
	\item Nominal Mode: same as [Mahony, 2008].
	\vspace*{0.05cm} 
	\item \Emph{Expelling Mode}: 2 expelling modes are designed to steer the attitude configuration away from 3 undesired equilibria.	 
	\end{itemize}
\vspace*{0.2cm} 	
\item Topological obstruction is avoided due to discontinuity.	
%\vspace*{0.1cm} 
%\item Starting from nominal mode, switching logic is triggered when the estimate attitude is near to undesired equilibria.	
\end{itemize} 
\end{itemize} 
\end{frame}   %-----------------------------%

\begin{frame} %-----------------------------%
\frametitle{Hybrid Observer -- Switching Logic}
\only<1>{
\begin{minipage}[t]{0.50\linewidth}
	\begin{itemize} 
	\vspace*{-0.1cm}
	\item Nominal Mode
 	\begin{figure} %figure%
 	\hspace*{-1.5cm}   
    \includegraphics[width=1.1\columnwidth]{4_Nominal.pdf}
  	\hspace*{1.5cm}   
  	\end{figure}	
  	\vspace*{0.05cm}
	\end{itemize}	
\end{minipage}\hfill
%\begin{minipage}[t]{0.50\linewidth}
%	\begin{itemize} 
%	\item Expelling Mode
%	\begin{figure} %figure%
%	\hspace*{-0.8cm}
%    \centering\includegraphics[width=1.1\columnwidth]{4_Expell.pdf}
%    \hspace*{0.8cm}
%  	\end{figure}	
%	\end{itemize}
%\end{minipage}
}
\only<2>{
\begin{minipage}[t]{0.50\linewidth}
	\begin{itemize} 
	\item Nominal Mode
%	\vspace*{-0.1cm}
 	\begin{figure} %figure%
 	\hspace*{-1.5cm}   
    \includegraphics[width=1.1\columnwidth]{4_Nominal.pdf}
  	\hspace*{1.5cm}   
%  	\vspace*{0.4cm}
  	\end{figure}
	\end{itemize}	
\end{minipage}\hfill
\begin{minipage}[t]{0.50\linewidth}
	\begin{itemize} 
	\item Expelling Mode
	\begin{figure} %figure%
	\hspace*{-0.8cm}
    \centering\includegraphics[width=1.1\columnwidth]{4_Expell.pdf}
    \hspace*{0.8cm}
    \vspace*{0.4cm}
  	\end{figure}	
	\end{itemize}
\end{minipage}
}
\end{frame}   %-----------------------------%

%\begin{frame} %-----------------------------%
%\frametitle{Hybrid Observer}
%\begin{itemize} 
%\item Once the estimate attitude is away from undesired equilibria, error function is switched back to nominal mode.
%\pause
%\vspace*{0.1cm} 
%\item Topological obstruction is avoided due to discontinuity. 
%\vspace*{0.1cm} 
%\item Hysteresis-based switching for robustness of the system. 
%
%\end{itemize} 
%\end{frame}   %-----------------------------%


\section*{Attitude Observer with Global Asymptotic Stability}
\begin{frame} %-----------------------------%
\frametitle{Hybrid Observer Design}
\begin{prop}[] %~~~~~Poposition~~~~~%
{\small The hybrid attitude observer is defined as
\vspace*{-0.15cm}
	\begin{gather*} 
	\dot{\bar R}=\bar{R}[(\W_y-\bar{\ga})+k_R\tc{purple}{e_H}]^\vee,\quad \dot{\bar \ga}=-k_I\tc{purple}{e_H},\quad e_H=\sum_{i=1}^n\lambda_i e_{H_i},
	\end{gather*}
\vspace*{-0.15cm}
where the $i$-th hybrid innovation term $e_{H_i}$ is given by	
	\begin{align*}
  	e_{H_1}&=\begin{cases}
               b_1\t\bar{b}_1 & \text{if}\quad \m=\Ra,\Rb,\\
               -\be(b_3\t\bar{b}_1) & \text{if}\quad \m=\Rc,\\
            \end{cases} 
  	\\
  	e_{H_2}&=\begin{cases}
               b_2\t\bar{b}_2 & \text{if}\quad \m=\Ra,\Rc,\\
               -\be(b_3\t\bar{b}_2) & \text{if}\quad \m=\Rb,\\
            \end{cases}
  	\\          
  	e_{H_3}&= b_3\t\bar{b}_3  \qquad\quad\text{for}\quad \m=\Ra,\Rb,\Rc. 
	\end{align*}
Then, the desired equilibrium $(\bar{R},\bar{\ga})=(R,\ga)$ is globally asymptotically stable, and the number of discrete jumps is finite.
}
\end{prop} 
%\begin{itemize} 
%\item 
%\end{itemize} 
\end{frame}   %-----------------------------%

\begin{frame} %-----------------------------%
\frametitle{Hysteresis Switching}
\begin{itemize} 
\item Switching Logic
	\begin{itemize} 
	\item The observer is switched into the new mode that yields the minimum error function
		{\small\begin{gather*}
		\rho(\bar R)=\min_{\m\in\M}\{\Psi_\m(\bar R)\},\quad \text{for } \M\in\{\Ra,\Rb,\Rc\}.
		\end{gather*}}	
	\item $\Psi_m:$ current error function; $\rho:$ minimum error function.		
	\vspace*{0.1cm}	
	\item It may cause \Emph{chattering} due to measurement noise if the switch events happen whenever $\Psi_\m>\rho$. 			
	\end{itemize}
\pause	
\vspace*{0.2cm}
\item Hysteresis Gap
	\begin{itemize}
	\item A positive hysteresis gap: $\delta\in\Re$. 
	\vspace*{0.1cm}
	\item Switching is triggered if  \tc{purple}{$\Psi_\m-\rho\geq\delta$}. 
	\vspace*{0.1cm}
	\item Increase the \Emph{robustness} of the observer.
	\end{itemize} 
\end{itemize} 		
\end{frame}   %-----------------------------%




\section*{Numerical Analysis}
\begin{frame} %-----------------------------%
\frametitle{Numerical Integraton}
\begin{itemize} 
\item Numerical Integration Error
	\begin{itemize} 
	\item $\SO$ has \Emph{higher cumulative error} during numerical integration since to the group operation is not additive. Let $R_1,R_2\in\SO$,
		{\small\begin{gather*}
		R_1+R_2\notin\SO,\qquad R_1R_2\in\SO.
		\end{gather*}}	
	\item Conventional Runge-Kutta
		{\small\begin{gather*}
		\bar{R}_{n+1}'=\bar{R}_{n}+hK.
		\end{gather*}}				
	\end{itemize}
\pause	
\item Solutions
	\begin{itemize}
	\item Reduce error tolerance numerically: \\
	\begin{alltt}
	{\scriptsize OdeOption=odeset('RelTol',1e-10,'AbsTol',1e-10)}.
	\end{alltt}
	\item \Emph{Geometric numerical integrator}: preserve the structure of $\SO$ during simulation.
	\end{itemize} 
\end{itemize} 		
\end{frame}   %-----------------------------%


\begin{frame} %-----------------------------%
\frametitle{Geometric Numerical Integrator}
\only<1>{
	\begin{itemize} 
	\item Geometric Runge-Kutta (2nd order)
	\begin{itemize}
	\item Intermediate update:\\
	{\small $\bar{R}_{n+1}'=\exp^{hK_{{\bar R}_1}}\bar{R}_{n}$,~ $K_{{\bar R}_1} =(\bar{R}_{n}\bar{\Omega}_n)^\wedge$.}\\
	{\small $\bar\ga_{n+1}'=\bar\gamma_n+hK_{{\bar\gamma}_1}$,~ $K_{{\bar\gamma}_1}=-k_Ie_{H_n}(R_n,\bar{R}_n)$.}\\
%		{\small\begin{gather*} 
%		\bar{R}_{n+1}'=\exp^{hK_{{\bar R}_1}}\bar{R}_{n},\qquad K_{{\bar R}_1} =(\bar{R}_{n}\bar{\Omega}_n)^\wedge\qquad\\ 
%		\bar\ga_{n+1}'=\bar\gamma_n+hK_{{\bar\gamma}_1},\qquad K_{{\bar\gamma}_1}=-k_Ie_{H_n}(R_n,\bar{R}_n)
%		\end{gather*}}
\vspace*{0.2cm}		
	\item Full-step update:	
	{\small\begin{gather*} 
	\bar{R}_{n+1}=\exp^{\frac{1}{2}h(K_{{\bar R}_1}+K_{{\bar R}_2})}\bar{R}_{n},\quad   \bar{\ga}_{n+1}=\bar\ga_n+\frac{1}{2}h(K_{\bar{\ga}_1}+K_{\bar{\ga}_2}).
	\end{gather*}}
	\end{itemize}				
	\end{itemize}
\vspace*{4.2cm}			
}
\only<2>{	
	\begin{itemize} 
	\item Geometric Runge-Kutta (2nd order)
	\begin{itemize}
	\item Intermediate update:\\
	{\small $\bar{R}_{n+1}'=\exp^{hK_{{\bar R}_1}}\bar{R}_{n}$,~ $K_{{\bar R}_1} =(\bar{R}_{n}\bar{\Omega}_n)^\wedge$.}\\
	{\small $\bar\ga_{n+1}'=\bar\gamma_n+hK_{{\bar\gamma}_1}$,~ $K_{{\bar\gamma}_1}=-k_Ie_{H_n}(R_n,\bar{R}_n)$.}\\
\vspace*{0.2cm}		
	\item Full-step update:	
	{\small\begin{gather*} 
	\bar{R}_{n+1}=\exp^{\frac{1}{2}h(K_{{\bar R}_1}+K_{{\bar R}_2})}\bar{R}_{n},\quad   \bar{\ga}_{n+1}=\bar\ga_n+\frac{1}{2}h(K_{\bar{\ga}_1}+K_{\bar{\ga}_2}).
	\end{gather*}}
\vspace*{-0.4cm}	 
	\begin{figure} %figure%
	\hspace*{-0.8cm}
    \centering\includegraphics[width=0.45\columnwidth]{fig_4_ni150.pdf}
    \hspace*{0.8cm}
  	\end{figure}	
\vspace*{-0.3cm}  		
  	\item Result: eliminate cumulative error without reducing tolerance error.	
	\end{itemize}				
	\end{itemize}
}		
\end{frame}   %-----------------------------%





\begin{frame} %-----------------------------%
\frametitle{Numerical Examples}
	\begin{itemize} 
	\item Without Gyro Bias
		\begin{figure}
		\centering	
		\hspace*{-0.1\columnwidth}
		\subfigure{\includegraphics[width=0.29\columnwidth]{fig_4_att.pdf}}
%		\hspace*{-0.03\columnwidth}
		\subfigure{\includegraphics[width=0.29\columnwidth]{fig_4_sw.pdf}}
%		\hspace*{-0.03\columnwidth}			
		\subfigure{\includegraphics[width=0.29\columnwidth]{fig_4_eR.pdf}}								
		\end{figure}	
	\item With Gyro Bias
		\begin{figure}
		\centering	
		\hspace*{-0.11\columnwidth}
		\subfigure{\includegraphics[width=0.27\columnwidth]{fig_4bia_att.pdf}}
		\hspace*{-0.048\columnwidth}
		\subfigure{\includegraphics[width=0.27\columnwidth]{fig_4bia_sw.pdf}}
		\hspace*{-0.048\columnwidth}			
		\subfigure{\includegraphics[width=0.27\columnwidth]{fig_4bia_eR.pdf}}	
		\hspace*{-0.048\columnwidth}			
		\subfigure{\includegraphics[width=0.27\columnwidth]{fig_4bia_bE.pdf}}						
		\end{figure}		
	\end{itemize}	
\end{frame}   %-----------------------------%

\begin{frame} %-----------------------------%
\frametitle{Experimental Validation} 
	\begin{itemize} 
	\item Fully-Actuated Hexdrotor
	\end{itemize}
		\begin{figure}
		\centering	
%		\hspace*{-0.05\columnwidth}
		\subfigure{\includegraphics[height=0.22\columnwidth]{fig_4_hex1.pdf}}
		\hspace*{0.05\columnwidth}
		\subfigure{\includegraphics[height=0.22\columnwidth]{fig_4_hex2.pdf}}					
		\end{figure}
	\begin{figure} %figure%
%	\hspace*{-0.8cm}
    \centering\includegraphics[width=0.45\columnwidth]{fig_4_hex3.pdf}
%    \hspace*{0.8cm}
  	\end{figure}		
\end{frame}   %-----------------------------%


\begin{frame} %-----------------------------%
\frametitle{Experimental Result} 
\centerline{
\includemovie[poster]{80mm}{60mm}{Defense.mov}
}
\end{frame}   %-----------------------------%

%\subsection*{}
\begin{frame} %-----------------------------%
\frametitle{Conclusions \& Contributions}
\begin{itemize} 
\item Conclusions
	\begin{itemize} 
	\item Substantially faster convergence rate.
	\vspace*{0.15cm}
	\item Initial guess does NOT have to be close to the true value of the state. 	
	\vspace*{0.15cm} 
	\item Corrupted angular velocity is compensated.
	\vspace*{0.15cm}
	\item The estimated attitude evolves on the special group.
	\end{itemize} 
\vspace*{0.3cm}
\pause
\item Contribution
	\begin{itemize} 
	\item First attitude observer with global asymptotic stability.
	\vspace*{0.15cm} 
	\item Hybrid attitude observer with straightforward and intuitive switching logic.
%	\vspace*{0.1cm}
%	\item  
%	\vspace*{0.1cm} 	 
	\end{itemize}  
\end{itemize} 
\vspace*{0.6cm}
\end{frame}   %-----------------------------%


%\begin{frame} %-----------------------------%
%\frametitle{}
%\begin{itemize} \item \vspace*{0.1cm} \end{itemize} 
%\end{frame}   %-----------------------------%
%\begin{tcolorbox}\end{tcolorbox}