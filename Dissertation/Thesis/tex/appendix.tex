% !TEX root = ../thesis-sample.tex
\appendix
\doublespacing
\chapter{Proof of Proposition 1}

\paragraph{Unnormalized Reduced Map Inverse Sensor Model}
For the reduced map of the $l$-th ray, the unnormalized probability that corresponds to the terms without $\eta_{t,l}$ in \refeqn{InvSenModWithProbDens} is given by
\begin{align}
\tilde P&(\mathbf{r}_{l,k}|z_{t,l},X_{1:t},Z_{1:t-1})= \sum_{r\in\mathcal{M}_{k}} p(z_{t,l}|r,X_t) P(r|X_{1:t-1},Z_{1:t-1}).\label{eqn:tildePlk}
\end{align}
Recall $\mathcal{M}_k$ corresponds to the set of maps where the $k$-th cell is occupied. Define a subset $\mathcal{N}_{i,k}\subset \mathcal{M}_k$ for $1\leq i\leq k$ be the set of maps where the $i$-th cell is the first occupied cell. More explicitly, 
\begin{align*}
\mathcal{N}_{i,k} & = \{ r\in\mathcal{M}_k| \mathbf{r}_{l,i+}\}= \{ r\in\mathcal{M}_k| \mathbf{r}_{l,1}=0,\ldots, \mathbf{r}_{l,i-1}=0, \mathbf{r}_{l,i}=1,
\mathbf{r}_{l,k}=1\}.
\end{align*}
Then, $\mathcal{M}_k$ can be written as $\mathcal{M}_k =\bigcup_{i=1}^{k} \mathcal{N}_{i,k}$. Using this, the summation over $\mathcal{M}_k$ at \refeqn{tildePlk} can be decomposed of the summation over each $\mathcal{N}_{i,k}$ to obtain
\begin{align*}
\tilde P(\mathbf{r}_{l,k}|z_{t,l},X_{1:t},Z_{1:t-1})= \sum_{i=1}^k\braces{\sum_{r\in\mathcal{N}_{i,k}} p(z_{t,l}|r,X_t) P(r|X_{1:t-1},Z_{1:t-1})}.
\end{align*}
This is motivated by the fact that the forward sensor model $p(z_{t,l}|r,X_t)$ is identical for all maps in $\mathcal{N}_{i,k}$, such that $p(z_{t,l}|r,X_t)=p(z_{t,l}|\mathbf{r}_{l,i+},X_t)$ and it is moved left of the summation to obtain
\begin{align}
\tilde P(\mathbf{r}_{l,k}|z_{t,l},X_{1:t},Z_{1:t-1})= \sum_{i=1}^k \braces{p(z_{t,l}|\mathbf{r}_{l,i+},X_t) \sum_{r\in\mathcal{N}_{i,k}} P(r|X_{1:t-1},Z_{1:t-1})}.\label{eqn:tildePlk0}
\end{align}
The last term of the above expression corresponds to the a priori probability of $\mathcal{N}_{i,k}$. When $i<k$, it is given by
\begin{align}
\sum_{r\in\mathcal{N}_{i,k}}  &P(r|X_{1:t-1},Z_{1:t-1}) 
\nonumber\\&= \bigg\{\prod_{j=0}^{i-1}P(\bar{\mathbf{r}}_{l,j}|X_{1:t-1},Z_{1:t-1})\bigg\}
P({\mathbf{r}}_{l,i}|X_{1:t-1},Z_{1:t-1})P({\mathbf{r}}_{l,k}|X_{1:t-1},Z_{1:t-1}),\label{eqn:PNik}
\end{align}
and when $i=k$, 
\begin{align}
\sum_{r\in\mathcal{N}_{k,k}} & P(r|X_{1:t-1},Z_{1:t-1})= \bigg\{\prod_{j=0}^{k-1}P(\bar{\mathbf{r}}_{l,j}|X_{1:t-1},Z_{1:t-1})\bigg\}P({\mathbf{r}}_{l,k}|X_{1:t-1},Z_{1:t-1}).\label{eqn:PNkk}
\end{align}
Substituting \refeqn{PNik} and \refeqn{PNkk} into \refeqn{tildePlk0}, we obtain \refeqn{Unnormalized}.



\paragraph{Complement of the Unnormalized Reduced Map Inverse Sensor Model}
An analytic expression for the complement of the unnormalized inverse sensor model is also required  for obtaining the normalizer of the $l$-th ray at time $t$, namely $\eta_{t,l}$. Let $\bar{\mathcal{M}}_k$ be the set of maps where the $k$-th cell is unoccupied, i.e., $\bar{\mathcal{M}}_k = \{ r\in\{0,1\}^{n_{r,l}}\,|\, \mathbf{r}_{l,k}=0\}$. Similar to \refeqn{tildePlk},
\begin{align}
\tilde P&(\bar{\mathbf{r}}_{l,k}|z_{t,l},X_{1:t},Z_{1:t-1})= \sum_{r\in\bar{\mathcal{M}}_{k}} p(z_{t,l}|r,X_t) P(r|X_{1:t-1},Z_{1:t-1}).\label{eqn:tildePbarlk}
\end{align}

Let $\bar{\mathcal{N}}_{i,k}\subset \bar{\mathcal{M}}_k $ for $1\leq i\leq n_{r,l}$ and $i\neq k$ be the set of maps where the $i$-th cell is the first occupied cell. More explicitly, 
\begin{align*}
\bar{\mathcal{N}}_{i,k}=\{r\in\bar{\mathcal{M}}_k\,|\, \mathbf{r}_{l,1}=0,\ldots,\mathbf{r}_{l,i-1}=0,
\mathbf{r}_{l,i}=1,\mathbf{r}_{l,k}=0\}.
\end{align*}
Then, we have $\bar{\mathcal{M}}_k=\bigcup_{\substack{i=1\\i\neq k}}^{n_{r,l}} \bar{\mathcal{N}}_{i,k}$. Note that the forward sensor model $p(z_{t,l}|r,X_t)$ is identical for any maps in $\bar{\mathcal{N}}_{i,k}$, such that $p(z_{t,l}|r,X_t)=p(z_{t,l}|\mathbf{r}_{l,i+},X_t)$. Similar to \refeqn{tildePlk0},
\begin{align}
\tilde P&(\bar{\mathbf{r}}_{l,k}|z_{t,l},X_{1:t},Z_{1:t-1})= \sum_{\substack{i=1\\i\neq k}}^{n_{r,l}} \braces{p(z_{t,l}|\mathbf{r}_{l,i+},X_t) \sum_{r\in\bar{\mathcal{N}}_{i,k}} P(r|X_{1:t-1},Z_{1:t-1})},\label{eqn:tildePbarlk0}
\end{align}
where the last term corresponds to the a priori probability of $\bar{\mathcal{N}}_{i,k}$. When $i<k$, it is given by
\begin{align}
\sum_{r\in\bar{\mathcal{N}}_{i,k}} & P(r|X_{1:t-1},Z_{1:t-1}) \nonumber\\&= \bigg\{\prod_{j=0}^{i-1}P(\bar{\mathbf{r}}_{l,j}|X_{1:t-1},Z_{1:t-1})\bigg\} P({\mathbf{r}}_{l,i}|X_{1:t-1},Z_{1:t-1})P(\bar{\mathbf{r}}_{l,k}|X_{1:t-1},Z_{1:t-1}),\label{eqn:PbarNik1}
\end{align}
and when $k<i$,
\begin{align}
\sum_{r\in\bar{\mathcal{N}}_{i,k}} & P(r|X_{1:t-1},Z_{1:t-1}) = \bigg\{\prod_{j=0}^{i-1}P(\bar{\mathbf{r}}_{l,j}|X_{1:t-1},Z_{1:t-1})\bigg\}
P({\mathbf{r}}_{l,i}|X_{1:t-1},Z_{1:t-1}).\label{eqn:PbarNik2}
\end{align}
Substituting \refeqn{PbarNik1} and \refeqn{PbarNik2} into \refeqn{tildePbarlk0}, we obtain
\begin{align}
\tilde P &(\bar{\mathbf{r}}_k|z_{t,l},X_{1:t},Z_{1:t-1})
=P(\bar{\mathbf{r}}_{l,k}|X_{1:t-1},Z_{1:t-1})\nonumber\\
&\quad\times \bigg[\sum_{i=1}^{k-1}\bigg\{\prod_{j=0}^{i-1}P(\bar{\mathbf{r}}_{l,j}|X_{1:t-1},Z_{1:t-1})\bigg\} p(z_{t,l}|\mathbf{r}_{l,i+},X_t)P(\mathbf{r}_{l,i}|X_{1:t-1},Z_{1:t-1})\bigg]
\nonumber
\\
&\quad
+
\bigg[\sum_{i=k+1}^{n_{r,l}+1}\bigg\{\prod_{j=0}^{i-1}P(\bar{\mathbf{r}}_{l,j}|X_{1:t-1},Z_{1:t-1})\bigg\} p(z_{t,l}|\mathbf{r}_{i+},X_t)P(\mathbf{r}_{l,i}|X_{1:t-1},Z_{1:t-1})\bigg],\label{eqn:tildePbar}
\end{align}
where $p(z_{t,l}|\mathbf{r}_{(n_r+1)+},X_t)$ corresponds to the forward sensor model of maximum reading and $P(\mathbf{r}_{l,n_{r,l}+1}|X_{1:t-1},Z_{1:t-1})=1$ for convenience for the empty map case.

\paragraph{Normalizer}
We have 
\begin{align*}
P(\mathbf{r}_{l,k}|z_{t,l},X_{1:t},Z_{1:t-1})+
P(\bar{\mathbf{r}}_{l,k}|z_{t,l},X_{1:t},Z_{1:t-1})=1.
\end{align*}
Since they share the same normalizer $\eta_{t,l}$, this implies 
\begin{align*}
\eta_{t,l}=[\tilde P(\mathbf{r}_{l,k}|z_{t,l},X_{1:t},Z_{1:t-1})+
\tilde P(\bar{\mathbf{r}}_{l,k}|z_{t,l},X_{1:t},Z_{1:t-1})]^{-1}.
\end{align*}
Substituting \refeqn{Unnormalized} and \refeqn{tildePbar}, and rearranging, we obtain \refeqn{allEta}.





