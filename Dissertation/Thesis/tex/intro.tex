% !TEX root = ../thesis-sample.tex


\chapter{Introduction} \label{chap:intro}

This dissertation explores three key aspects of mobile robotics: occupancy grid mapping, autonomous exploration, and cooperation among robot teams. Several novel contributions are presented on these topics for complex and dynamic situations.


\section{Motivation and Goals}

The study of robotics has expanded tremendously in recent years as robots have replaced humans for dangerous, difficult, and repetitive tasks. Certain objectives, such as search-and-rescue, surveillance, and cleaning, require some level of autonomy, particularly with mobile robots exploring new and uncertain environments. While these tasks provide significant value to our society, they present nontrivial challenges for the robots, particularly in mapping, exploration, and cooperation.


\subsection{Building Probabilistic Maps}

The first topic this dissertation addresses is building accurate probabilistic maps, known as \emph{mapping}. Map generation is crucial for simultaneous localization and mapping (SLAM) because maps provide important information for the robot to determine its pose (localization), which includes its position and attitude. Mapping is also an integral part of vital mobile robot objectives, such as collision avoidance and trajectory planning. In short, mapping is of fundamental importance in mobile robotics.

Therefore, the map \emph{quality} deserves much attention. Robotics is inherently \emph{probabilistic}; no sensor reading or action from an actuator is deterministic. All robotic processes have some degree of uncertainty, and the mapping must account for this. Onboard sensors serve to improve the understanding of surrounding spaces, but the stochastic properties of those sensors must be reflected in the stochastic properties of the map. Therefore, we propose to develop a probabilistic map that accounts for the history of measurements and poses, and their associated stochastic properties. Here, each element of the map holds a probability based on measured data. In turn, this probability can be applied to tasks such as collision avoidance and predicting future map outcomes.


\subsection{Exploration in Uncertain Spaces}

The second topic of this dissertation focusses on the motions robots must choose to maximize their knowledge about the map while avoiding collisions, known as \emph{autonomous exploration}. Unlike conventional motion planning, where the map of the environment is assumed known, the robot must select motions with only the current knowledge of the environment. Therefore, real-time implementation of an active motion planning scheme is necessary while the probabilistic map is changing.

There are several important aspects to consider in autonomous exploration. Most importantly, the robot motion must be constrained to collision-free paths. The probabilistic map provides collision-free regions for this objective, and it also serves as a graph to determine an ordered set of collision-free waypoints. Additionally, the uncertainty of the map must be considered in the policy governing robotic motion. The method of decreasing map uncertainty can be computationally-expensive, so careful consideration with respect to algorithm complexity is vital. Furthermore, the efficiency of robot motions should be considered to avoid unnecessary travel distances. These objectives must all be achieved for an effective autonomous exploration policy.

%We proposed an autonomous exploration approach based on maximizing map information gain. Shannon's entropy serves as measure of map uncertainty to be minimized. This exploration scheme is formulated as a constrained optimization problem, where the robot predicts how much information will be acquired from various possible future actions, while also considering travel distance.

\subsection{Cooperation Among Autonomous Systems}

The final topic of this dissertation is cooperation among members of a robot team. All robots share the mutual objective of learning the map as quickly as possible. Each robot must select actions that consider the other team members. These considerations involve avoiding collisions among members, and avoiding actions that provide too much coverage of one region of the map while another region remains uncertain. Much like single-vehicle autonomous exploration, the multi-vehicle problem requires a computationally-inexpensive algorithm to handle numerous possible robot motions. These nontrivial goals can be handled by extending single-vehicle autonomous exploration with a bidding-based approach.

Multi-vehicle autonomous exploration can be extended further for surveillance purposes with autonomous patrol. The goal is to periodically capture many regions within a large space. Frequently, the robots are subject to dynamic obstacles (e.g. a walking person) or a changing environment (e.g. moving a chair), so the robots must account for these as well. Furthermore, an effective patrol policy can decrease the number of robots required to patrol an environment. The multi-vehicle autonomous exploration can be integrated with probabilistic map degradation to achieve autonomous patrol.

\section{Literature Review}

In this section, we discuss the existing approaches to occupancy grid mapping, autonomous exploration, and multi-vehicle cooperation. Several shortcomings with existing approaches are highlighted, which are addressed in the dissertation contributions.

\subsection{Occupancy Grid Mapping}

Here we describe the existing approaches to building probabilistic occupancy grid maps. Though several variations have been proposed, no approach uses the stochastic properties of depth sensors directly to obtain an exact Bayesian solution that can be computed in real-time.

\paragraph{Inverse Sensor Model} The key to generating a probabilistic occupancy grid map is known as the \emph{inverse sensor model}~\cite{ThrBurFox05}. The main idea is that a depth sensor has a probability density (e.g. Gaussian) that describes the stochastic properties of a sensor reading (e.g. a LIDAR range) with respect to the expected value of a reading (e.g. distance to a wall). This probability density is referred to as the \emph{forward sensor model}, as it relates a depth measurement conditioned on the occupancy of the map. 

In contrast, the inverse sensor model relates the occupancy of the map conditioned on the depth measurement. This problem is solved using Bayes' theorem. However, the occupancy grid map properties yield a complicated Bayesian probability; the computational cost has exponential complexity with respect to the number of grid cells captured within the depth sensor limits. It was previously-believed that this restriction makes solving the inverse sensor model in real-time impossible. This motivated approximate or estimated solutions, described next.

\paragraph{Approximated Function for the Inverse Sensor Model} Moravec and Elfes originally proposed a grid cell-based mapping for sonar sensors in~\cite{MorElf85}, and the probabilistic properties of the occupancy grid were formalized in~\cite{MorElf85,Elf89}. This approach models measurements with Gaussian probability density, although the occupancy of grid cells is not exactly known. Instead, the probability of grid cells are estimated with a Gaussian-like \emph{approximate function}. The authors show that occupancy probability converges to a final value following a Bayesian framework, though there is no conclusive proof that this probability is accurate. Nevertheless, this approach garnered much attention because it could efficiently produce robotic maps that nicely-resembled the occupancy of spaces around a robot.

Several other research studies have applied approximate functions in place of the inverse sensor model. For example, \cite{ChoLynHutKanBurKavThr05} proposed a simplified continuous function composed of lines patched together near a measurement reading. Furthermore, variations of approximate smoother functions in~\cite{and09,PirRutBisSch11,KhoElb12} contain Gaussian-like terms to apply occupancy grid mapping to 3D environments with more modern sensors. Numerous research studies have used the proposed approximate functions with questionable accuracy.

\paragraph{Learning the Inverse Sensor Model} The approximate nature of the inverse sensor model motivated a solution using machine learning. The main idea is to simulate maps, robot poses, and measurements with simulated noised based on the sensor stochastic properties. Then, an approximation the inverse sensor model is learned using an expectation maximization algorithm~\cite{Thr01,ThrBurFox05}. Unlike approximate functions based on intuition, the machine learning approach is based on the physical properties of sensors. However, this approach is undesirable in practice due to complexities associated to implementing a learning algorithm. For example, the accuracy of such inverse sensor models strongly depends on the samples selected for learning, but it is unclear how to select those samples, or how many samples are required to obtain a reasonable approximation. Furthermore, it is challenging to apply any learning algorithm over the large dimensional space composed of maps, poses, and measurements.

\paragraph{Log-Odds Ratio} Both approaches described above use the log-odds ratio to update grid cell probabilities. The log-odds ratio formulation is a popular framework for updating binary random variables with static state within a Bayesian filter. The main idea is that instead of multiplying terms from prior and current time steps, the properties of logarithms allow these terms to be simply added, while avoiding some truncation issues associated with probabilities close to $0$ or $1$~\cite{ThrBurFox05}. This approach is also popular because the inverse sensor model needs not consider prior probabilities. Instead, these approaches typically consider fixed uniform initial probabilities of all grid cells, which makes the formulation simpler.

However, the log-odds ratio formulation makes an assumption that is not consistent with the occupancy grid mapping problem. The approach yields a simplified solution by assuming that the probability of a measurement is independent of past measurements and robot poses. This assumption is frequently violated when past measurements and poses indicate the occupancy of other cells between the robot and the cell in question. As such, this approach neglects potentially important information when considering grid cell occupancy probability.

In short, other existing solutions to occupancy probability are inexact, and involve a potentially-harmful log-odds ratio assumption. These shortcomings motivate an exact solution to occupancy grid mapping that can be computed in real-time.

%It was previously-believed that the exact solution to occupancy grid mapping was impossible in real-time.



\subsection{Autonomous Exploration}

The next topic this dissertation addresses is autonomous exploration. The vast majority of work in simultaneous localization and mapping (SLAM) deals only with the aspect of estimating the environment and the poses of vehicles. These approaches are passive in the sense that SLAM is performed on incoming sensor measurements from vehicles following an arbitrary path. As such, human teleoperation and monitoring are often essential to guide the vehicles safely through unknown surroundings. Therefore, it is desirable that an active motion planning scheme is developed and integrated with SLAM such that vehicles are able to determine their path without human intervention, and explore unknown areas autonomously. The two approaches used to solve this problem are described next.

\paragraph{Frontier-Based Autonomous Exploration}

The most popular approach to solve the autonomous exploration problem is known as frontier-based exploration, originally-proposed in~\cite{Yam97,Yam98} for 2D applications, and was later extended to 3D with a visibility metric in~\cite{SawKriSri09} and~\cite{ZhuDinLinWu15,SenWan16,KleDor13} using the popular Octomap representation (a variation on occupancy grid mapping). The main idea is that robots toward the border between certain and uncertain space, referred to as a frontier. Then the robot takes depth measurement at this frontier, thus pushing back the boundary. This process is repeated until the map is well-known. Frontier-based exploration assumes that repeatedly moving toward the closest frontier and taking measurements are the best actions to gain new information about the map. However, these systematic actions, based on ad-hoc rules for what constitutes a frontier, do not consider the future uncertainty of the probabilistic map. Frontier-based approaches provide an intuitive solution, but lack optimality in map uncertainty or exploration time.

\paragraph{Entropy-Based Autonomous Exploration}
Entropy-based approaches address the suboptimal nature of frontier-based approaches by selecting robotic actions designed to decrease minimize map uncertainty~\cite{StaGriBur05,BurMooStaSch05}. These approaches use a measure of uncertainty known as Shannon's entropy, which becomes smaller as cell probabilities approach $0$ or $1$ (becoming more certain). Existing entropy-based approaches simulate possible future measurements from various locations with so-called ``hallucination measurements,'' which typically corresponds to expected depth measurements from the robot to the closest grid cell that is possibly occupied. Then, the robot analyzes how this measurement would affect the map uncertainty. However, this approach assumes that expected map entropy is equivalent to map entropy from expected measurements. This is not the case in general, however, as there is typically a nonzero probability that the hallucination measurement and true future measurement are not the same. Entropy-based approaches seek to optimize map uncertainty, but require large approximations which can decrease the accuracy of predictions.

In conclusion, existing autonomous exploration approaches are effective for eliminating the need for humans to control the robot. However, optimality remains quite difficult, so these approaches apply suboptimal or approximated solutions to achieve autonomous motions.


\subsection{Multi-Vehicle Cooperation}

The final topic relates to coordinating robotic efforts to build occupancy grid maps together. These concepts are further extended to autonomous cooperative patrol, where robots periodically revisit regions to monitor a large environment.

\paragraph{Cooperative Autonomous Exploration}

Multi-vehicle cooperation is essential for robots to explore large spaces autonomously so that robots coordinate their efforts to improve map information and avoid collisions. However, optimizing map coverage simultaneously from multiple vehicles becomes very complicated and computationally intractable. Instead, an auction-based approach addresses the computational restrictions effectively. Bidding-based approaches were introduced with~\cite{SimApfBurFoxMooThrYou00,DiaSte00}, and extended to several other robotic problems~\cite{GerMaj02,ZloSteDiaTha02,SarBal05,ChoBruHow09} exploring decentralization and task allocation with varying robot capabilities. In~\cite{SimApfBurFoxMooThrYou00}, a central executive assigns tasks to robots for maximizing overall utility of the group. This is accomplished by minimizing a coverage overlap among the members. Robots submit bids based on expected coverage of uncertain spaces and travel distance. The robot winning the auction is awarded the task, and the remaining bids are discounted to prevent robots from covering the same area, thereby coordinating the mapping efforts. Bidding-based approaches simplify multi-vehicle cooperation in a computationally-efficient manner, and integrate effectively with the autonomous mapping and exploration approaches proposed in this dissertation. 

\paragraph{Autonomous Patrol}

Robotic surveillance and cleaning are examples of tasks that require periodic observations of the same regions. Several patrolling approaches have been proposed in recent years to address this need. A popular approach is decomposing the reachable space into Voronoi diagrams~\cite{KolCar08,PorRoc10,PipChrWei13}. In~\cite{KolCar08}, edges between nodes of the Voronoi graph are heavily sensed to track intruders. In~\cite{PorRoc10}, regions of the Voronoi diagram are assigned to different robots, and issues with robots having underperforming surveillance capabilities are addressed in~\cite{PipChrWei13}. These approaches assume the map is well-known in advance, and require a Voronoi graph be generated from the map.

Other patrol approaches create new metrics about the environment. Cell periodicity is estimated in~\cite{KraFenCieDonDuc14,KraFenHanDuc16} by predicting and observing the long-term dynamics of objects at the same locations on an occupancy grid. This work is extended to exploring and navigating a human-populated environment for long-term autonomy. A bio-inspired approach is proposed in~\cite{ZhaXia11}, where robots cooperate by dropping digital pheromones based on the importance of events, where the pheromones weaken over time to prioritize future tasks. Another approach is formulating patrol as an optimization problem with respect to robot idleness in~\cite{YanZha16}. The prior work in autonomous patrol demonstrates that periodic actions can be incentivized with a metric to be optimized.


\section{Outline of Dissertation}

ADD ONCE COMPLETE

\section{Contributions}

There are three key contributions in this dissertation. The first two are novel solutions that find exact occupancy probabilities and predicted entropies, respectively. These contributions improve the quality and efficiency of autonomous mapping and exploration of uncertain environments for single-vehicle missions. The third contribution extends these concepts to multi-vehicle scenarios, where multiple members of a team work together to autonomously explore and patrol large environments.




\subsection{Summary of Contributions}

The first contribution is an exact Bayesian solution to occupancy grid mapping. The proposed approach considers that occupied spaces occlude measurement behind them, which allows certain mapping outcomes to be systematically grouped together. The exact solution to occupancy grid cell probability, which was previously-believed to have exponential complexity, is now possible to solve in real-time. Numerous numerical simulations and experiments demonstrate that the proposed mapping approach is effective and inexpensive in 2D and 3D scenarios.

% JINT17: This paper proposes a computationally efficient algo- rithm to construct the exact inverse sensor model. More explicitly, for a given forward sensor model defined by the probability distribution of range measurements, this algo- rithm yields the a posteriori probabilities of occupancy of all the cells within the area covered by the range sensor from the range measurements. The key idea is reducing the computational load by using the fact that if a cell is occu- pied, the occupancies of the other cells blocked by it is irrelevant to the forward sensor model, and this property is systematically utilized with various probabilistic prop- erties to derive a computationally-efficient solution to the inverse sensor model. Furthermore, the proposed approach integrates a priori probabilities of occupancy and multiple range measurements according to the Bayesian framework to obtain more accurate maps. As such, it contrasts from the existing framework based on log-odds ratios that impose an additional Markov assumption.


The second contribution is a new approach to autonomous exploration, where the robot predicts future occupancy grid map uncertainty, using a measure known as Shannon's entropy. This approach computes the expected values of grid cell entropies from potential future measurement rays with a novel approach to predict future measurement probabilities. These predictions are used to evaluate the benefit of possible robot actions to maximize map information gain. Then, we formulate autonomous exploration as an optimization problem to maximize an objective function that includes map uncertainty and travel costs. The probabilistic map is used for collision-avoidance and motion planning. These processes are designed for real-time applications, where the robot makes decisions as it learns information about the surrounding environment.

The final contribution coordinates the mapping and exploration efforts of multiple robots together. The mapping process is multi-threaded such that multiple members update the same occupancy grid. The exploration algorithm follows a bidding-based structure. Robots compete for tasks in a series of auctions, where the winning bid of one auction affects the bids of subsequent auctions, thereby coordinating the robot efforts. This cooperative approach is computationally-efficient in large environments, and follows a receding-horizon framework, such that map expected information gains are updated as quickly as possible. The proposed coordination is extended to autonomous patrol, where occupancy grid map probabilities are degraded over time, which promotes the robots to revisit areas periodically.

\subsection{List of Publications}

The following is a list of publications. These include research included in this dissertation, as well as contributions in control, data association, and estimation.

% TODO: update this list

\begin{itemize}
	\item E. Kaufman, K. Takami, T. Lee, and Z. Ai, ``Autonomous Exploration with Exact Inverse Sensor Models,'' \textit{Journal of Intelligent }\&\textit{ Robotic Systems}, 2016, submitted.
	\item E. Kaufman, T. Lee, and Z. Ai, ``Autonomous Exploration by Expected Information Gain from Probabilistic Occupancy Grid Mapping,'' \textit{IEEE International Conference on Simulation, Modeling, and Programming for Autonomous Robots}, San Francisco, December 2016, accepted.
	\item E. Kaufman, T. Lee, Z. Ai, and I. S. Moskowitz, ``Bayesian Occupancy Grid Mapping via an Exact Inverse Sensor Model,'' \textit{Proceedings of the American Control Conference}, Boston, July 2016, pp. 5709-5715.
	\item E. Kaufman, T. A. Lovell, and T. Lee, ``Nonlinear Observability Measure for Relative Orbit Determination with Angles-Only Measurements,'' \textit{The Journal of the Astronautical Sciences}, 63(1): pp. 60-80, 2016, doi: 10.1007/s40295-015-0082-9.
	\item E. Kaufman, T. A. Lovell, and T. Lee, ``Minimum Uncertainty JPDA Filters and Coalescence Avoidance for Multiple Object Tracking,'' \textit{The Journal of the Astronautical Sciences}, 63(4): pp. 308-334, 2016, doi: 10.1007/s40295-016-0092-2.
	\item T. Wu, E. Kaufman, and T. Lee, ``Globally Asymptotically Stable Attitude Observer on SO(3),'' \textit{Proceedings of the 54th IEEE Conference on Decision and Control}, pp. 2164-2168, Osaka, Japan, December 2015.
	\item E. Kaufman, T. A. Lovell, and T. Lee, ``Nonlinear Observability Measure for Relative Orbit Determination with Angles-Only Measurements,'' \textit{Proceedings of the 25th AAS/AIAA Space Flight Mechanics Meeting}, Williamsburg, VA, Jan. 2015, AAS 15-451.
	\item E. Kaufman, T. A. Lovell, and T. Lee, ``Minimum Uncertainty JPDA Filter and Coalescence Avoidance Performance Evaluations,'' \textit{Proceedings of the 25th AAS/AIAA Space Flight Mechanics Meeting, Williamsburg}, VA, Jan. 2015, AAS 15-432.
	\item E. Kaufman, K. Caldwell, D. Lee, and T. Lee, ``Design and Development of a Free-Floating Hexrotor UAV for 6-DOF Maneuvers,'' \textit{Proceedings of the IEEE Aerospace Conference}, Mar. 2014, ASC 14-2527.
	\item E. Kaufman, T. A. Lovell, and T. Lee, ``Optimal Joint Probabilistic Data Association Filter Avoiding Coalescence in Close Proximity,'' \textit{Proceedings of the European Control Conference}, pp. 2709-2714, June 2014.
	\vspace*{0.5cm}
\end{itemize}







