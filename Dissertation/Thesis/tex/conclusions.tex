\chapter{Conclusions}\label{chap:Conclusions}

\section{Summary of Contributions}
In this dissertation, we propose several contributions in probabilistic occupancy grid mapping and autonomous exploration in 2D and 3D environments in single- and multi-vehicle scenarios. Computational efficiency and numerical accuracy are held paramount in derivations and algorithmic structure. Several numerical examples and experiments demonstrate the efficacy of each contribution.

The primary contribution is an exact and efficient solution to probabilistic occupancy grid mapping. Previously believed too computationally expensive, the exact solution can now be applied in real-time to 2D and 3D environments. The proposed novel Bayesian solution, referred to as the inverse sensor model, relies on grouping several possible map combinations with identical forward sensor models, rather than considering each map combination individually. This approach exploits properties of occupancy grid maps, and follows a recursive algorithm that avoids repeated calculations efficiently.

The solution to probabilistic occupancy grid mapping naturally extends to predicting future maps. The proposed autonomous exploration scheme capitalizes on map predictions using Shannon's entropy as an uncertainty measure. With the novel proposed approach, probabilities of future map measurements can be predicted, which can be used in conjunction with the inverse sensor model to predict future map entropy. Robots select poses to minimize expected entropy, equivalently maximizing map information gain. The algorithm is designed to be applied in real-time such that robots can handle uncertain environments without prior information.

The contributions in probabilistic occupancy grid mapping and autonomous exploration are extended in several ways. Probabilistic mapping and autonomous exploration are extended from 2D to 3D with two approaches: projecting a 3D map onto a 2D plane for exploration, or using the full 3D map for entropy predictions. This concept is also extended to cooperative multi-vehicle autonomous exploration, where a bidding-based framework coordinates robotic efforts to understand the map. The approach follows a receding horizon, which improves collision-avoidance and substantially reduces unnecessary robotic actions. Furthermore, this approach is extended to autonomous patrol, where an exponential map degradation incentivizes robot team members to periodically revisit areas. These extensions to occupancy grid mapping and autonomous exploration are demonstrated with numerous simulations and experimental results.

\section{Future Directions}

There are several possible future directions with this research. Most importantly, localization, where the robot determines its own position and attitude, should be included in conjunction with the proposed mapping approach, thereby completing the full simultaneous localization and mapping (SLAM) problem. Secondly, the cost map portion of Dijkstra's search can be expensive, particularly with multiple vehicles in a large space, or in 3D environments. A more efficient approach that provides collision-free distance costs during autonomous exploration could improve computational speed for autonomous exploration. Finally, enhancements in memory allocation and distributed computing are required for these algorithms to be applied in a decentralized manner for systems with more autonomy.
 


