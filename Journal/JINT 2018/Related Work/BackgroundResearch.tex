\documentclass[review]{elsarticle}

\usepackage{lineno,hyperref}
\modulolinenumbers[5]

\journal{Journal of Robotics and Autonomous Systems}

%%%%%%%%%%%%%%%%%%%%%%%
%% Elsevier bibliography styles
%%%%%%%%%%%%%%%%%%%%%%%
%% To change the style, put a % in front of the second line of the current style and
%% remove the % from the second line of the style you would like to use.
%%%%%%%%%%%%%%%%%%%%%%%

%% Numbered
%\bibliographystyle{model1-num-names}

%% Numbered without titles
%\bibliographystyle{model1a-num-names}

%% Harvard
%\bibliographystyle{model2-names.bst}\biboptions{authoryear}

%% Vancouver numbered
%\usepackage{numcompress}\bibliographystyle{model3-num-names}

%% Vancouver name/year
%\usepackage{numcompress}\bibliographystyle{model4-names}\biboptions{authoryear}

%% APA style
%\bibliographystyle{model5-names}\biboptions{authoryear}

%% AMA style
%\usepackage{numcompress}\bibliographystyle{model6-num-names}

%% `Elsevier LaTeX' style
\bibliographystyle{elsarticle-num}
%%%%%%%%%%%%%%%%%%%%%%%

\begin{document}

\begin{frontmatter}

\title{Background Research on Patrol}
%\tnotetext[mytitlenote]{Fully documented templates are available in the elsarticle package on \href{http://www.ctan.org/tex-archive/macros/latex/contrib/elsarticle}{CTAN}.}

%% Group authors per affiliation:
\author[GWU]{Evan Kaufman\corref{mycorrespondingauthor}}

\author[GWU]{Kuya Takami}

\author[NRL]{Zhuming Ai}

\author[GWU]{Taeyoung Lee}
%% or include affiliations in footnotes:
%\author[mymainaddress,mysecondaryaddress]{Elsevier Inc}
%\ead[url]{www.elsevier.com}

\fntext[GWU]{Department of Mechanical and Aerospace Engineering\\
The George Washington University, Washington, DC 20052\\
Email: \{evankaufman, kuya, tylee\}@gwu.edu}
\fntext[NRL]{Information Management \& Decision Architectures
\\ U.S. Naval Research Laboratory, Washington, DC 20375}


%\author[mysecondaryaddress]{Global Customer Service}
\cortext[mycorrespondingauthor]{Corresponding author}
%\ead{support@elsevier.com}

%\address[mymainaddress]{1600 John F Kennedy Boulevard, Philadelphia}
%\address[mysecondaryaddress]{360 Park Avenue South, New York}



\end{frontmatter}

\begin{itemize}
	\item Survey
	\begin{itemize}
		\item A Survey on Multi-robot Patrolling Algorithms~\cite{PorRoc11}
		\begin{itemize}
			\item Refs 2,3,4 use unweighted or weighted graphs to visit various environment nodes using local or global information
			\item Ref. 5 uses minimal-cost cyclic patrol paths (Hamilton cycles) at an optimal frequency, and independent of robot number (robust, best for largely-connected closed-loop graphs)
			\item Ref. 6 is like 5 except it uses a TSP cycle and segmenting the environment into different partitioned regions for patrolling (best for long corridors)
			\item Ref. 7 uses reinforcement learning to select actions to maximize a long-term performance criterion
			\item Ref. 8 has auctions between negotiating agents to visit random vertices to minimize visits to the same vertex
			\item Ref. 9 considers the length of patrolling roots for negotiations, but is only tested in open space
			\item Ref. 12 uses a neural net for patrol patterns, where a threat mode is activated when an intruder is detected
			\item Ref. 13 has robots drop pheromones that evaporate to indicate time since a robot has visited
			\item Ref. 14 focusses on catching intruders with various algorithms
			\item Refs 15,16 present a centralized and efficient algorithm for multi-level partitioning of the space to assign different regions to different agents
		\end{itemize}
	\end{itemize}
	\item Cell periodicity
	\begin{itemize}
		\item Spectral Analysis for Long-Term Robotic Mapping~\cite{KraFenCieDonDuc14}, Persistent Localization and Life-long Mapping in Changing Environments using the Frequency Map Enhancement~\cite{KraFenHanDuc16}
		\begin{itemize}
			\item Long-term dynamics of objects are predicted by re-observing the same locations on an occupancy grid using the periodicity of individual cells
			\item Experiment with exploring and navigating a human-populated environment for long-term autonomy
			\item Uses the \emph{FreMEn map} which provided probabilities as a function of time based on re-observations of each cell
		\end{itemize}
	\end{itemize}
	\item Voronoi Diagrams
	\begin{itemize}
		\item Extracting Surveillance Graphs from Robot Maps~\cite{KolCar08}
		\begin{itemize}
			\item Create surveillance graphs from occupancy grids: breaking down environment into partitions and tracking intruders through regions
			\item Voronoi Diagram extracted from 2D maps
			\item Use \emph{blocking}: detect intruders passing through a heavily-sensed edge
			\item Use \emph{contractions}: graph modifiers to use less robots for the same surveillance tasks
		\end{itemize}
		\item MSP Algorithm Multi-Robot Patrolling based on Territory Allocation using Balanced Graph Partitioning~\cite{PorRoc10}
		\begin{itemize}
			\item Ref. 16 of survey paper
			\item Known environment
			\item Assigns different regions (subgraphs) to different robots
			\item Deterministic
		\end{itemize}
		\item Performance Based Task Assignment in Multi-Robot Patrolling~\cite{PipChrWei13}
		\begin{itemize}
			\item Distribute patrol areas among robot team members
			\item Addresses issue when some team members patrol inefficiently
			\item Monitor robot performance and dynamically reassigns tasks from poorly-performing members
			\item Assume map provided in advance
			\item Partition using same Voronoi approach as Portugal and Rocha
		\end{itemize}	
	\end{itemize}
	\item Other Metrics
	\begin{itemize}
		\item A Patrolling Scheme in Wireless Sensor and Robot Networks~\cite{ZhaXia11}
		\begin{itemize}
			\item Use pheromones to show where robots were last, but does not reference Ref. 13 of survey paper
		\end{itemize}
		\item Multi-robot patrol: A distributed algorithm based on expected idleness~\cite{YanZha16}
		\begin{itemize}
			\item Decentralized patrol algorithm along a graph to minimize idleness
			\item Assumes a graph is already known, possibly from a Voronoi Diagram
		\end{itemize}
	\end{itemize}
\end{itemize}
% NOTE: at GW, look up "degradation" and "patrol" together (maybe "occupancy grid map")

%Example reference: \cite{KauTakAiLee17}.

\section*{References}

\bibliography{../../../BibSources}% master source for all publications

\end{document}