\documentclass[11pt]{article}
\usepackage[letterpaper,text={6.5in,8.6in},centering]{geometry}
\usepackage{amssymb,amsmath,times,url}
\usepackage{xr,color}
\usepackage{hyperref}


\externaldocument[]{JIRS16_rev1}

\newcommand{\norm}[1]{\ensuremath{\left\| #1 \right\|}}
\newcommand{\bracket}[1]{\ensuremath{\left[ #1 \right]}}
\newcommand{\braces}[1]{\ensuremath{\left\{ #1 \right\}}}
\newcommand{\parenth}[1]{\ensuremath{\left( #1 \right)}}
\newcommand{\pair}[1]{\ensuremath{\langle #1 \rangle}}
\newcommand{\met}[1]{\ensuremath{\langle\langle #1 \rangle\rangle}}
\newcommand{\refeqn}[1]{(\ref{eqn:#1})}
\newcommand{\reffig}[1]{Fig. \ref{fig:#1}}
\newcommand{\tr}[1]{\mathrm{tr}\ensuremath{\negthickspace\bracket{#1}}}
\newcommand{\trs}[1]{\mathrm{tr}\ensuremath{[#1]}}
\newcommand{\deriv}[2]{\ensuremath{\frac{\partial #1}{\partial #2}}}
\newcommand{\SO}{\ensuremath{\mathsf{SO(3)}}}
\newcommand{\T}{\ensuremath{\mathsf{T}}}
\renewcommand{\L}{\ensuremath{\mathsf{L}}}
\newcommand{\so}{\ensuremath{\mathfrak{so}(3)}}
\newcommand{\SE}{\ensuremath{\mathsf{SE(3)}}}
\newcommand{\se}{\ensuremath{\mathfrak{se}(3)}}
\renewcommand{\Re}{\ensuremath{\mathbb{R}}}
\newcommand{\aSE}[2]{\ensuremath{\begin{bmatrix}#1&#2\\0&1\end{bmatrix}}}
\newcommand{\ase}[2]{\ensuremath{\begin{bmatrix}#1&#2\\0&0\end{bmatrix}}}
\newcommand{\D}{\ensuremath{\mathbf{D}}}
\newcommand{\Sph}{\ensuremath{\mathsf{S}}}
\renewcommand{\S}{\Sph}
\newcommand{\J}{\ensuremath{\mathbf{J}}}
\newcommand{\Ad}{\ensuremath{\mathrm{Ad}}}
\newcommand{\intp}{\ensuremath{\mathbf{i}}}
\newcommand{\extd}{\ensuremath{\mathbf{d}}}
\newcommand{\hor}{\ensuremath{\mathrm{hor}}}
\newcommand{\ver}{\ensuremath{\mathrm{ver}}}
\newcommand{\dyn}{\ensuremath{\mathrm{dyn}}}
\newcommand{\geo}{\ensuremath{\mathrm{geo}}}
\newcommand{\Q}{\ensuremath{\mathsf{Q}}}
\newcommand{\G}{\ensuremath{\mathsf{G}}}
\newcommand{\g}{\ensuremath{\mathfrak{g}}}
\newcommand{\Hess}{\ensuremath{\mathrm{Hess}}}
\newcommand{\refprop}[1]{Proposition \ref{prop:#1}}

\newcommand{\RNum}[1]{\uppercase\expandafter{\romannumeral #1\relax}}
\newcommand{\RI}{\text{\RNum{1}}}
\newcommand{\RII}{\text{\RNum{2}}}
\newcommand{\RIII}{\text{\RNum{3}}}

\newenvironment{correction}{\begin{list}{}{\setlength{\leftmargin}{1cm}\setlength{\rightmargin}{1cm}}\vspace{\parsep}\item[]``}{''\end{list}}


\newcommand{\EditTL}[1]{{\color{red}\protect #1}}


\begin{document}

%\pagestyle{empty}

\section*{Response to the Reviewers' Comments for JINT-D-16-00659}

The authors would like to thank the reviewers for their thoughtful comments, which are aimed toward improving the quality of the paper and the clarity of the results. In accordance with the comments and suggestions, the paper has been revised as follows. 


\subsection*{Reviewer 1}
\begin{itemize}
\item The authors have payed attention to suggestions in my previous review. I do not have new questions to him.
\end{itemize}

The authors appreciate your questions and suggestions, which have improved the paper quality.

\subsection*{Reviewer 2}

\setlength{\leftmargini}{0pt}

\begin{enumerate}\setlength{\itemsep}{2\parsep}

\item {\itshape Reviewer \#2: Under the exact probabilistic occupancy grid mapping framework, how to propose a simple controller to complete the control task such as stabilization or tracking problem?}

The focus of this paper is probabilistic mapping and autonomous exploration, not control system design for stabilization or tracking. In fact, the exploration algorithm serves to determine an optimal trajectory for the robot as it gathers knowledge about its environment, and as such, it deals with a guidance problem. Designing a specific control system to follow this trajectory may be achieved by a wide range of controllers, and it is out of scope of this paper. 

To clarify this, the end of Section 5 is modified as follows. 

\begin{correction}The starting and ending positions and attitudes may be constrained, and polynomials patched together for long trajectories share a common position and velocity with respect to time. Then, the robot tracks this trajectory until the robot falls within acceptable thresholds of the final optimal pose. For generality of the proposed exploration algorithm, any position controller yielding robotic motion that follows this desired trajectory may be selected. Once the robot completes this motion, the entire process is repeated.\end{correction}


\item {\itshape Reviewer \#2: I suggest add a simple application to the visual servoing control design problem of robotic system by using the grid mapping inverse sensor models as an additional experiment to highlight its utility.}

Visual servoing involves using visual information to control the pose of a robot precisely relative to a target object, such as controlling the pose of a vehicle relative to a landmark. While this important field supports numerous applications in estimation and control of robotic systems, it is distant from the submitted manuscript, focused on probabilistic mapping and motion planning. 

Certainly, one may apply both visual servoing and the proposed mapping algorithm together, as a control scheme and a motion planning scheme, respectively. However, as discussed above, control system design is beyond the focus of the manuscript, and illustrating the efficacy of the proposed exploration algorithm with another type of control system will not support the main contribution of the submitted manuscript. 

Instead, the following is added to the end of Section 3.3 of the revised manuscript. 

\begin{correction}This proposed mapping approach follows a Bayesian framework that focusses on occupied and free space, not tracking particular systems or mapping features, such as visual servoing [17]. The proposed algorithm may be combined with feature-based approaches for object tracking or stabilization purposes, with the added challenge of associating occupied spaces with features.\end{correction}


%[1] Chen H, Wang C, Liang Z, et al. Robust practical stabilization of nonholonomic mobile robots based on visual servoing feedback with inputs saturation[J]. Asian Journal of Control, 2014, 16(3): 692-702.
%
%[2] Chen H, Ding S, Chen X, et al. Global finite-time stabilization for nonholonomic mobile robots based on visual servoing[J]. International Journal of Advanced Robotic Systems, 2014, 11(11): 180.
%
%[3] Chen H, Zhang J, Chen B, et al. Global practical stabilization for non-holonomic mobile robots with uncalibrated visual parameters by using a switching controller[J]. IMA Journal of Mathematical Control and Information, 2013: dns044.


\end{enumerate}
















\end{document}
